\documentclass[mathNotesPreamble]{subfiles}
\begin{document}
\relscale{1.4} %TODO
\section{15.2: Limits and Continuity}

  \begin{defn*}[Limit of a Function of Two Variables]
    The function $f$ has the \textbf{limit} $L$ as $P(x,y)$ approaches $P_0(a,b)$, written
      \[\lim_{(x,y)\to (a,b)} f(x,y)=\lim_{P\to P_0}f(x,y)=L,\]
    if, given any $\eps>0$, there exists a $\delta>0$ such that
      \[\abs{f(x,y)-L}<\eps\]
    whenever $(x,y)$ is in the domain of $f$ and 
      \[0< \abs{PP_0}=\sqrt{(x-a)^2+(y-b)^2}<\delta.\]
  \end{defn*}

  \noindent
  \fbox{\parbox{0.9875\linewidth}{
    \textbf{Theorem 15.1: Limits of Constant and Linear Functions}
    
    Let $a$, $b$, and $c$ be real numbers.
    \begin{enumerate}
      \TabPositions{0.45\linewidth}
      \item Constant function $f(x,y)=c$: \tab $\ds\lim_{(x,y)\to (a,b)} c=c$
      \item Linear function $f(x,y)=x$: \tab $\ds\lim_{(x,y)\to (a,b)} x=a$
      \item Linear function $f(x,y)=y$: \tab $\ds\lim_{(x,y)\to (a,b)} y=b$
    \end{enumerate}
  }}

  \noindent
  \fbox{\parbox{0.9875\linewidth}{
    \textbf{Theorem 15.2: Limit Laws for Functions of Two Variables}
    
    Let $L$ and $M$ be real numbers and suppose $\ds\lim_{(x,y)\to (a,b)} f(x,y)=L$ and \newline $\ds\lim_{(x,y)\to (a,b)} g(x,y)=M$. Assume $c$ is constant, and $n>0$ is an integer.
    \begin{enumerate}[itemsep=0.75\baselineskip]
      \TabPositions{0.325\linewidth, 0.7\linewidth}
      \item \textbf{Sum} \tab $\ds\lim_{(x,y)\to (a,b)} \parens{f(x,y)+g(x,y)}=L+M$
      \item \textbf{Difference} \tab $\ds\lim_{(x,y)\to (a,b)} \parens{f(x,y)-g(x,y)}=L-M$
      \item \textbf{Constant multiple} \tab $\ds\lim_{(x,y)\to (a,b)} cf(x,y)=cL$
      \item \textbf{Product} \tab $\ds\lim_{(x,y)\to (a,b)} f(x,y)g(x,y)=LM$
      \item \textbf{Quotient} \tab $\ds\lim_{(x,y)\to (a,b)} \frac{f(x,y)}{g(x,y)}=\frac{L}{M}$, \tab provided $M\neq 0$
      \item \textbf{Power} \tab $\ds\lim_{(x,y)\to (a,b)} \parens{f(x,y)}^n=L^n$
      \item \textbf{Root} \tab $\ds\lim_{(x,y)\to (a,b)} \parens{f(x,y)}^{1/n}=L^{1/n}$, \tab when $L>0$ if $n$ is even.
    \end{enumerate}
  }}

  \begin{defn*}[Interior and Boundary Points]
    Let $R$ be a region in $\bbr^2$. An \textbf{interior point} $P$ of $R$ lies entirely within $R$, which means it is possible to find a disk centered at $P$ that contains only points of $R$.
    \vspace*{\baselineskip}
    
    A \textbf{boundary point} $Q$ of $R$ lies on the edge of $R$ in the sense that every disk centered at $Q$ contains at least one point in $R$ and at least one point not in $R$.
  \end{defn*}

  \begin{defn*}[Open and Closed Sets]
    A region is \textbf{open} if it consists entirely of interior points. A region is \textbf{closed} if it contains all its boundary points.
  \end{defn*}

  \noindent
  \fbox{\parbox{0.9875\linewidth}{
    \textbf{Procedure: Two-Path Test for Nonexistence of Limits}
    
    If $f(x,y)$ approaches two different values as $(x,y)$ approaches $(a,b)$ along two different paths in the domain of $f$, then $\ds\lim_{(x,y)\to (a,b)} f(x,y)$ does not exist.
  }}

  \begin{defn*}[Continuity]
    The function $f$ is continuous at the point $(a,b)$ provided
    \begin{enumerate}
      \item $f$ is defined at $(a,b)$
      \item $\ds\lim_{(x,y)\to (a,b)} f(x,y)$ exists, and 
      \item $\ds\lim_{(x,y)\to (a,b)} f(x,y) = f(a,b)$.
    \end{enumerate}
  \end{defn*}

  \noindent
  \fbox{\parbox{0.9875\linewidth}{
    \textbf{Theorem 15.3: Continuity of Composite Functions}
    
    If $u=g(x,y)$ is continuous at $(a,b)$ and $z=f(u)$ is continuous at $g(a,b)$, then the composite function $z=f\parens{g(x,y)}$ is continuous at $(a,b)$.
  }}

  

  \pagebreak
  
\end{document}