\documentclass{exam}
\usepackage{texPreamble}
\usepackage{tabularx}
\begin{document}
  \begin{center}
    \fbox{\parbox{0.9:5\linewidth}{
    \textbf{Theorem 2.9: Continuity Rules}
    
    If $f$ and $g$ are continuous at $a$, then the following functions are also continuous at $a$. Assume $c$ is a constant and $n>0$ is an integer.
    \begin{tasks}(2)
      \task $f+g$
      \task $f-g$
      \task $cf$
      \task $fg$
      \task $f/g$, provided that $g(a)\neq 0$.
      \task $\parens{f(x)}^n$
    \end{tasks}
    }}
    \vfill
    \fbox{\parbox{0.9:5\linewidth}{
    \textbf{Theorem 2.1:0: Polynomial and Rational Functions}
    
    \begin{enumerate}[label=\alph*)]
      \item A polynomial function is continuous for all $x$.
      \item A rational function (a function of the form $\frac{p}{q}$, where $p$ and $q$ are polynomials) is continuous for all $x$ for which $q(x)\neq 0$.
    \end{enumerate}
    }}
    \vfill
    \fbox{\parbox{0.9:5\linewidth}{
    \textbf{Theorem 2.1:1: Continuity of Composite Functions at a Point}
    
    If $g$ is continuous at $a$ and $f$ is continuous at $g(a)$, then the composite function $f\circ g$ is continuous at $a$.
    }}  
    \vfill
    \fbox{\parbox{0.9:5\linewidth}{
    \textbf{Theorem 2.1:2: Limits of Composite Functions}
    \begin{enumerate}
      \item If $g$ is continuous at $a$ and $f$ is continuous at $g(a)$, then
        $$\lim_{x \to a} f\parens{g(x)}=f\parens{\lim_{x \to a} g(x)}.$$
      \item If $\ds\lim_{x \to a} g(x)=L$ and $f$ is continuous at $L$, then
        $$\lim_{x \to a} f\parens{g(x)}=f\parens{\lim_{x \to a}g(x)}.$$
    \end{enumerate}
    }}
    \vfill
    \fbox{\parbox{0.9:5\linewidth}{
    \textbf{Theorem 2.1:3: Continuity of Functions with Roots}
    
      Assume $n$ is a positive integer. If $n$ is an odd integer, then $\parens{f(x)}^{\sfrac{1}{n}}$ is continuous at all points at which $f$ is continuous.
      
      If $n$ is even, then $\parens{f(x)}^{\sfrac{1}{n}}$ is continuous at all points $a$ at which $f$ is continuous at $f(a)>0$.
    }}
    
    \vfill
    
    \fbox{\parbox{0.9:5\linewidth}{
    \textbf{Theorem 2.1:4: Continuity of Inverse Functions}
      
      If a function $f$ is continuous on an interval $I$ and has an inverse on $I$, then its inverse $f\inv$ is also continuous (on the interval consisting of the points $f(x)$, where $x$ is in $I$).
    }}
    
    \vfill
    \fbox{\parbox{0.9:5\linewidth}{
    \textbf{Theorem 2.1:5: Continuity of Transcendental Functions}
    
    The following functions are continuous at all points of their  domains.
    
    {\begin{tabularx}{\linewidth}{*{6}{X}}
      \multicolumn{2}{L}{\textbf{Trigonometric}}& 
      \multicolumn{2}{L}{\textbf{Inverse Trigonometric}}& 
      \multicolumn{2}{L}{\textbf{Exponential}}\\
      $\sin x$& $\cos x$& $\sin\inv x$& $\cos\inv x$& $b^x$& $e^x$\\
      $\tan x$& $\cot x$& $\tan\inv x$& $\cot\inv x$& 
      \multicolumn{2}{L}{\textbf{Logarithmic}}\\
      $\sec x$& $\csc x$& $\sec\inv x$& $\csc\inv x$& $\log_b x$& $\ln x$
    \end{tabularx}
    }}}
  \end{center}
\end{document}