\documentclass[mathNotesPreamble]{subfiles}
\begin{document}
%\relscale{1.4} %TODO
\section{MATH 1040 Review}
  For the following functions, find their derivatives:
  \begin{tasks}[after-item-skip=\stretch{0.5}, label=](2)
    \task $y=\sqrt[7]{x^3}-\pi e^x+x^e+3e\inv[x]$
    \task $f(x)=\parens{\frac{1-\sin(x)}{1+\cos(x)}}$
    \task $g(x)=\parens{\frac{x^2+3x+1}{e^x}}$
    \task $h(y)=-5\cot\parens{3e^{4y}}+e^\pi$
  \end{tasks}
  \vspace*{\stretch{0.5}}
  Find $f''(x)$ for $f(x)=\tan(x)$
  \vspace*{\stretch{1}}
  \pagebreak
  
  Find the equation of the line tangent to $\ell(x)=x\sqrt{5-x^2}$ at the point $(1,2)$.
  \vspace*{\stretch{1}}
  
  Where is the tangent line of $u=\dfrac{1}{\sqrt x}$ parallel to the line $y=-4x-3$?
  \vspace*{\stretch{1}}
  \pagebreak

  %% 01/14/19
  \noindent
  \textbf{Note:} Limits will be on Test 1 and the final exam.
  \begin{ex*}~

  Using the graph below, evaluate each limit:
  \end{ex*}
  
  \begin{minipage}{0.275\linewidth}
    \begin{tikzpicture}
      \begin{axis}[
        axis lines=center,
        axis line style={->},
        xmin=-1.5, xmax=2.25, 
        ymin=-0.5, ymax=1.5,
        xtick={-6,-5,...,6},
        ytick={-6,-5,...,6},
        ticklabel style={font=\footnotesize,inner sep=0.5pt,fill=white,opacity=1.0, text opacity=1},
        height=0.95*2.0in, width=0.75*3.75in,
        xlabel=$x$, xlabel style={at={(ticklabel* cs:1)},anchor=north west},
        ylabel=$y$, ylabel style={at={(ticklabel* cs:1)},anchor=south west},
        every axis plot/.append style={line width=0.95pt}
        ]
        \addplot[-] expression[domain=-1:1, blue]{x^2};
        \node[anchor=south west] at(axis cs: 0.85,1.05) {$y=f(x)$};
        \addplot[-] expression[domain=1:2, blue]{0};
        \addplot[soldot] coordinates{(-1,1)(0,1)(1,0)(2,0)};
        \addplot[holdot] coordinates{(0,0)(1,1)};
      \end{axis}
    \end{tikzpicture}
  \end{minipage}%
  \begin{minipage}{0.725\linewidth}
    \begin{tasks}[after-item-skip=20pt,label=](3)
      \task $\ds\lim_{x\to -1^+} f(x)$
      \task* $\ds\lim_{x\to 2^-} f(x)$
      \task $\ds\lim_{x\to 0^-} f(x)$
      \task $\ds\lim_{x\to 0^+} f(x)$
      \task $\ds\lim_{x\to 0} f(x)$
      \task $\ds\lim_{x\to 1^-} f(x)$
      \task $\ds\lim_{x\to 1^+} f(x)$
      \task $\ds\lim_{x\to 1} f(x)$
    \end{tasks}
  \end{minipage}
  
  \vspace*{15pt}
  State the intervals of continuity on $[-1,2]$.
  \vspace*{\stretch{0.5}}
  \begin{ex*}
    Algebraically, evaluate the following limits
  \end{ex*}
  \begin{tasks}[after-item-skip=\stretch{1}, label=](2)
    \task $\ds\lim_{x\to 0}\parens{\sin^2 x+\sec x}$
    \task $\ds\lim_{y\to 0}\frac{5y^3+8y^2}{3y^4-16y^2}$
  \end{tasks}
  \vspace*{\stretch{1}}
  \pagebreak
  
  \begin{tasks}[after-item-skip=\stretch{1}, label=](2)
    \task $\ds\lim_{x\to \frac{1}{2}^-} \frac{4x-2}{\abs{2x^3-x^2}}$
    \task $\ds\lim_{x\to 0} \frac{1-\cos x}{\cos^2 x-3\cos x+2}$
    \task $\ds\lim_{x\to 0} \frac{x}{\sqrt{5x+1}-1}$
    \task $\ds\lim_{x\to 0} \frac{e^{4x}-1}{e^{x}-1}$
    \task $\ds\lim_{x\to 0} \frac{\sin(x)}{x}$
    \task $\ds\lim_{x\to 0} \frac{\tan(3x)}{5x}$
  \end{tasks}
  \vspace*{\stretch{1}}
  \pagebreak
  
  \begin{tasks}[after-item-skip=\stretch{1}, label=](2)
    \task $\ds\lim_{x\to \infty} \frac{4x^3+1}{2x^3+\sqrt{16x^6+1}}$
    \task $\ds\lim_{x\to -\infty} \frac{4x^3+1}{2x^3+\sqrt{16x^6+1}}$
    \task $\ds\lim_{x\to -\infty} \parens{x+\sqrt{x^2+2x}}$
  \end{tasks}
  \vspace*{\stretch{1}}
  \pagebreak
  
  \begin{tasks}[after-item-skip=\stretch{1}, label=](2)
    \task $\ds\lim_{t\to -2^-} \frac{t^3-5t^2+6t}{t^4-4t^2}$
    \task $\ds\lim_{t\to -2^+} \frac{t^3-5t^2+6t}{t^4-4t^2}$
    \task $\ds\lim_{x\to -\infty} \frac{3x+7}{x^2-4}$
  \end{tasks}
  \vspace*{\stretch{1}}
  \pagebreak
  
  Find the equation of the slant (oblique) asymptote of $\ds f(x)=\frac{3x^5+x^4+2x^2+1}{x^4+3}$.
  \pagebreak
  
  Find $k$ such that $f(x)$ is continuous at $x=1$:
    $$f(x)=\begin{cases}
      k\,\tan\parens{\frac{\pi x}{3}},& x\geq 1\\
      x-2,& x<1
    \end{cases}$$
  \vspace{\stretch{1}}
  
  Find $c$ such that $f(x)$ is continuous:
    $$f(x)=\begin{cases}
      \dfrac{\sin^2 3x}{x^2},& x\neq 0\\
      c,& x=0
    \end{cases}$$
  \vspace*{\stretch{1}}
  \pagebreak 
  
\textbf{$\delta \textnormal{ --- } \eps$ proofs:}

  \begin{center}
    \begin{tikzpicture}[scale=0.825]
      \begin{groupplot}[
        group style={group size=3 by 1},
        axis lines=center,
        axis line style={->},
        xmin=0, xmax=4,
        ymin=0, ymax=4,
        enlargelimits={abs=0.65},
        ticklabel style={font=\large, inner sep=0.75pt,fill=white},
	      every axis plot/.append style={line width=0.95pt}
        ]
        \nextgroupplot[xtick={2.675},ytick={2.9429},
          xticklabels={$a$},yticklabels={$L$},]
          \addplot[-] expression[domain=1.575:3.865, blue] {cot(deg(x-pi/3))+3)};
          \draw[dashed, line width=0.75pt] (axis cs: 0,2.9429) -- (axis cs: 2.675,2.9429) -- (axis cs:2.675,0);
        \nextgroupplot [xtick={2.675},
          ytick={1.7853,2.9429,3.9658},
          xticklabels={$a$},yticklabels={$L-\eps$,$L$,$L+\eps$},]
          \fill[fill=ClemsonPurple, opacity=0.25] (axis cs:0,1.7853) rectangle ++(6,2.1805);
          \addplot[-] expression[domain=1.575:3.865, blue] {cot(deg(x-pi/3))+3)};
          \draw[dashed, line width=0.5pt] (axis cs: 0,1.7853) -- (axis cs: 6,1.7853);
          \draw[dashed, line width=0.5pt] (axis cs: 0,3.9658) -- (axis cs: 6,3.9658);
          \draw[dashed, line width=0.75pt] (axis cs: 0,2.9429) -- (axis cs: 2.675,2.9429) -- (axis cs:2.675,0);
        \nextgroupplot[xtick={1.85,2.675,3.5},ytick={1.7853,2.9429,3.9658},
          xticklabels={$a-\delta$,$a$,$a+\delta$},
          yticklabels={$L-\eps$,$L$,$L+\eps$},]
          \fill[fill=ClemsonOrange, opacity=0.25] (axis cs:1.85,0) rectangle ++(1.65,6);
          \fill[fill=ClemsonPurple, opacity=0.25] (axis cs:0,1.7853) rectangle ++(6,2.1805);
          \addplot[-] expression[domain=1.575:3.865, blue] {cot(deg(x-pi/3))+3)};
          \draw[dashed, line width=0.5pt] (axis cs: 1.85,0) -- (axis cs: 1.85,6);
          \draw[dashed, line width=0.5pt] (axis cs: 3.5,0) -- (axis cs: 3.5,6);
          \draw[dashed, line width=0.5pt] (axis cs: 0,1.7853) -- (axis cs: 6,1.7853);
          \draw[dashed, line width=0.5pt] (axis cs: 0,3.9658) -- (axis cs: 6,3.9658);
          \draw[dashed, line width=0.75pt] (axis cs: 0,2.9429) -- (axis cs: 2.675,2.9429) -- (axis cs:2.675,0);
      \end{groupplot}
    \end{tikzpicture}
  \end{center}


\def\scale{0.85}
  \begin{ex*}
    Use the graph of $f$ below to find a number $\delta$ such that if $0<\abs{x-2.25}<\delta$ then $\abs{f(x)-2.159}<1$.
  \end{ex*}
\begin{flushright}
  \begin{tikzpicture}[scale=\scale]
    \begin{axis}[
      axis lines=center,
      axis line style={->},
      xmin=-0.5, xmax=3.25,
      ymin=-0.5, ymax=4,
      xtick={1.409,2.25,2.991},
      xticklabels={1.409,2.250,2.991},
      ytick={1.159,2.159,3.159},
      yticklabels={1.159,2.159,3.159},
      xticklabel style={font=\large, rotate=-25, yshift=5pt},
      yticklabel style={font=\large},
      every axis plot/.append style={line width=0.95pt}
      ]
      \addplot[-] expression[domain=-1.25:5, blue] {5/(1+3^(2.5-x))};
      \draw[dashed] (axis cs: 0,1.159) -- (axis cs: 1.409,1.159) -- (axis cs: 1.409,0);
      \draw[dashed] (axis cs: 0,3.159) -- (axis cs: 2.991,3.159) -- (axis cs: 2.991,0);
      \draw[loosely dotted, line width=1.25pt] (axis cs: 0,2.159) -- (axis cs: 2.25,2.159) -- (axis cs: 2.25,0);
    \end{axis}
  \end{tikzpicture}
\end{flushright}
\begin{ex*}
  Use the graph of $g(x)=\sqrt x+1$ to help find a number $\delta$ such that if $\abs{x-4}<\delta$ then $\ds\abs{\parens{\sqrt x+1}-3}<\frac{1}{2}$.
\end{ex*}
  \begin{tikzpicture}[scale=\scale]
    \begin{axis}[
      axis lines=center,
      axis line style={->},
      xmin=-0.25, xmax=7.25,
      ymin=-0.25, ymax=4.5,
      xtick={2.25,4,6.25},
      xticklabels={$x_1$,4,$x_2$},
      ytick={2.5,3,3.5},
      ticklabel style={font=\large, inner sep=0.75pt},
      every axis plot/.append style={line width=0.95pt}
      ]
      \addplot[-] expression[blue, samples at={0,0.05,...,7.25}] {sqrt(x)+1};
      \draw[dashed] (axis cs: 0,2.5) -- (axis cs: 2.25,2.5) -- (axis cs: 2.25,0);
      \draw[dashed] (axis cs: 0,3.5) -- (axis cs: 6.25,3.5) -- (axis cs: 6.25,0);
      \draw[loosely dotted, line width=1.25pt] (axis cs: 0,3) -- (axis cs: 4,3) -- (axis cs: 4,0);
    \end{axis}
  \end{tikzpicture}
\pagebreak

\begin{ex*}
Algebraically, prove the following limits:
\end{ex*}
\begin{enumerate}[itemsep=\stretch{1}, label=]
  \item $\ds\lim_{x\to 3}\parens{10-3x}=1$
  \item $\ds\lim_{x\to 14}\parens{2-\frac{2}{7}x}=-2$
  \item $\ds\lim_{x\to 3}\frac{x^2+x-12}{x-3}=7$
\end{enumerate}
\vfill
\pagebreak

\textbf{Rates of change}
\begin{ex*}
  Find the average rate of change of $f(x)=3x^2-4x$ over the interval $\sbrkt{-1,4}$ and the instantaneous rate of change at $x=3$.
\end{ex*}
\vfill
\textbf{Limit definition of the derivative}
Recall the following definition:
  $$f'(x)=\lim_{h \to 0}\frac{f(x+h)-f(x)}{h}$$
\begin{ex*}
  Use the limit definition of the derivative to find $f'(x)$ when $f(x)=-\frac{1}{x^2}$ and then evaluate $f'(3)$.
\end{ex*}
\vfill
\pagebreak

\begin{ex*}
  Use the limit definition of the derivative to find $f'(x)$ when $f(x)=\frac{1-x}{2x}$.
\end{ex*}
\vfill

\noindent
\begin{minipage}{0.5\linewidth}
  \begin{center}
    \begin{tabularx}{0.9\linewidth}{@{}YY@{}}\toprule
      Function& Derivative\\\midrule
      Increasing&\\
      Decreasing&\\
      Max/Min&\\
      Inflection point&\\
      Constant&\\
      Linear&\\
      Quadratic\\\bottomrule
    \end{tabularx}
  \end{center}
\end{minipage}%
\begin{minipage}{0.5\linewidth}
  A function is not differentiable wherever it has a
  \begin{enumerate}[itemsep=20pt]
    \item \fillin[][\linewidth]
    \item \fillin[][\linewidth]
    \item \fillin[][\linewidth]
  \end{enumerate}
\end{minipage}%
\pagebreak

\textbf{The Chain Rule and Product Rule}
\begin{ex*}
  Find the derivatives of the following functions
\end{ex*}
\begin{tasks}[after-item-skip=\stretch{1}, label=](2)
  \task $y=\cos\parens{2x^5+7x}$
  \task $p(x)=\sqrt2x+\sqrt{3x}$
  \task $y=x^{2e}-e^{\frac{3x-2}{x^2-3x}}$
  \task $y=f\parens{\sqrt[3]{g(4x^3)}}$
\end{tasks}
\vfill
\pagebreak

\begin{tasks}[after-item-skip=\stretch{1}, label=](2)
  \task $\dfrac{d}{dx}\sbrkt{\frac{f(x)-3g(x)}{2}}$
  \task $\dfrac{d}{dx}\sbrkt{\frac{x\sbrkt{g(x)}^2}{h(x)}}$
  \task $h(\theta)=\sqrt[3]{-\theta+\cot(9+2\theta)}$
  \task $y(\theta)=\tan^2\parens{\cot(3\theta)}$
\end{tasks}
\vfill
\pagebreak

\begin{tasks}[after-item-skip=\stretch{1}, label=](2)
  \task $y=3x^2\parens{e\inv[x]+2}^4\tan\parens{3x+2}$
  \task $h(x)=\frac{-1}{2\,\sqrt[5]{\csc^2(4x)}}$
\end{tasks}
\vfill 

\begin{ex*}
  Let $f(1)=3, f'(1)=4, g(1)=2, g'(1)=6, g(3)=5$ and $g'(3)=2$. 
  
  Now, let $H(x)=\parens{g\circ f})(x)=g\parens{f(x)}$ and find $H'(1)$.
\end{ex*}
\vfill 

\begin{ex*}
  Find $\dfrac{d^2}{d\theta^2}\sbrkt{\sin^2(3\theta)}$.
\end{ex*}
\vfill 
\pagebreak 

  %TODO Comment out relsize!
\end{document}
