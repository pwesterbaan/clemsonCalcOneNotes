\documentclass[mathNotesPreamble]{subfiles}
\begin{document}
    \section{JIT 3.2: Equations of Degree 2 (Quadratic equations)}
      \begin{defn*}
        The expression $ax^2+bx+c$ with $a\neq0$ is a polynomial of degree 2, and the equation $ax^2+bx+c=0$ is called an \textbf{equation of degree 2} or a \textbf{quadratic equation}. The roots of a \textbf{quadratic equation} can be found using the \textbf{quadratic formula}:
          $$x=\dfrac{-b\pm\sqrt{b^2-4ac}}{2a}$$
      \end{defn*}
      %\vspace*{25pt}
      \begin{ex*}
        Solve for $s:\hspace*{50pt} s^2+4s+4=0$.
        \vspace*{\stretch{1}}
      \end{ex*}
      \begin{defn*}
        In the quadratic formula, if $b-4ac<0$ (called the \textbf{discriminant}), then the equation contains no Real roots. If we define $i=\sqrt{-1}$, which is an \textbf{imaginary number}, then we have a root that's a \textbf{complex number}, $a+bi$.
      \end{defn*}
      \begin{ex*}
        Solve for $y: \hspace*{50pt}y^2+2y+2=0$.
        \vspace*{\stretch{1}}
      \end{ex*}
      \pagebreak
\end{document}