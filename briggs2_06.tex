\documentclass[answers]{exam}
\usepackage{texPreamble}
\usepackage{relsize}
\usepackage{tabularx}
\extraheadheight{0.25in}
\extrafootheight{1.0in}
\extrawidth{1in}
% ----------------------------------------------------------------
\makeatletter
\title{Fall 2018 Class notes}
\author{\thefname\ \thelname}

\pagestyle{headandfoot}

\firstpageheader{\@title\\\@date}{}{Math 1040}
\firstpageheadrule

\newcommand{\currentname}{\@currentlabelname}

\runningfootrule
\runningfooter{\currentname}{\thepage}{\@title}
\makeatother
\begin{document}
%\relscale{1.4}

\section{2.6: Continuity}
  \begin{defn*}[Continuity at a point]
    A function $f$ is \textbf{continuous} at $a$ if $\ds\lim_{x \to a}f(x)=f(a)$.
  \end{defn*}
  
  \begin{center}
    \fbox{\parbox{0.9\linewidth}{\textbf{Continuity Checklist:}
    
    In order for $f$ to be continuous at $a$, the following three conditions must hold:
    \begin{enumerate}
      \item $f(a)$ is defined ($a$ is in the domain of $f$),
      \item $\ds\lim_{x \to a} f(x)$ exists,
      \item $\ds\lim_{x \to a} f(x)=f(a)$ (the value of $f$ equals the limit of $f$ at $a$).
    \end{enumerate}
    }}
  \end{center}
  \vspace*{\stretch{1}}
  
  \textbf{Graphically:}
  \uplevel{\centering
  \begin{tikzpicture}[scale=0.65]
    \begin{groupplot}[
      group style={group size=4 by 1, horizontal sep=0.65cm},
      axis lines=center,
      axis line style={->},
      xmin=-4, xmax=4,
      ymin=-4, ymax=4,
      xmajorticks=false,
      ymajorticks=false,
      xlabel=$x$, xlabel style={at={(ticklabel* cs:1)},anchor=north west},
      ylabel=$y$, ylabel style={at={(ticklabel* cs:1)},anchor=south west},
      every axis plot/.append style={line width=0.95pt}
      ]
    \nextgroupplot
      \addplot[<->] expression[domain=-4:4,blue] {3/2*x};
    \nextgroupplot[
        ymin=-1.5, ymax=1.5,
        ]
        \addplot[<->] expression[domain=-4:4,blue, samples=100] {-sin(\x r)};
    \nextgroupplot[
      ymin=-6, ymax=6,
      ]
      \addplot[<-] expression[domain=-3.5:-2,blue] {8-x^2};
      \addplot[-]  expression[domain=-2:2,blue] {x^2};
      \addplot[->] expression[domain=2:3.5,blue] {8-x^2};
    \nextgroupplot[
      ymin=-4, ymax=4,
      ]
      \addplot[<-] expression[domain=-4:1.5,blue] {1.5};
      \addplot[->] expression[domain=1.5:4,blue] {x};
    \end{groupplot}
  \end{tikzpicture}}
  \vspace*{\stretch{1}}
  \textbf{Types of discontinuity:}
  
  \uplevel{\centering
  \begin{tikzpicture}[scale=0.65]
    \begin{groupplot}[
      group style={group size=4 by 1, horizontal sep=0.65cm},
      axis lines=center,
      axis line style={->},
      xmin=-0.5, xmax=5.5,
      ymin=-0.5, ymax=2.5,
      xmajorticks=false,
      ymajorticks=false,
      xlabel=$x$, xlabel style={at={(ticklabel* cs:1)},anchor=north west},
      ylabel=$y$, ylabel style={at={(ticklabel* cs:1)},anchor=south west},
      every axis plot/.append style={line width=0.95pt}
      ]
    \nextgroupplot
      \addplot[<->] expression[domain=-0.45:4.85,blue, samples=101] {-(x-0.5)^2/8+2};
      \addplot[holdot] coordinates{(2.5,1.5)} node[above right, black, yshift=-5pt] {$\parens{\frac{5}{2},\frac{3}{2}}$};
      \node[align=left, anchor=south west] at (axis cs: 0.25,0.25) [draw, rectangle, rounded corners] {Removable\\ discontinuity};
    \nextgroupplot
      \addplot[<->] expression[domain=-0.45:4.85,blue, samples=101] {-(x-0.5)^2/8+2};
      \addplot[holdot] coordinates{(2.5,1.5)} node[above right, black, yshift=-5pt] {$\parens{\frac{5}{2},\frac{3}{2}}$};
      \addplot[soldot] coordinates{(2.5,2)} node[above right, black] {$\parens{\frac{5}{2},2}$};
      \node[align=left, anchor=south west] at (axis cs: 0.25,0.25) [draw, rectangle, rounded corners] {Removable\\ discontinuity};
    \nextgroupplot
      \addplot[<-] expression[domain=-0.45:2.5,blue, samples=101] {-(x-0.5)^2/8+2};
      \addplot[->] expression[domain=2.5:4.85,blue, samples=101] {-(x-0.5)^2/8+2.5};
      \addplot[holdot] coordinates{(2.5,1.5)} node[below left, black, yshift=5pt] {$\parens{\frac{5}{2},\frac{3}{2}}$};
      \addplot[soldot] coordinates{(2.5,2)} node[above right, black] {$\parens{\frac{5}{2},2}$};
      \node[align=left, anchor=south west] at (axis cs: 0.25,0.25) [draw, rectangle, rounded corners] {Jump\\ discontinuity};
    \nextgroupplot[
      xmin=-0.5, xmax=3.5,
      ymin=-7, ymax=35,
      ]
      \addplot[<->] expression[unbounded coords=jump, domain=-0.5:3.5,blue, samples=99] {(x-2.5)^-2};
      \node[align=left, anchor=south west] at (axis cs: 3.5/22,3.5) [draw, rectangle, rounded corners] {Infinite\\ discontinuity};
    \end{groupplot}
  \end{tikzpicture}}
  %
  
  \pagebreak
  \begin{defn*}\ 
    \begin{enumerate}[label=,itemsep=\stretch{1}]
      \item A \textbf{removable discontinuity} at $x=a$ is one that disappears when the function becomes continuous after defining $f(a)=\ds\lim_{x \to a} f(x)$.
      \item A \textbf{jump discontinuity} is one that occurs whenever $\ds\lim_{x \to a^-} f(x)$ and $\ds\lim_{x \to a^+} f(x)$ both exist, but $\ds\lim_{x \to a^-} f(x)\neq\lim_{x \to a^+} f(x)$.
      \item A \textbf{vertical discontinuity} occurs whenever $f(x)$ has a vertical asymptote.
    \end{enumerate}
  \end{defn*}
  \vspace*{\stretch{1}}
  \uplevel{\centering
    \begin{tikzpicture}[scale=0.65]
      \begin{groupplot}[
        group style={group size=3 by 2, horizontal sep=2cm, vertical sep=2cm},
        axis lines=center,
        axis line style={->},
        xmin=-4, xmax=4,
        ymin=-4, ymax=4,
        xmajorticks=false,
        ymajorticks=false,
        xlabel=$x$, xlabel style={at={(ticklabel* cs:1)},anchor=north west},
        ylabel=$y$, ylabel style={at={(ticklabel* cs:1)},anchor=south west},
        every axis plot/.append style={line width=0.95pt}
        ]
      \nextgroupplot
        \addplot[<->] expression[domain=-4:2, blue] {x+2};
        \addplot[holdot] coordinates{(0,2)};
      \nextgroupplot[
        xmin=-4, xmax=4,
        ymin=-4, ymax=4,
        ]
        \addplot[<->] expression[domain=-4:2, blue] {x+2};
        \addplot[holdot] coordinates{(0,2)};
        \addplot[soldot] coordinates{(0,2.5)};
      \nextgroupplot[
        xmin=-1.5, xmax=2.5,
        ymin=-4, ymax=4,
        ]
        \addplot[-] expression[domain=-1:0, blue] {-3*x-3};
        \addplot[-] expression[domain=0:2, blue] {-2*x+3};
        \addplot[holdot] coordinates{(-1,0)(0,3)};
        \addplot[soldot] coordinates{(0,-3)(2,-1)};
      \nextgroupplot[
        xmin=-4, xmax=4,
        ymin=-1.95, ymax=1.95,
        ]
        \addplot[<-] expression[domain=-4:0, blue] {-1};
        \addplot[->] expression[domain=0:4, blue] {1};
        \node at(axis cs: 2.5,-0.90) {$\dfrac{\abs{x}}{x}$};
        \addplot[holdot] coordinates{(0,-1)(0,1)};
      \nextgroupplot[
        xmin=-4, xmax=4,
        ymin=-4, ymax=4,
        ]
        \foreach \n in {-3, ..., 3}{
          \addplot[-] expression[domain=\n:\n+1, blue]{\n};
          \addplot[holdot] coordinates{(\n,\n)};
          \addplot[soldot] coordinates{(\n+1,\n)};
        }
        \node at(axis cs: 2.5,-1.75) {$\floor{x}$};
      \nextgroupplot[
        xmin=-2, xmax=6,
        ymin=-0.5, ymax=4,
        ]
        \addplot[<->] expression[domain=-2:6, blue, samples=101] {(x-2)^-2};
        \node at(axis cs: 5,2) {$\dfrac{1}{(x-2)^2}$};
      \end{groupplot}
    \end{tikzpicture}}
  \pagebreak
  \vspace*{-1cm}
  \uplevel{\centering
    \begin{tikzpicture}[scale=0.65]
      \begin{groupplot}[
        group style={group size=3 by 2, horizontal sep=2cm, vertical sep=0.5cm},
        axis lines=center,
        axis line style={->},
        xmin=-4, xmax=4,
        ymin=-4, ymax=4,
        xmajorticks=false,
        ymajorticks=false,
        xlabel=$x$, xlabel style={at={(ticklabel* cs:1)},anchor=north west},
        ylabel=$y$, ylabel style={at={(ticklabel* cs:1)},anchor=south west},
        every axis plot/.append style={line width=0.95pt}
        ]
      \nextgroupplot
        \addplot[-] expression[domain=-4:-1.05, blue, samples=51, unbounded coords=jump] {(x+1)^-1};
        \addplot[-] expression[domain=-0.95:4, blue, samples=51, unbounded coords=jump] {(x+1)^-1};
        \node at(axis cs: 2,-2) {$\dfrac{1}{x+1}$};
      \nextgroupplot[
        xmin=-0.75, xmax=0.75,
        ymin=-2, ymax=2,
        ]
        \def\n{3}
        \addplot[-] expression[domain=-4:-2/pi, blue, samples=50] {cos(deg 1/x)};
        \addplot[-] expression[domain=2/pi:4, blue, samples=50] {cos(deg 1/x)};
        %% Draw 1000 points on finer intervals
        \foreach \k in {1, ...,\n}{
          \addplot[-] expression[domain=-2/(\k*pi):-1/(\k*pi), blue, samples=1000] {cos(deg 1/x)};
          \addplot[-] expression[domain=1/(\k*pi):2/(\k*pi), blue, samples=1000] {cos(deg 1/x)};
        }
        %% Fudge the middle because resolution
        \addplot[-] expression[domain=-1/(\n*pi):1/(\n*pi), blue, samples=1000] {cos(deg 1/x)};
        \node at(axis cs: 0.5,-1.25) {$\cos\parens{\dfrac{1}{x}}$};
      \nextgroupplot[
        xmin=-4, xmax=4,
        ymin=-4, ymax=4,
        ]
        \addplot[-] expression[unbounded coords=jump, domain=-4:-0.01, blue, samples=99] {(x+2)/x};
        \addplot[-] expression[unbounded coords=jump, domain=0.01:4, blue, samples=99] {(x+2)/x};
        \addplot[holdot] coordinates{(1,3)};
        \node at(axis cs: 2.5,-1.75) {$\dfrac{x^2+x-2}{x^2-x}$};
      \nextgroupplot[
        xmin=-4, xmax=4,
        ymin=-2/3, ymax=4,
        ]
        \addplot[-] expression[domain=-2:-1, blue] {-3*x-3};
        \addplot[-] expression[domain=-1:0, blue] {3*x+3};
        \addplot[-] expression[domain=0:3, blue] {-4/3*(x^2-3*x)};
        \addplot[holdot] coordinates{(0,3)(2,8/3)};
        \addplot[soldot] coordinates{(-2,3)(0,0)(2,1)(3,0)};
      \nextgroupplot[
        xmin=-1, xmax=6,
        ymin=-1, ymax=6,
        ]
        \addplot[-] expression[domain=0:1, blue] {2*x+1};
        \addplot[-] expression[domain=1:2, blue] {4-x};
        \addplot[-] expression[domain=2:2.9, blue] {-(x-3)^-2+5};
        \addplot[-] expression[domain=3.1:4, blue] {-(x-3)^-2+5};
        \addplot[-] expression[domain=4:5, blue] {-4*x+20};
        \addplot[holdot] coordinates{(2,4)(4,4)};
        \addplot[soldot] coordinates{(0,1)(2,2)(4,2)(5,0)};
      \nextgroupplot[
        xmin=-1, xmax=6,
        ymin=-5/6, ymax=5,
        ]
        \addplot[-] expression[domain=0:1, blue] {x+2};
        \addplot[-] expression[domain=1:2, blue] {4-x};
        \addplot[-] expression[domain=2:3, blue] {(x-2)^2+3};
        \addplot[-] expression[domain=3:5, blue] {-3/4*(x-3)^2+4};
        \addplot[holdot] coordinates{(0,2)(1,3)(2,3)(3,4)(5,1)};
        \addplot[soldot] coordinates{(1,4)(2,2)};
      \end{groupplot}
    \end{tikzpicture}}
  \vspace*{\stretch{1}}
  \begin{defn*}[Continuity at Endpoints]
    A function $f$ is
      \begin{itemize}
        \item 
          \textbf{continuous from the right} (or \textbf{right-continuous}) at $a$ if $\ds\lim_{x \to a^+} f(x)=f(a)$.
        \item 
          \textbf{continuous from the left} (or \textbf{left-continuous}) at $b$ if $\ds\lim_{x \to b^-} f(x)=f(b)$.
      \end{itemize}
  \end{defn*}
  \vspace*{\stretch{1}}
  \begin{defn*}[Continuity on an Interval]
    A function $f$ is \textbf{continuous on an open interval $I$} if it is continuous at all points in $I$.
    \begin{itemize}
        \item If $f$ is also left-continuous at $b$, then we say $f$ is \textbf{continuous on $(a,b]$}.
        \item If $f$ is also right-continuous at $a$, then we say $f$ is \textbf{continuous on $[a,b)$}.
        \item If $f$ is also left- and right-continuous at $b$ and $a$, respectively, then we say $f$ is \textbf{continuous on $[a,b]$}.
    \end{itemize}
  \end{defn*}
  
  \pagebreak
  
  \uplevel{
  \centering
  \begin{tabularx}{\linewidth}{YYY}
    Continuous on $[a,b)$&
    Continuous on $(a,b]$&
    Continuous on $(a,b)$
  \end{tabularx}
  \begin{tikzpicture}[scale=0.8]
    \begin{groupplot}[
      group style={group size=3 by 1, horizontal sep=1.25cm},
      axis lines=center,
      axis line style={->},
      xmin=-0.5, xmax=3,
      ymin=-0.5, ymax=2.5,
      xmajorticks=false,
      ymajorticks=false,
      xlabel=$x$, xlabel style={at={(ticklabel* cs:1)},anchor=north west},
      ylabel=$y$, ylabel style={at={(ticklabel* cs:1)},anchor=south west},
      every axis plot/.append style={line width=0.95pt}
      ]
    \nextgroupplot
      \addplot[-] expression[domain=0.5:2.5, blue] {-((x-0.5)/2)^2+2};
      \addplot[soldot] coordinates{(0.5,2)};
      \addplot[holdot] coordinates{(2.5,1)};
    \nextgroupplot
      \addplot[-] expression[domain=0.5:2.5, blue] {-((x-0.5)/2)^2+2};
      \addplot[holdot] coordinates{(0.5,2)};
      \addplot[soldot] coordinates{(2.5,1)};
    \nextgroupplot
      \addplot[-] expression[domain=0.5:2.5, blue] {-((x-0.5)/2)^2+2};
      \addplot[holdot] coordinates{(0.5,2)};
      \addplot[holdot] coordinates{(2.5,1)};
    \end{groupplot}
  \end{tikzpicture}}
  \vspace*{\stretch{1}}
  \begin{ex*}
    Determine the interval of continuity for the following:
    
    \noindent
    \begin{minipage}[t]{0.5\linewidth}\ 
    
      $$f(x)=\begin{cases}
        x^2+1,& x\leq 0\\
        3x+5,& x>0
      \end{cases}$$
    \end{minipage}%
    \begin{minipage}[t]{0.5\linewidth}\ 
    
      \begin{flushright}
        \begin{tikzpicture}
          \begin{axis}[
            axis lines=center,
            axis line style={->},
            xmin=-4.25, xmax=4.25,
            ymin=-2.25, ymax=10.5,
            xlabel=$x$, xlabel style={at={(ticklabel* cs:1)},anchor=north west},
            ylabel=$y$, ylabel style={at={(ticklabel* cs:1)},anchor=south west},
            every axis plot/.append style={line width=0.95pt}
            ]
            \addplot[-] expression[domain=-4:0] {x^2+1};
            \addplot[-] expression[domain=0:4] {3*x+5};
            \addplot[holdot] coordinates{(0,5)};
            \addplot[soldot] coordinates{(0,1)};
          \end{axis}
        \end{tikzpicture}
      \end{flushright}
    \end{minipage}%
  \end{ex*}
  
  
  
  \uplevel{\centering
    \begin{tikzpicture}[scale=0.8]
      \begin{groupplot}[
        group style={group size=3 by 1, horizontal sep=1.25cm},
        axis lines=center,
        axis line style={->},
        xmin=-4, xmax=4,
        ymin=-4/5, ymax=4,
        xtick={-4,-3,...,4},
        ytick={0,1,...,6},
        enlargelimits={abs=0.75},
        xlabel=$x$, xlabel style={at={(ticklabel* cs:1)},anchor=north west},
        ylabel=$y$, ylabel style={at={(ticklabel* cs:1)},anchor=south west},
        every axis plot/.append style={line width=0.95pt}
        ]
      \nextgroupplot
        \addplot[-] expression[domain=-2:-1, blue] {-3*x-3};
        \addplot[-] expression[domain=-1:0, blue] {3*x+3};
        \addplot[-] expression[domain=0:3, blue] {-4/3*(x^2-3*x)};
        \addplot[holdot] coordinates{(0,3)(2,8/3)};
        \addplot[soldot] coordinates{(-2,3)(0,0)(2,1)(3,0)};
      \nextgroupplot[xmin=-0.5, xmax=4]
        \addplot[-] expression[domain=0:1, blue] {2*x+1};
        \addplot[-] expression[domain=1:2, blue] {4-x};
        \addplot[-] expression[domain=2:2.9, blue] {-(x-3)^-2+5};
        \addplot[-] expression[domain=3.1:4, blue] {-(x-3)^-2+5};
        \addplot[-] expression[domain=4:5, blue] {-4*x+20};
        \addplot[holdot] coordinates{(2,4)(4,4)};
        \addplot[soldot] coordinates{(0,1)(2,2)(4,2)(5,0)};
      \nextgroupplot[xmin=-0.5, xmax=4]
        \addplot[-] expression[domain=0:1, blue] {x+2};
        \addplot[-] expression[domain=1:2, blue] {4-x};
        \addplot[-] expression[domain=2:3, blue] {(x-2)^2+3};
        \addplot[-] expression[domain=3:5, blue] {-3/4*(x-3)^2+4};
        \addplot[holdot] coordinates{(0,2)(1,3)(2,3)(3,4)(5,1)};
        \addplot[soldot] coordinates{(1,4)(2,2)};
      \end{groupplot}
    \end{tikzpicture}}
  \pagebreak
  \begin{ex*}
    Determine whether the following are continuous at $a$:
    \begin{enumerate}[label=,itemsep=\stretch{1}]
      \item $f(x)=x^2+\sqrt{7-x},\ a=4$
      \item $g(x)=\dfrac{1}{x-3},\ a=3$
      \item 
        $h(x)=\begin{cases}
          \dfrac{x^2+x}{x+1}, &x\neq -1\\
          0,& x=-1
        \end{cases},\ a=-1$
      \item 
        $j(x)=\abs{x}=\begin{cases}
          x,& x\geq 0\\
          -x& x<0
        \end{cases},\ a=0$
      \item 
        $k(x)=\begin{cases}
          \dfrac{x^2+x-6}{x^2-x},& x\neq 2\\
          -1,& x=2
        \end{cases},\ a=2$
    \end{enumerate}
  \end{ex*}
  \vspace*{\stretch{1}}
  \pagebreak

  \begin{center}
    \fbox{\parbox{0.9\linewidth}{
    \textbf{Theorem 2.9: Continuity Rules}
    
    If $f$ and $g$ are continuous at $a$, then the following functions are also continuous at $a$. Assume $c$ is a constant and $n>0$ is an integer.
    \begin{tasks}(2)
      \task $f+g$
      \task $f-g$
      \task $cf$
      \task $fg$
      \task $f/g$, provided that $g(a)\neq 0$.
      \task $\parens{f(x)}^n$
    \end{tasks}
    }}
    \vfill
    \fbox{\parbox{0.9\linewidth}{
    \textbf{Theorem 2.10: Polynomial and Rational Functions}
    
    \begin{enumerate}[label=\alph*)]
      \item A polynomial function is continuous for all $x$.
      \item A rational function (a function of the form $\frac{p}{q}$, where $p$ and $q$ are polynomials) is continuous for all $x$ for which $q(x)\neq 0$.
    \end{enumerate}
    }}
    \vfill
    \fbox{\parbox{0.9\linewidth}{
    \textbf{Theorem 2.11: Continuity of Composite Functions at a Point}
    
    If $g$ is continuous at $a$ and $f$ is continuous at $g(a)$, then the composite function $f\circ g$ is continuous at $a$.
    }}  
    \vfill
    \fbox{\parbox{0.9\linewidth}{
    \textbf{Theorem 2.12: Limits of Composite Functions}
    \begin{enumerate}
      \item If $g$ is continuous at $a$ and $f$ is continuous at $g(a)$, then
        $$\lim_{x \to a} f\parens{g(x)}=f\parens{\lim_{x \to a} g(x)}=f\parens{g(a)}.$$
      \item If $\ds\lim_{x \to a} g(x)=L$ and $f$ is continuous at $L$, then
        $$\lim_{x \to a} f\parens{g(x)}=f\parens{\lim_{x \to a}g(x)}=f\parens{L}.$$
    \end{enumerate}
    }}
    \pagebreak

    \fbox{\parbox{0.9\linewidth}{
    \textbf{Theorem 2.13: Continuity of Functions with Roots}
    
      Assume $n$ is a positive integer. If $n$ is an odd integer, then $\parens{f(x)}^{\sfrac{1}{n}}$ is continuous at all points at which $f$ is continuous.
      
      If $n$ is even, then $\parens{f(x)}^{\sfrac{1}{n}}$ is continuous at all points $a$ at which $f$ is continuous at $f(a)>0$.
    }}
    
    \vspace*{15pt}
    
    \fbox{\parbox{0.9\linewidth}{
    \textbf{Theorem 2.14: Continuity of Inverse Functions}
      
      If a function $f$ is continuous on an interval $I$ and has an inverse on $I$, then its inverse $f\inv$ is also continuous (on the interval consisting of the points $f(x)$, where $x$ is in $I$).
    }}
    
    \vspace*{15pt}
    \fbox{\parbox{0.9\linewidth}{
    \textbf{Theorem 2.15: Continuity of Transcendental Functions}
    
    The following functions are continuous at all points of their domains.
    
    {\begin{tabularx}{\linewidth}{*{6}{X}}
      \multicolumn{2}{L}{\textbf{Trigonometric}}& 
      \multicolumn{2}{L}{\textbf{Inverse Trigonometric}}& 
      \multicolumn{2}{L}{\textbf{Exponential}}\\
      $\sin x$& $\cos x$& $\sin\inv x$& $\cos\inv x$& $b^x$& $e^x$\\
      $\tan x$& $\cot x$& $\tan\inv x$& $\cot\inv x$& 
      \multicolumn{2}{L}{\textbf{Logarithmic}}\\
      $\sec x$& $\csc x$& $\sec\inv x$& $\csc\inv x$& $\log_b x$& $\ln x$
    \end{tabularx}
    }}}
  \end{center}
  
  \begin{ex*}
    Determine the intervals of continuity for the following functions:
  \end{ex*}
  \begin{tasks}[after-item-skip=\stretch{1}](2)
    \task $g(x)=\dfrac{3x^2-6x+7}{x^2+x+1}$
    \task $h(x)=\dfrac{3x^2-6x+7}{x^2-x-1}$
    \task $s(x)=\dfrac{x^2-4x+3}{x^2-1}$
    \task $t(x)=\dfrac{x^2-4x+3}{x^2+1}$
  \end{tasks}
  \vspace*{\stretch{1}}
  \pagebreak
  
  \begin{tasks}[resume,after-item-skip=\stretch{1}](2)
    \task $q(x)=\sqrt[3]{x^2-2x-3}$
    \task $r(x)=\sqrt{x^2-2x-3}$
    \task $a(x)=\sec x$
    \task $b(x)=\sqrt{\sin x}$
    \task 
      $\ell(x)=\begin{cases}
        x^3+4x+1,& x\leq 0\\
        2x^3,& x>0
      \end{cases}$
    \task 
      $m(x)=\begin{cases}
        \sin x,& x<\frac{\pi}{4}\\
        \cos x,& x\geq \frac{\pi}{4}
      \end{cases}$
  \end{tasks}
  \vspace*{\stretch{1}}
  \pagebreak
  \begin{ex*}
    Sketch a function that:
    
    \uplevel{\centering
      \begin{tabularx}{\linewidth}{YY}
        Is defined, but not continuous at $x=1$,&
        Has a limit, but not continuous at $x=1$.\\
      \end{tabularx}
      \begin{tikzpicture}
        \begin{groupplot}[
          group style={group size=2 by 1, horizontal sep=3cm},
          axis lines=center,
          axis line style={->},
          xmin=-0.5, xmax=4,
          ymin=-0.5, ymax=4,
          xtick={-4,-3,...,4},
          ytick={-2,-1,...,6},
          xmajorticks=false,
          ymajorticks=false,
          enlargelimits={abs=0.75},
          ticklabel style={font=\tiny, inner sep=0.75pt,fill=white},
          xlabel=$x$, xlabel style={at={(ticklabel* cs:1)},anchor=north west},
          ylabel=$y$, ylabel style={at={(ticklabel* cs:1)},anchor=south west},
          every axis plot/.append style={line width=0.95pt}
          ]
        \nextgroupplot
        \nextgroupplot
        \end{groupplot}
      \end{tikzpicture}}
  \end{ex*}
  \begin{ex*}
    Determine the value of the unknown parameters so that $f(x)$ is continuous:
  \end{ex*}
    \begin{enumerate}[itemsep=\stretch{1}]
      \item 
        $f(x)=\begin{cases}
          \dfrac{x^3-1}{x-1},& x\neq 1\\
          a,& x=1
        \end{cases}$
      \item 
        $f(t)=\begin{cases}
          \dfrac{t^2+3t-10}{t-2},& t\neq2\\
          a,& t=2
        \end{cases}$
      \vspace*{\stretch{1}}
    \pagebreak

      \item 
        $f(x)=\begin{cases}
          \dfrac{x^2-4}{x-2},& x<2\\
          ax^2-bx+3,& 2\leq x<3\\
          2x-a+b,& x\geq 3
        \end{cases}$
    \end{enumerate}
  \vspace*{\stretch{1}}
  \begin{ex*}
    Redefine the following functions so that they are continuous everywhere:
    \begin{enumerate}[itemsep=\stretch{1}]
      \item $g(x)=\dfrac{x^3-x^2-2x}{x-2}$
      \item $g(x)=\dfrac{x^2+x-6}{x-2}$
    \end{enumerate}
  \end{ex*}
  \vspace*{\stretch{1}}
  \pagebreak
  
  \fbox{\parbox{0.9\linewidth}{
  \textbf{Theorem 2.16: Intermediate Value Theorem}
  
  Suppose $f$ is continuous on the interval $\sbrkt{a,b}$ and $L$ is a number strictly between $f(a)$ and $f(b)$. Then there exists at least one number $c$ in $(a,b)$ satisfying $f(c)=L$.
  }}
  
  \begin{center}
    \begin{tikzpicture}
      \begin{groupplot}[
        group style={group size=2 by 1, horizontal sep=2.75cm},
        axis lines=center,
        axis line style={->},
        xmin=-0.0625, xmax=3,
        ymin=-0.0625, ymax=2, 
        enlargelimits={abs=0.75},
        xlabel=$x$, xlabel style={at={(ticklabel* cs:1)},anchor=north west},
        ylabel=$y$, ylabel style={at={(ticklabel* cs:1)},anchor=south west},
        every axis plot/.append style={line width=0.95pt}
        ]
      \nextgroupplot[
        xtick={1,2.414,3},
        xticklabels={$a$,$c$,$b$},
        ytick={0.5,1.5,2.5},
        yticklabels={$f(a)$,$L$,$f(b)$},
        ]
          \addplot[-] expression[domain=1:3] {(x-1)^2/2+0.5};
          \addplot[soldot] coordinates{(1,0.5)};
          \addplot[soldot] coordinates{(3,2.5)};
          \draw[dashed] (axis cs:0,1.5) to (axis cs:4,1.5);
          \draw[dashed] (axis cs:2.414,0) to (axis cs:2.414, 1.5);
      \nextgroupplot[
        xtick={0.5,1,2,3,3.5},
        xticklabels={$a$,$c_1$,$c_2$,$c_3$,$b$},
        ytick={0.5625,1.5,2.4375},
        yticklabels={$f(b)$,$L$,$f(a)$},
        ]
          \addplot[-] expression[domain=0.5:3.5, samples=50] {-0.5*(x-1)*(x-2)*(x-3)+1.5};
          \addplot[soldot] coordinates{(0.5,2.4375)};
          \addplot[soldot] coordinates{(3.5,0.5625)};
          \draw[dashed] (axis cs:0,1.5) to (axis cs:4,1.5);
          \draw[dashed] (axis cs:1,0) to (axis cs:1, 1.5);
          \draw[dashed] (axis cs:2,0) to (axis cs:2, 1.5);
          \draw[dashed] (axis cs:3,0) to (axis cs:3, 1.5);
      \end{groupplot}
    \end{tikzpicture}
  \end{center}
  
  \vspace*{\stretch{1}}
  \textit{Note:} It is important that the function be continuous on the interval $[a,b]$:
  \begin{center}
    \begin{tikzpicture}
      \begin{axis}[
        axis lines=center,
        axis line style={->},
        xmin=-0.5, xmax=3.5,
        ymin=-0.5, ymax=3.5,
        xtick={0.5,4},
        xticklabels={$a$,$b$},
        ytick={0.53125,1.5,3.5},
        yticklabels={$f(a)$,$L$,$f(b)$},
        enlargelimits={abs=0.75},
        xlabel=$x$, xlabel style={at={(ticklabel* cs:1)},anchor=north west},
        ylabel=$y$, ylabel style={at={(ticklabel* cs:1)},anchor=south west},
        every axis plot/.append style={line width=0.95pt}
        ]
        \addplot[-] expression[domain=0.5:2] {x^2/8+0.5};
        \addplot[-] expression[domain=2:4] {x^2/8+1.5};
        \addplot[soldot] coordinates{(0.5,0.53125)(2,2)(4,3.5)};
        \addplot[holdot] coordinates{(2,1)};
        \draw[dashed] (axis cs:0,1.5) to (axis cs: 5,1.5);
      \end{axis}
    \end{tikzpicture}
  \end{center}
  \pagebreak
  
  \begin{ex*}
    Show that $f(x)$ has a root using the IVT:\ $f(x)=x^3+4x+4$
      \vspace*{\stretch{1}}
  \end{ex*}
  \begin{ex*}
    Show that $\sqrt{x^4+25x^3+10}=5$ on the interval $(0,1)$.
    \vspace*{\stretch{1}}
  \end{ex*}
  \begin{ex*}
    Show that $-x^5-4x^2+2\sqrt x+5=0$ on $(0,3)$.
    \vspace*{\stretch{1}}
  \end{ex*}
  \pagebreak
\end{document}