\documentclass[answers]{exam}
\usepackage{texPreamble}
\usepackage{relsize}
\usepackage{calc}
\usepackage{tabularx}
\extraheadheight{0.25in}
\extrafootheight{1.0in}
\extrawidth{1in}
% ----------------------------------------------------------------
\firstpagefootrule
\runningfootrule
\begin{document}
%\relscale{1.4}
\section{4.9: Antiderivatives}
  \begin{defn*}[Antiderivative]
    A function $F$ is an \textbf{antiderivative} of $f$ on an interval $I$ provided $F'(x)=f(x)$, for $x$ in $I$.
  \end{defn*}
  \begin{center}
    \textit{Note:} we will denote this relationship in the following way:
    
    \begin{tabular}{@{}cc@{}}\toprule
      Function& Anti-derivative\\\midrule
      $f'(x)$& $f(x)$\\
      $f(x)$& $F(x)$\\\bottomrule
    \end{tabular}
  \end{center}
  
  \vspace*{\stretch{0.15}}
  \begin{ex*}
    If $f(x)=\tan(x)$, then $f'(x)=\sec^2(x)$. In this case, $\tan(x)$ is the \textit{antiderivative} of $\sec^2(x)$.
  \end{ex*}
  \vspace*{\stretch{1}}
  
  \noindent
  \fbox{\parbox{0.9875\linewidth}{
    \textbf{Theorem 4.15: The Family of Antiderivatives}
    
    Let $F$ be any antiderivative of $f$ on an interval $I$. Then \textit{all} antiderivatives of $f$ on $I$ have the form $F+C$, where $C$ is an arbitrary constant.
  }}
  \vspace*{\stretch{0.15}}
  
  \noindent
  \begin{minipage}[t]{0.5\linewidth}
    \mbox{}\\[-\baselineskip]
    \begin{ex*}
    If $f'(x)=x^2$, then $f(x)=\dfrac{x^3}{3}+C$ is the family of antiderivatives of $f'(x)$.
  \end{ex*}
  \end{minipage}%
  \begin{minipage}[t]{0.5\linewidth}
    \mbox{}\\[-\baselineskip]
    \begin{flushright}
      \begin{tikzpicture}[scale=1.05]
        \begin{axis}[
          axis lines=center,
          axis line style={-},
          xmin=-3, xmax=3,
          ymin=-5, ymax=5,
          xmajorticks=false,
          ymajorticks=false,
          ticklabel style={font=\footnotesize,inner sep=0.5pt,fill=white,opacity=1.0, text opacity=1},
          every axis plot/.append style={line width=0.95pt, color=blue, samples=100}
          ]
          \addplot[black,-] expression[domain=-3:3]{x^3/3+3};
          \addplot[ClemsonOrange, -] expression[domain=-3:3]{x^3/3+2};
          \addplot[-] expression[domain=-3:3]{x^3/3+1};
          \addplot[red, -] expression[domain=-3:3]{x^3/3};
          \addplot[green!75!blue!85, -] expression[domain=-3:3]{x^3/3-1};
          \addplot[ClemsonPurple, -] expression[domain=-3:3]{x^3/3-2};
        \end{axis}
      \end{tikzpicture}
    \end{flushright}
  \end{minipage}
  \pagebreak
  
  \begin{ex*}
    Find the most general antiderivative of the following functions
  \end{ex*}
  \begin{tasks}[after-item-skip=\stretch{1}, label=\mbox{}](2)
    \task $f(x)=\sin(x)$
    \task $k(x)=\dfrac{1}{1+x^2}$
    \task $g(x)=x^n,\ n\neq -1$
    \task $h(x)=\sfrac{1}{x}$
    \task $j(x)=3x^2$
    \task $n(x)=6x^5$
    \task $\ell(x)=\dfrac{41}{x}+4e^x$
    \task $m(x)=\dfrac{1}{2x^3}$
  \end{tasks}
  \vspace*{\stretch{1}}
  
  \pagebreak
  \newlength{\paWidth}
  \settowidth{\paWidth}{\textnormal{antiderivative}}
  \newlength{\fWidth}
  \settowidth{\fWidth}{\textnormal{Function}\hspace*{40pt}}
  \begin{center}
    \renewcommand{\arraystretch}{2}
    \begin{tabular}{@{}>{$}m{\fWidth}<{$}>{$}m{0.3\linewidth}<{$}>{$}m{\fWidth}<{$}>{$}m{\paWidth}<{$}@{}}\toprule
      \textnormal{Function}& \textnormal{Particular}\break
        \textnormal{antiderivative}&
      \textnormal{Function}& \textnormal{Particular}\break
        \textnormal{antiderivative}\\\midrule
      cf(x)& cF(x)& f(x)+g(x)& F(x)+G(x)\\
      x^n\ (n\neq -1)& \dfrac{x^\npo}{\npo}& \dfrac{1}{x}& \ln\abs{x}\\
      \cos(x)& \sin(x)& \sin(x)& -\cos(x)\\
      \sec^2(x)& \tan(x)& \sec(x)\tan(x)& \sec(x)\\
      \dfrac{1}{\sqrt{1-x^2}}& \sin\inv(x)& \dfrac{1}{1+x^2}& \tan\inv(x)\\
      \dfrac{1}{\abs{x}\sqrt{x^2-1}}& \sec\inv(x)& e^x& e^x\\\bottomrule
    \end{tabular}
    
    \vspace*{\stretch{1}}
    \parbox{0.725\linewidth}{
      \textit{Note:} There are some more `complicated' antiderivatives as well:
        \begin{align*}
          f(x)=e^{g(x)} &\quad\Rightarrow\quad F(x)=\dfrac{e^{g(x)}}{g'(x)}\\
          f(x)=k\sec^2(kx) &\quad\Rightarrow\quad F(x)=\tan(kx)
        \end{align*}
      Focus more on ``What can I take the derivative of to get ...'' rather than memorizing formulas.}
    \vspace*{\stretch{1}}
  \end{center}
  \pagebreak
  
  \begin{defn*}
    Recall that $\dfrac{d}{dx}\sbrkt{f(x)}$ represents taking the derivative of $f(x)$ with respect to $x$.
    \begin{itemize}
      \item 
        Finding the antiderivative of $f$ with respect to $x$ is the \textbf{indefinite integral} $\ds\int f(x)\,dx$.
      \item 
        The \textbf{integrand} is the function $f(x)$ we are integrating.
      \item 
        The \textbf{variable of integration}, $dx$, indicates which variable we are integrating with respect to.
    \end{itemize}
  \end{defn*}
  \vspace*{\stretch{1}}
  
  \noindent
  \fbox{\parbox{0.9875\linewidth}{
    \textbf{Theorem 4.16: Power Rule for Indefinite Integrals}
    
    \[\int x^p\,dx=\dfrac{x^{p+1}}{p+1}+C\]
    where $p\neq -1$ is a real number and $C$ is an arbitrary constant.
  }}
  \vspace*{\stretch{1}}
  
  \noindent
  \fbox{\parbox{0.9875\linewidth}{
  \textbf{Theorem 4.17: Constant Multiple and Sum Rules}
  
  \textbf{Constant Multiple Rule:} $\ds\int cf(x)\,dx=c\int f(x)\,dx$, for real numbers $c$.
  
  \textbf{Sum Rule:} $\ds\int\parens{f(x)+g(x)}\,dx=\int f(x)\,dx+\int g(x)\,dx$
  }}
  \vspace*{\stretch{1}}
  \pagebreak
  
  \noindent
  \textbf{Table 4.9: Indefinite Integrals of Trigonometric Functions}
  \begin{center}
    \renewcommand{\arraystretch}{2.25}
    \begin{tabular}{@{}R@{\,=\,}L@{$\qquad\Longrightarrow\qquad$}R@{\,=\,}R@{}}\toprule
      \ds\ddx\sbrkt{\sin(x)}&\phantom{-}\cos(x)& \int\cos(x)\,dx&\sin(x)+C\\
      \ds\ddx\sbrkt{\cos(x)}&-\sin(x)& \int\sin(x)\,dx&-\cos(x)+C\\
      \ds\ddx\sbrkt{\tan(x)}&\phantom{-}\sec^2(x)& \int\sec^2(x)\,dx&\tan(x)+C\\
      \ds\ddx\sbrkt{\cot(x)}&-\csc^2(x)& \int\csc^2(x)\,dx&-\cot(x)+C\\
      \ds\ddx\sbrkt{\sec(x)}&\phantom{-}\sec(x)\tan(x)& \int\sec(x)\tan(x)\,dx&\sec(x)+C\\
      \ds\ddx\sbrkt{\csc(x)}&-\csc(x)\cot(x)& \int\csc(x)\cot(x)\,dx&-\csc(x)+C\\\bottomrule
    \end{tabular}
  \end{center}
  \vspace*{\stretch{1}}
  
  \noindent
  \textbf{Table 4.10: Other Indefinite Integrals}
  \begin{center}
    \renewcommand{\arraystretch}{2.25}
    \begin{tabular}{@{}L@{\,=\,}L@{$\qquad\Longrightarrow\qquad$}R@{\,=\,}R@{}}\toprule
      \ds\ddx\sbrkt{e^x}&e^x& \ds\int e^x\,dx&e^x+C\\
      \ds\ddx\sbrkt{\ln\abs{x}}&\dfrac{1}{x}& \ds\int \dfrac{1}{x}\,dx&\ln\abs{x}+C\\
      \ds\ddx\sbrkt{\sin\inv(x)}&\dfrac{1}{\sqrt{1-x^2}}& \ds\int \dfrac{dx}{\sqrt{1-x^2}}&\sin\inv(x)+C\\
      \ds\ddx\sbrkt{\tan\inv(x)}&\dfrac{1}{1+x^2}& \ds\int \dfrac{dx}{1+x^2}&\tan\inv(x)+C\\
      \ds\ddx\sbrkt{\sec\inv\abs{x}}&\dfrac{1}{x\sqrt{x^2-1}}& \ds\int \dfrac{dx}{x\sqrt{x^2-1}}&\sec\inv\abs{x}+C\\\bottomrule
    \end{tabular}
  \end{center}
  \pagebreak
  
  \begin{ex*}
    Verify the following integration formulas using differentiation.
  \end{ex*}
  \begin{tasks}[after-item-skip=\stretch{1}, label=\mbox{}](1)
    \task $\ds\int \frac{1}{(x+1)^2}\,dx=\dfrac{x}{x+1}+C$
    \task $\ds\int\sec^2\parens{5x-1}\,dx=\dfrac{1}{5}\tan\parens{5x-1}+C$
    \task $\ds\int\cos^3(x)\,dx=\sin(x)-\dfrac{1}{3}\sin^3(x)+C$
    \task $\ds\int\dfrac{x}{\sqrt{x^2+1}}=\sqrt{x^2+1}+C$
  \end{tasks}
  \vspace*{\stretch{1}}
  \pagebreak
  
  \begin{ex*}
    Find the most general anti-derivative or indefinite integral
  \end{ex*}
  \begin{tasks}[after-item-skip=\stretch{1}, label=\mbox{}](2)
    \task $\ds\int\parens{\frac{t^2}{2}+4t^3}\,dt$
    \task $\ds\int\parens{\dfrac{1}{5}-\dfrac{2}{x^3}+2x}\,dx$
    \task $\ds\int\parens{\dfrac{\sqrt{x}}{2}+\dfrac{2}{\sqrt{x}}}\,dx$
    \task $\ds\int\dfrac{\csc\theta\cot\theta}{2}\,d\theta$
  \end{tasks}
  \vspace*{\stretch{1}}
  \pagebreak
  
  \begin{tasks}[after-item-skip=\stretch{1}, label=\mbox{}](2)
    \task $\ds\int\parens{1+\dfrac{1}{4u^3}-\parens{3u}^2}\,du$
    \task $\ds\int\parens{\dfrac{7}{\sqrt{1-x^2}}-\dfrac{3}{\cos^2(x)}}\,dx$
    \task $\ds\int\parens{\dfrac{1}{4e^x}-\dfrac{4}{x}+4^x}\,dx$
    \task $\ds\int\dfrac{x^2-36}{x-6}\,dx$
  \end{tasks}
  \vspace*{\stretch{1}}
  \pagebreak
  
  \begin{tasks}[after-item-skip=\stretch{1}, label=\mbox{}](2)
    \task $\ds\int\dfrac{2+3\cos(y)}{\sin^2(y)}\,dy$
    \task $\ds\int\parens{u+4}\parens{2u+1}\,du$
    \task $\ds\int e^{x+2}\,dx$
    \task $\ds\int\parens{\sqrt[4]{x^3}+\sqrt{x^5}}\,dx$
  \end{tasks}
  \vspace*{\stretch{1}}
  \pagebreak
  
  \begin{tasks}[after-item-skip=\stretch{1}, label=\mbox{}](2)
    \task $\ds\int\parens{x^2+1+\dfrac{1}{x^2+1}}\,dx$
    \task $\ds\int\parens{\sin(4x)-\dfrac{3}{\sin^2(x)}}\,dx$
    \task $\ds\int \parens{\csc^2(2t)-2e^t}\,dt$
    \task $\ds\int x\parens{1+2x^4}\,dx$
  \end{tasks}
  \vspace*{\stretch{1}}
  \pagebreak
  
  \begin{tasks}[after-item-skip=\stretch{1}, label=\mbox{}](2)
    \task $\ds\int\dfrac{\cos\theta}{\sin^2\theta}\,d\theta$
    \task $\ds\int\dfrac{\sin(2x)}{\sin(x)}\,dx$
    \task $\ds\int\parens{\pi+\dfrac{2}{yt}}\,dt$
    \task $\ds\int\dfrac{t^2-e^{2t}}{t+e^t}\,dt$
  \end{tasks}
  \vspace*{\stretch{1}}
  \pagebreak
  
  \begin{defn*}
    \begin{itemize}
      \item 
        An equation involving an unknown function and its derivative is called a \textbf{differential equation}.
      \item 
        An \textbf{initial condition} allows us to determine the arbitrary constant $C$.
      \item 
        A differential equation coupled with an initial condition is called an \textbf{initial value problem}.
    \end{itemize}
    
    \[\begin{array}{r@{\,=\,}ll@{\qquad}l}
      f'(x)& g(x),& \textnormal{where }g\textnormal{ is given, and}& \textnormal{\color{blue!90}Differential equation}\\
      f(a)& b,& \textnormal{where }a\textnormal{ and }b\textnormal{ are given.}& \textnormal{\color{blue!90}Initial condition}
    \end{array}\]
  \end{defn*}
  \begin{ex*}
    Solve the initial value problem:
  \end{ex*}
  \begin{tasks}[after-item-skip=\stretch{1}, label=\mbox{}](2)
    \task $\ds\dydx=9x^2-4x+5,\ y(-1)=0$
    \task $f'(x)=8x^3+12x+3,\ f(1)=6$
    \task $f'(x)=1+3\sqrt{x},\ f(4)=25$
    \task $\dfrac{dr}{d\theta}=\cos(\pi\theta),\ r(0)=1$
  \end{tasks}
  \vspace*{\stretch{1}}
  \pagebreak
  
  \begin{tasks}[after-item-skip=\stretch{1}, label=\mbox{}](1)
    \task $f'(t)=2\cos(t)+\sec^2(t),\ \ -\dfrac{\pi}{2}<t<\dfrac{\pi}{2},\ \ f\parens{\dfrac{\pi}{3}}=4$
    \task $g'(x)=7x\parens{x^6-\frac{1}{7}};\ \ g(1)=2$
    \task $f'''(x)=\sin(x),\ \ f(0)=1,\ \ f'(0)=1,\ \ f''(0)=1$
  \end{tasks}
  \vspace*{\stretch{1.5}}
  \pagebreak
  
  \begin{tasks}[after-item-skip=\stretch{1}, label=\mbox{}](1)
    \task $f'(x)=\dfrac{4}{\sqrt{1-x^2}},\ f\parens{\dfrac{1}{2}}=1$
    \task $\dfrac{d^2r}{dt^2}=\dfrac{2}{t^3};\ \ \left.\dfrac{dr}{dt}\right|_{t=1}=1,\ \ r(1)=1$
    \task $f''(x)=4+6x+4x^2,\ f(0)=3,\ f(1)=10$
  \end{tasks}
  \vspace*{\stretch{1}}
  \pagebreak
  
  \noindent
  \fbox{\parbox{0.9875\linewidth}{
    \textbf{Initial Value Problems for Velocity and Position}
    
    Suppose an object moves along a line with a (known) velocity $v(t)$, for $t\geq 0$. Then its position is found by solving the initial value problem.
      \[s'(t)=v(t),\ s(0)=s_0,\ \textnormal{where }s_0\textnormal{ is the (known) initial position.}\]
    If the (known) acceleration of the object $a(t)$ is given, then its velocity is found by solving the initial value problem
      \[v'(t)=a(t),\ v(0)=v_0,\ \textnormal{where }v_0\textnormal{ is the (known) initial velocity.}\]
  }}
  \begin{quote}
    \textit{Recall:}
      \[\begin{array}{ll}
          \textnormal{Position }& s(t)\\
          \textnormal{Velocity }& v(t)=s'(t)\\
          \textnormal{Acceleration }& a(t)=v'(t)=s''(t)
        \end{array}\]
  \end{quote}
  \begin{ex*}
    Solve the following velocity and position initial value problems
  \end{ex*}
  \begin{tasks}[after-item-skip=\stretch{1}, label=\mbox{}](1)
    \task $v(t)=\sin(t)+3\cos(t);\ \ s(0)=4$
    \task $a(t)=2e^t-12,\ \ v(0)=1,\ \ s(0)=0$
  \end{tasks}
  \vspace*{\stretch{1}}
  \pagebreak
  
  \begin{ex*}
    The acceleration of gravity near the surface of Mars is $3.72 m/s^2$. A rock is thrown straight up from the surface with an initial velocity of $23\ m/s$. How high does the rock go?
  \end{ex*}
  \begin{tasks}[after-item-skip=\stretch{1}](1)
    \task Write the initial value problem
    \task Find the velocity and position functions.
    \task Maximum height is reached when velocity is 0. Find the time when this happens and the maximum height.
  \end{tasks}
  \vspace*{\stretch{1}}
  \pagebreak
  
  \begin{ex*}
    A ball is thrown vertically upward from a height of 48 feet above ground at a speed of $32 ft/s$. Assume the acceleration due to gravity is $32\ ft/s^2$.
  \end{ex*}
  \begin{tasks}[after-item-skip=\stretch{1}](1)
    \task How high above the ground will it get?
    \task How long after it is thrown will it hit the ground?
  \end{tasks}
  \vspace*{\stretch{1}}
  \pagebreak
  
  \begin{ex*}
    A stone was dropped off a cliff and hits the ground with a speed of $120 ft/s$. What is the height of the cliff (assuming $a(t)=-32 ft/s^2$).
  \end{ex*}
  \vspace*{\stretch{1}}
  
  \begin{ex*}
    Find the antiderivative of $f(x)=\dfrac{2+x^2}{1+x^2}$
  \end{ex*}
  \vspace*{\stretch{1}}
  \pagebreak
  
\end{document}
