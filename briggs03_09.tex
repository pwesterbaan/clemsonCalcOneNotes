\documentclass[mathNotesPreamble]{subfiles}
\begin{document}
%\relscale{1.4}
\section{3.9: Derivatives of Logarithmic and Exponential Functions}
  Recall that $y=\log_a(x)$ and $y=a^x$ are inverse functions:

    \noindent
    \begin{minipage}{0.65\linewidth}
      \fbox{\parbox{0.9875\linewidth}{\centering
        \textbf{Inverse Properites of $a^x$ and $\log_a(x)$}
        
        \begin{enumerate}
          \item $a^{\log_a(x)}=x$, for $x>0$, and $\log_a(a^x)=x$, for all $x$.
          \item $y=\log_a(x)$ if and only if $x=a^y$.
          \item For real numbers $x$ and $b>0$, $b^x=a^{\log_a(b^x)}=a^{x\log_a(b)}$.
        \end{enumerate}
      }}
    \end{minipage}%
    \begin{minipage}{0.35\linewidth}
      \begin{flushright}
        \begin{tikzpicture}[scale=0.8]
          \begin{axis}[
            axis lines=center,
            axis line style={->},
            axis equal,
            xmin=-4.25, xmax=4.25,
            ymin=-4.25, ymax=4.25,
            xmajorticks=false,
            ymajorticks=false,
            ticklabel style={font=\tiny,inner sep=0.75pt,fill=white},
            xlabel=$x$, xlabel style={at={(ticklabel* cs:1)},anchor=north west},
            ylabel=$y$, ylabel style={at={(ticklabel* cs:1)},anchor=south west},
            every axis plot/.append style={line width=1.25pt}
            ]
            \addplot[<->] expression[domain=-5.125:1.45, ClemsonPurple, samples=100] {e^x} node[black, right, pos=0.95, fill=white, xshift=3pt] {$a^x$};
            \addplot[<->] expression[domain=0.0142:5.1, ClemsonOrange,  samples=100] {ln(x)} node[black, above, pos=0.875, fill=white, yshift=5pt] {$\log_a(x)$};
            \addplot[dashed] expression[domain=-5.5:5.5, black!50] {x};
            \addplot[soldot, black, mark size=2pt] coordinates{(0,1)(1,0)};
            \node[anchor = north west] at (1,0) {$(1,0)$};
            \node[anchor = south east] at (0,1) {$(0,1)$};
          \end{axis}
        \end{tikzpicture}
      \end{flushright}
    \end{minipage}%

  \vfill
  \noindent
    \fbox{\parbox{0.9875\linewidth}{
      \textbf{Theorem 3.15: Derivative of $\ln(x)$.}
        $$\ddx\parens{\ln(x)}=\frac{1}{x}, \text{ for } x>0\qquad \ddx\parens{\ln\abs{x}}=\frac{1}{x}, \text{ for } x\neq 0$$
      
      If $u$ is differentiable at $x$ and $u(x)\neq 0$, then
        $$\ddx\parens{\ln\abs{u(x)}}=\frac{u'(x)}{u(x)}$$
    }}
  \vfill
  \begin{center}
    \begin{tikzpicture}
      \begin{groupplot}[
        group style={group size=2 by 1, horizontal sep=50pt},
        axis lines=center,
        axis line style={->},
        xmax=3.75,
        ymin=-4, ymax=4,
        xtick={-5,-3,...,7},
        ytick={-5,-3,...,7},
        ticklabel style={font=\footnotesize,inner sep=0.5pt,fill=white,opacity=1.0, text opacity=1},
        xlabel=$x$, xlabel style={at={(ticklabel* cs:1)},anchor=north west},
        ylabel=$y$, ylabel style={at={(ticklabel* cs:1)},anchor=south west},
        every axis plot/.append style={line width=0.95pt, color=blue}
        ]
        \nextgroupplot[  
          xmin=-0.75, 
          ]
          \addplot[->] expression[domain=exp(-4):3.75, black, samples=301]{ln(x)}
            node[above, pos=0.9] {$y=\ln(x)$};
          \addplot[->] expression[domain=0.25:3.75, red, samples=100]{1/x}
            node[right, pos=0.15] {$y'=\dfrac{1}{x},\ x>0$};
        \nextgroupplot[
          xmin=-3.75, 
          ]
          \addplot[->] expression[domain=exp(-4):3.75, black, samples=301]{ln(x)}
            node[above, pos=0.825, yshift=5pt] {$y=\ln\abs{x}$};
          \addplot[->] expression[domain=0.25:3.75, red, samples=100]{1/x}
            node[left, pos=0.15, xshift=-15pt] {$y'=\dfrac{1}{x}, x\neq0$};
          \addplot[<-] expression[domain=-3.75:-exp(-4), black, samples=301]{ln(-x)};
          \addplot[<-] expression[domain=-3.75:-0.25, red, samples=100]{1/x};
      \end{groupplot}
    \end{tikzpicture}
  \end{center}
  \vfill
  \pagebreak
  
  \begin{ex*}
    Use implicit differentiation to prove $\ds\ddx\ln(x)=\frac{1}{x}$. 
    
    \noindent 
    Then, use the piecewise definition of $\abs{x}$ to prove that $\ds\ddx\ln\abs{x}=\frac{1}{x}$.
  \end{ex*}
  \vfill
  
  \begin{ex*}
    Find the derivatives of the following functions:
  \end{ex*}
  \begin{tasks}[after-item-skip=\stretch{1}, label=~](3)
    \task $y=\ln(x)$
    \task $y=\ln(4x)$
    \task $y=\ln\parens{4x^2+2}$
    \task $f(x)=\sqrt x\ln(x^2)$
    \task $f(x)=\ln\parens{\frac{10}{x}}$
    \task $f(x)=\dfrac{\ln(x)}{1+\ln(x)}$
  \end{tasks}
  \vfill
  
  \pagebreak
  \begin{tasks}[after-item-skip=\stretch{1}, label=~](2)
    \task $f(x)=\sqrt[5]{\ln\parens{3x^4}}$
    \task $f(x)=\ln\sqrt[5]{3x}$
    \task $f(x)=\ln\parens{\ln\parens{\ln(4x)}}$
    \task $f(x)=\ln\abs{x^2-1}$
    \task $y=\ln\parens{\sec^2\theta}$
    \task $y=\parens{\ln\parens{\sin(3x)}}^2$
  \end{tasks}
  \vfill 
  \pagebreak
  
  \noindent
    \fbox{\parbox{0.9875\linewidth}{
      \textbf{Theorem 3.16: Derivative of $b^x$.}
      
      If $b>0$ and $b\neq 1$, then for all $x$.
        $$\ddx\sbrkt{b^x}=b^x\ln(b).$$
    }}
  \begin{ex*}
    Using the properties of exponents and logarithms, prove the above theorem. 
    
    \noindent
    Extend this theorem by stating the derivative of $y=b^{f(x)}$.
  \end{ex*}
  \vfill
  
  \begin{ex*}
    Find the derivatives of the following functions:
  \end{ex*}
  \begin{tasks}[after-item-skip=\stretch{1}, label=~](2)
    \task $y=5^{3x}$
    \task $s(t)=\cos(2^t)$
    \task $g(v)=10^v\parens{\ln(10^v)-1}$
    \task $y=6^{x\ln(x)}$
  \end{tasks}
  \vfill 
  
  \pagebreak
  
  \noindent
    \fbox{\parbox{0.9875\linewidth}{
      \textbf{Theorem 3.18: Derivative of $\log_b(x)$.}
      
      If $b>0$ and $b\neq 1$, then
        \[\ddx\sbrkt{\log_b(x)}=\frac{1}{x\ln b},\text{ for }x>0 \text{ and }\ddx\sbrkt{\log_b\abs{x}}=\frac{1}{x\ln b},\text{ for } x\neq 0.\]
    }}

  \begin{ex*}
    Using the properties of exponents and logarithms, prove the above theorem. 
    
    \noindent
    Extend this theorem by stating the derivative of $y=\log_b\parens{g(x)}$.
  \end{ex*}
  \vfill
  \begin{ex*}
    Find the derivatives of the following functions:
  \end{ex*}
  \begin{tasks}[after-item-skip=\stretch{1}, label=~](2)
    \task $f(x)=\log_4\parens{4x^2+3x}$
    \task $f(x)=\log_5\parens{x e^x}$
    \task $y=2x\log_{10}\sqrt x$
    \task $y=\dfrac{\log_3\parens{\tan\parens{e^2x}}}{\pi\cdot e\inv[4x]}$
  \end{tasks}
  \vfill
  \pagebreak
  
  \begin{center}
    \vfill
    \setlength{\jot}{10pt}
    \noindent
    \fbox{\parbox{0.9875\linewidth}{
      \textbf{Derivative rules for exponential functions:}
      \begin{align*}
        \hspace*{-10pt}
        \ddx \sbrkt{e^x}&=e^x&  \ddx \sbrkt{e^{f(x)}}&=e^{f(x)}\cdot f'(x)\\
        \hspace*{-10pt}
        \ddx \sbrkt{b^x}&=\ln(b)\cdot b^x&  \ddx \sbrkt{b^{g(x)}}&=\ln(b)\cdot b^{g(x)}\cdot g'(x)\hspace*{-10pt}
      \end{align*}
    }}
    \vfill
    
    \fbox{\parbox{0.9875\linewidth}{
      \textbf{Derivative rules for logarithmic functions:}
      \begin{align*}
        \hspace*{-10pt}
        &\ddx\sbrkt{\ln x}=\frac{1}{x}& &\ddx\sbrkt{\ln\parens{f(x)}}=\frac{1}{f(x)}\cdot f'(x)\hspace*{-10pt}\\
        \hspace*{-10pt}
        &\ddx\sbrkt{\log_b(x)}=\frac{1}{\ln(b) x}& &\ddx\sbrkt{\log_b\parens{g(x)}}=\frac{g'(x)}{\ln(b) g(x)}
      \end{align*} 
   }}
   \vfill
   \fbox{\parbox{0.9875\linewidth}{\textbf{Laws of Logarithms}
      
      For $x,y>0$:
      \begin{tasks}[after-item-skip=10pt, label=~](2)
        \task $\log_a(xy)=\log_a(x)+\log_a(y)$
        \task $\log_a\parens{\frac{x}{y}}=\log_a(x)-\log_a(x)$
        \task $\log_a(x^r)=r\log_a(x)$
        \task $\log_a(1)=0$
        \task $\log_a(x)=\dfrac{\log_b(x)}{\log_b(a)}$
        \task $\log_a(a)=1$
      \end{tasks}
    }}
    \vfill
  \end{center}
  \pagebreak
  
  \begin{ex*}
    For the following functions, use the laws of logarithms to rewrite the function before taking the derivative:
  \end{ex*}
  
  \noindent
  \begin{tasks}[after-item-skip=\stretch{1}, label=~](1)
    \task $F(t)=\ln\parens{\dfrac{\parens{2t+1}^3}{\parens{3t+1}^4}}$
    \task $y=\ln\sqrt[3]{\dfrac{1+x}{1-x}}$
    \task $y=\ln\sqrt{\dfrac{\parens{x+1}^5}{\parens{x+2}^{20}}}$
  \end{tasks}
  \vfill 
  
  \pagebreak
  \noindent
  \fbox{\parbox{0.9875\linewidth}{
    \textbf{Logarithmic Differentiation:}
    
    \begin{enumerate}
      \item Take the natural logarithm of both sides of the equation.
      \item Use logarithm laws to simplify.
      \item Use implicit differentiation to take the derivative of both sides.
      \item Solve for $\dydx$.
    \end{enumerate}
  }}
  \begin{ex*}
    Find the derivatives of the following functions:
  \end{ex*}
  \begin{tasks}[after-item-skip=\stretch{1}, label=~](2)
    \task $y=\dfrac{\sin^2(x)\tan^4(x)}{\parens{x^2+1}^2}$
    \task $h(\theta)=\dfrac{\theta \sin(\theta)}{\sqrt{\sec(\theta)}}$
    \task $g(x)=\sqrt[10]{\dfrac{3x+4}{2x-4}}$
    \task $y=\dfrac{e\inv[x]\cos^2(x)}{x^2+x+1}$
  \end{tasks}
  \vfill
  \pagebreak
  
  \noindent
  \textit{Note:} Whenever the function is of the form $f(x)^{g(x)}$, then \textit{Logarithmic Differentiation} is the only option!
  \begin{tasks}[after-item-skip=\stretch{1}, label=~](2)
    \task $y=x^x$
    \task $y=\parens{\ln(x)}^x$
    \task $y=\parens{\tan x}^\frac{1}{x}$
    \task $y=\parens{2x}^{3x}$
  \end{tasks}
  \vfill 
  \pagebreak
  
  \begin{ex*}
    Use the definition of the derivative to evaluate the following limits:
  \end{ex*}
  \begin{tasks}[after-item-skip=\stretch{1}, label=~](2)
    \task $\ds\lim_{x\to e} \dfrac{\ln(x)-1}{x-e}$
    \task $\ds\lim_{h\to 0} \dfrac{\ln(e^8+h)-8}{h}$
    \task $\ds\lim_{x\to 2} \dfrac{5^x-25}{x-2}$
    \task $\ds\lim_{h\to 0} \dfrac{\parens{3+h}^{3+h}-27}{h}$
  \end{tasks}
  \vfill
  \pagebreak
  
  %TODO comment out \relsize!
\end{document}
