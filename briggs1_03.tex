\documentclass[answers]{exam}
\usepackage{texPreamble}
\usepackage{relsize}
\usepackage{tabularx}
\extraheadheight{0.25in}
\extrafootheight{1.0in}
\extrawidth{1in}
% ----------------------------------------------------------------

\begin{document}
%\relscale{1.4}
  \section{1.3: Inverse, Exponential and Logarithmic Functions}
  \begin{defn*}
    An \textbf{exponential function} has the form
      \[f(x)=b^x\]
    where $b\neq 1$ is a positive real number. Exponential functions have a horizontal asymptote of $y=\underline{\hspace*{20pt}}$ and $y$-intercept of $(0,\underline{\hspace*{10pt}})$. When $b$ is such that $0<b<1$, then $f(x)$ is \underline{\hspace*{2.25in}} and when $b>1$, then $f(x)$ is \underline{\hspace*{2.25in}}. Exponential functions have domain \underline{\hspace*{35pt}} and range \underline{\hspace*{35pt}}.
  \end{defn*}

  \vspace*{\stretch{1}}
  \begin{center}
    \begin{tikzpicture}[scale=1.0]
      \begin{groupplot}[
        group style={group size=2 by 1, horizontal sep=2cm},
        grid style={line width=0.35pt, draw=gray!75},
        axis lines=center,
        axis line style={->},
        xmin=-2, xmax=2,
        ymin=-0.5, ymax=4,
        ytick={1},
        xmajorticks=false,
        ticklabel style={font=\footnotesize,inner sep=0.5pt,fill=white,opacity=1.0, text opacity=1},
        xlabel=$x$, xlabel style={at={(ticklabel* cs:1)},anchor=north west},
        ylabel=$y$, ylabel style={at={(ticklabel* cs:1)},anchor=south west},
        every axis plot/.append style={line width=1.0pt, color=blue, samples=100}
        ]
          \nextgroupplot[legend pos=north west]
            \addplot[-] expression[domain=-2:2,samples=100]{exp(x)}; \addlegendentry{$e^x$};
            \addplot[-] expression[domain=-2:2,samples=100, ClemsonOrange]{2^x}; \addlegendentry{$2^x$};
            \addplot[-] expression[domain=-2:2,samples=100, ClemsonPurple]{10^x}; \addlegendentry{$10^x$};
          \nextgroupplot[legend pos=north east]
            \addplot[-] expression[domain=-2:2,samples=100]{exp(-x)}; \addlegendentry{$\parens{\sfrac{1}{e}}^x$}
            \addplot[-] expression[domain=-2:2,samples=100, ClemsonOrange]{2^(-x)}; \addlegendentry{$\parens{\sfrac{1}{2}}^x$}
            \addplot[-] expression[domain=-2:2,samples=100, ClemsonPurple]{10^(-x)}; \addlegendentry{$\parens{\sfrac{1}{10}}^x$}
      \end{groupplot}
    \end{tikzpicture}
  \end{center}
  
  \vspace*{\stretch{1}}
  \begin{defn*}
    The \textbf{natural exponential function} is 
      \[f(x)=e^x.\]
    where $e$ is the irrational constant $e\approx2.718281828459045\dots$.
  \end{defn*}
  \pagebreak
  
  \noindent
  \fbox{\parbox{0.9875\linewidth}{
  \textbf{Laws of Exponents:}  
  For $a>0$, we have the following laws:
  \begin{tasks}(2)
    \task $a^{x+y}=a^xa^y$
    \task $a^{x-y}=\frac{a^x}{a^y}$
    \task $\parens{a^x}^y=a^{xy}$
    \task $\parens{ab}^x=a^xb^x$
  \end{tasks}
  }}
  \begin{ex*}
    For the following expressions, use the Laws of Exponents to simplify:
  \end{ex*}
  \begin{tasks}[after-item-skip=\stretch{1}](2)
    \task $\parens{x^2y^3}^5$
    \task $\parens{\sqrt 3}^{\sfrac{1}{2}}\cdot\parens{\sqrt{12}}^{\sfrac{1}{2}}$
    \task $\parens{\dfrac{x\inv[2]}{x^8}}\inv[2]$
    \task $\parens{\dfrac{-1}{27}}^{\sfrac{4}{3}}$
  \end{tasks}
  \vspace*{\stretch{1}}
  \pagebreak
  
  \begin{defn*}[One-to-One Functions and the Horizontal Line Test]
  
    A function $f$ is \textbf{one-to-one} on a domain $D$ if each value of $f(x)$ corresponds to exactly one value of $x$ in $D$. More precisely, $f$ is one-to-one on $d$ if $f(x_1)\neq f(x_2)$ whenever $x_1\neq x_2$, for $x_1$ and $x_2$ in $D$. 
    
    The \textbf{horizontal line test} says that every horizontal line intersects the graph of a one-to-one function at most once.
  \end{defn*}
  
  \vfill
  \begin{center}
    \begin{tikzpicture}[scale=1.1]
      \begin{groupplot}[
        group style={group size=2 by 1, horizontal sep=2cm},
        grid style={line width=0.35pt, draw=gray!75},
        axis lines=center,
        axis line style={->},
        xmin=-0.5, xmax=2,
        ymin=-2, ymax=6,
        ticklabel style={font=\footnotesize,inner sep=0.5pt,fill=white,opacity=1.0, text opacity=1},
        xlabel=$x$, xlabel style={at={(ticklabel* cs:1)},anchor=north west},
        ylabel=$y$, ylabel style={at={(ticklabel* cs:1)},anchor=south west},
        every axis plot/.append style={line width=0.95pt, color=blue, samples=100}
        ]
        \nextgroupplot[
          xtick={1.5},
          xticklabel style={yshift=-5pt},
          xticklabels={\Large $x_1$},
          ytick={2.98168907},
          yticklabels={\Large $y_1$},
          ] 
          \addplot[->] expression[domain=-0.5:2, samples=100]{exp(x)-1.5};
          \draw[dashed, red] (axis cs: 1.5,0) -- (axis cs: 1.5,2.98168907) -- (axis cs: 0,2.98168907);
          \addplot[soldot, black] coordinates{(1.5,2.98168907)};
        \nextgroupplot [
          xtick={-0.25,1,1.5},
          xticklabel style={yshift=-5pt},
          xticklabels={\Large $x_1$,\Large $x_2$,\Large $x_3$},
          yticklabel style={yshift=7.5pt},
          ytick={1},
          yticklabels={\Large $y$},
          ] 
          \addplot[->] expression[domain=-0.5:1.92, samples=100]{6*(x+0.25)*(x-1)*(x-1.5)+1};
          \draw[dashed, red] (axis cs: -0.25,0) -- (axis cs: -0.25,1) -- (axis cs: 1.5,1) -- (axis cs: 1.5,0);
          \draw[dashed, red] (axis cs: 1,1) -- (axis cs: 1,0);
          \addplot[soldot, black] coordinates{(-0.25,1)(1,1)(1.5,1)};
        \end{groupplot}
    \end{tikzpicture}
  \end{center}

  \vspace*{\stretch{2}}
  \pagebreak

  \noindent
  \fbox{\parbox{0.9875\linewidth}{
    \textbf{Finding an Inverse Function}
    Suppose $f$ is one-to-one on an interval $I$. To find $f\inv$, use the following steps:
    \begin{enumerate}
      \item Solve $y=f(x)$ for $x$. If necessary, choose the function that corresponds to $I$.
      \item Interchange $x$ and $y$ and write $y=f\inv(x)$.
    \end{enumerate}
  }}
  \begin{ex*}
    Find $f\inv(x)$:
    \begin{enumerate}[label=, itemsep=50pt]
      \item $f(x)=x^2-2x+1,\ x\geq 1$
      \item $g(x)=\dfrac{x}{2}-\dfrac{7}{2}$
      \item $h(x)=\sqrt[3]{5x+1}$
      \item $j(x)=\dfrac{2x}{1-x}$
      \item $k(x)=e^x$
    \end{enumerate}
  \end{ex*}
  \pagebreak
  
  \noindent
  \fbox{\parbox{0.9875\linewidth}{ \textbf{Existence of Inverse Functions}
  
  Let $f$ be a one-to-one function on a domain $D$ with a range $R$. Then $f$ has a unique inverse $f\inv$ with domain $R$ and range $D$ such that
    \[f\inv\parens{f(x)}=x\hspace*{30pt}\text{ and }\hspace*{30pt}f\parens{f\inv(y)}=y\]
  where $x$ is in $D$ and $y$ is in $R$.
  }}
  \vspace*{15pt}
  \begin{ex*}
    For $f(x)=\sqrt[3]{4x-1}+2$, show that $f\inv\parens{f(x)}=f\parens{f\inv(x)}=x$
  \end{ex*}
  
  \pagebreak
  \textit{Note:} A function is symmetric with it's inverse with respect to $y=x$.
  \vspace*{15pt}
  \begin{center}
    \begin{tabularx}{\linewidth}{*{3}{>{\centering\arraybackslash}X}}
      \begin{tabular}{@{}R@{\ =\ }L@{}}
        f(x)&\sqrt x\\
        f\inv(x)& x^2,\ x>0
      \end{tabular}&
      \begin{tabular}{@{}R@{\ =\ }L@{}}
        f(x)&x^3\\
        f\inv(x)& \sqrt[3]{x}=x^{\sfrac{1}{3}}
      \end{tabular}&  
      \begin{tabular}{@{}R@{\ =\ }L@{}}
        f(x)&\sin x \text{ on } \sbrkt{-\sfrac{\pi}{2},\sfrac{\pi}{2}}\\
        f\inv(x)& \sin\inv x
      \end{tabular}\\
    \end{tabularx}
    \begin{tikzpicture}[scale=0.75]
      \begin{groupplot}[
        group style={group size=3 by 1, horizontal sep=1.5cm},
        axis lines=center,
        axis line style={->},
        axis equal,
        xmin=-1, xmax=3,
        ymin=-1, ymax=3,
        xtick={-4,-3,...,4},
        ytick={-2,-1,...,6},
        enlargelimits={abs=0.75},
        ticklabel style={font=\tiny, inner sep=0.75pt,fill=white},
        xlabel=$x$, xlabel style={at={(ticklabel* cs:1)},anchor=north west},
        ylabel=$y$, ylabel style={at={(ticklabel* cs:1)},anchor=south west},
        every axis plot/.append style={line width=0.95pt}
        ]
      \nextgroupplot
        \addplot[->] expression[domain=0:4, blue, samples=50] {sqrt(x)};
        \addplot[->] expression[domain=0:1.9, red, samples=50] {x^2};
        \addplot[dashed] expression[domain=-2.5:4.5, black!50] {x};
      \nextgroupplot[
        xmin=-3, xmax=3,
        ymin=-3, ymax=3,
        ytick={-6,-5,...,6},
        ]
        \addplot[<->] expression[domain=-1.55:1.55, blue, samples=50] {x^3};
        \addplot[<->] expression[domain=-4.25:4.25, red, samples=100] {x/abs(x)*abs(x)^(1/3)};
        \addplot[dashed] expression[domain=-2.5:4.5, black!50] {x};
      \nextgroupplot[
        xmin=-0.45*pi, xmax=0.45*pi,
        ymin=-0.45*pi, ymax=0.45*pi,
        ]
        \addplot[<->] expression[domain=-0.5*pi:0.5*pi, blue, samples=50] {sin(\x r)};
        \addplot[<->] expression[domain=-1:1, red, samples=100] {rad(asin(x))};
        \addplot[dashed] expression[domain=-3:3, black!50] {x};
      \end{groupplot}
    \end{tikzpicture}
  \end{center}
  
  \vspace*{\stretch{1}}
  \begin{ex*}
    Draw the function inverses:
    
    \noindent
    \begin{minipage}[t]{0.5\linewidth}
      \begin{center}
        $$f(x)=2^x$$
        \begin{tikzpicture}[scale=1.25]
          \begin{axis}[
            axis lines=center,
            axis line style={->},
            axis equal,
            xmin=-5, xmax=5,
            xmajorticks=false,
            ymajorticks=false,
            every axis plot/.append style={line width=0.95pt}
            ]
            \addplot[->] expression[domain=-6:2.325, samples=100, blue]{2^x};
            \addplot[dashed] expression[domain=-5:5]{x};
          \end{axis}
        \end{tikzpicture}
      \end{center}
    \end{minipage}%
    \begin{minipage}[t]{0.5\linewidth}
      \begin{center}
        $$f(x)=\sqrt{x+1}-2$$
        \begin{tikzpicture}[scale=1.25]
          \begin{axis}[
            axis lines=center,
            axis line style={->},
            axis equal,
            xmin=-2, xmax=2,
            xmajorticks=false,
            ymajorticks=false,
            every axis plot/.append style={line width=0.95pt}
            ]
            \addplot[->] expression[domain=-1:2, samples=501, blue]{sqrt(x+1)-1};
            \addplot[dashed] expression[domain=-5:5]{x};
          \end{axis}
        \end{tikzpicture}
      \end{center}
    \end{minipage}
  \end{ex*}
  \pagebreak

  \begin{defn*}[Logarithmic Function Base $b$]
    For any base $b>0$, with $b\neq 1$, the \textbf{logarithmic function base $b$}, denoted $y~=~\log_b(x)$, is the inverse of the exponential function $y=b^x$. The inverse of the natural exponential function with base $b=e$ is the \textbf{natural logarithm function}, denoted $y=\ln(x)$.
  \end{defn*}
  
  \noindent
  \begin{minipage}{0.5\linewidth}
    \begin{center}
      \begin{tikzpicture}
        \begin{axis}[
          axis lines=center,
          axis line style={->},
          axis equal,
          xmin=-4.25, xmax=4.25,
          ymin=-4.25, ymax=4.25,
          xmajorticks=false,
          ymajorticks=false,
          ticklabel style={font=\tiny,inner sep=0.75pt,fill=white},
          xlabel=$x$, xlabel style={at={(ticklabel* cs:1)},anchor=north west},
          ylabel=$y$, ylabel style={at={(ticklabel* cs:1)},anchor=south west},
          every axis plot/.append style={line width=1.25pt}
          ]
          \addplot[->] expression[domain=-5.125:1.45, ClemsonPurple, samples=100] {e^x} node[black, right, pos=0.95, fill=white, xshift=3pt] {$a^x$};
          \addplot[->] expression[domain=0.0142:5.1, ClemsonOrange,  samples=100] {ln(x)} node[black, above, pos=0.875, fill=white, yshift=5pt] {$\log_a(x)$};
          \addplot[dashed] expression[domain=-5.5:5.5, black!50] {x};
        \end{axis}
      \end{tikzpicture}
    \end{center}
  \end{minipage}%
  \begin{minipage}{0.5\linewidth}
    \textit{Note:}
    \begin{center}
      \begin{tabular}{@{}lCC@{}}\toprule
        & a^x& \log_a(x)\\\midrule
        Domain:& (-\infty,\infty)& (0,\infty)\\
        Range: & (0,\infty)& (-\infty,\infty)\\\bottomrule
      \end{tabular}
    \end{center}
    \vspace*{55pt}
  \end{minipage}%
  
  \vspace*{-50pt}
  \begin{center}
    \fbox{\parbox{0.37\linewidth}{
    \centering $y=b^x\quad\iff\quad\log_b(y)=x$
    
    Think ``the base stays the base''
    }}
  \end{center}
  \begin{ex*}
    Evaluate:
    \begin{extasks}[after-item-skip=\stretch{1}](2)
      \task $\log_9(81)$
      \task $\log_3(\sqrt3)$
      \task $\log_{\frac{1}{2}}(8)$
      \task $\parens{\log_5(5\inv[3])}^2$
    \end{extasks}
  \end{ex*}
  \vspace*{\stretch{1}}
  \textit{Note:} In this course, the \textbf{common logarithm} is $\log_{10}(x)$ and is denoted by $\log(x)$. 
  
  -- Sometimes other disciplines use $\log(x)$ to represent other bases.
  \begin{ex*}
    Evaluate:
    \begin{extasks}(2)
      \task $\log 100000$
      \task $\log \frac{1}{1000}$
    \end{extasks}
  \end{ex*}
  \vspace*{\stretch{1}}
  
  \pagebreak
  Recall that for a function $f$ and its inverse $g$:
  \begin{tasks}[style=itemize](2)
    \task $f\parens{g(x)}=x$
    \task Domain of $f$=Range of $g$
    \task $g\parens{f(x)}=x$
    \task Domain of $g$=Range of $f$\\
  \end{tasks}
  \begin{center}
    \fbox{\parbox{0.9\linewidth}{\textbf{Inverse Relations for Exponential and Logarithmic Functions}
    
    For any base $b>0$, with $b\neq 1$, the following inverse relations hold:
      $$b^{\log_bx}=x \hspace*{0.2\linewidth}\log_b(b^x)=x,\text{ for all real values of }x$$
    }}
  \end{center}
  \begin{ex*}
    Evaluate:
    \begin{extasks}(3)
      \task $2^{\log_28}$
      \task $\log_bb^\pi$
      \task $\log10^3$
    \end{extasks}
  \end{ex*}
  \vfill

  \begin{ex*}
    Write each expression in terms of one logarithm:\\
    
    \noindent
    \begin{minipage}{0.6\linewidth}
      \begin{itemize}
        \item[] $\log_2 6-\log_2 15+\log_2 20$\\[50pt]
        \item[] $\log_3 100-\parens{\log_3 18+\log_3 50}$\\[50pt]
      \end{itemize}
    \end{minipage}%
    \begin{minipage}{0.4\linewidth}
      \begin{flushright}
        \fbox{\parbox{0.9\linewidth}{\textbf{Laws of Logarithms}
      
          For $x,y>0$:
            \begin{enumerate}
              \item $\log_a(xy)=\log_a(x)+\log_a(y)$
              \item $\log_a\parens{\frac{x}{y}}=\log_a(x)-\log_a(x)$
              \item $\log_a(x^r)=r\log_a(x)$
              \item $\log_a(1)=0$
              \item $\log_a(x)=\dfrac{\log_b(x)}{\log_b(a)}$
            \end{enumerate}
          }}
      \end{flushright}
    \end{minipage}
  \end{ex*}
  \pagebreak
  \begin{ex*}
    Solve each equation checking for extraneous solutions:
    \begin{enumerate}[label=, itemsep=\stretch{1}]
      \item $\log_{64}x^2=\frac{1}{3}$
      \item $\log(3x+2)+\log(x-1)=2$
      \item $\log_2 x^2-\log_2(3x-8)=2$
      \item $\log_4 x-\log_4(x-1)=\frac{1}{2}$
      \item $\log_3(x+6)-\log_3(x-6)=2$
      \item $\log_3(x^2-5)=2$
    \end{enumerate}
  \end{ex*}
  \vspace*{\stretch{1}}
  \pagebreak

  \noindent
  \begin{defn*}
    The \textbf{Natural Logarithmic Function} uses base $e$ and is denoted $\ln(x)=\log_e x$.
  \end{defn*}
  \begin{minipage}{0.5\linewidth}
    
    \begin{center}
      \textit{Note:} The natural log is the inverse of $e^x$:
      $$\ln(x)=y\ \iff\ e^y=x$$
    \end{center}
    \vspace*{5pt}
  \end{minipage}%
  \begin{minipage}{0.5\linewidth}
    \begin{flushright}
      \begin{tikzpicture}[scale=1.0]
        \begin{axis}[
          axis lines=center,
          axis line style={->},
          axis equal,
          xmin=-4.25, xmax=4.25,
          ymin=-4.25, ymax=4.25,
          xmajorticks=false,
          ymajorticks=false,
          ticklabel style={font=\tiny,inner sep=0.75pt,fill=white},
          xlabel=$x$, xlabel style={at={(ticklabel* cs:1)},anchor=north west},
          ylabel=$y$, ylabel style={at={(ticklabel* cs:1)},anchor=south west},
          every axis plot/.append style={line width=1.25pt}
          ]
          \addplot[->] expression[domain=-5.125:1.45, ClemsonPurple, samples=100] {e^x} node[black, right, pos=0.95, fill=white, xshift=3pt] {$e^x$};
          \addplot[->] expression[domain=0.0142:5.1, ClemsonOrange,  samples=100] {ln(x)} node[black, above, pos=0.875, fill=white, yshift=5pt] {$\ln(x)$};
          \addplot[dashed] expression[domain=-5.5:5.5, black!50] {x};
        \end{axis}
      \end{tikzpicture}
    \end{flushright}
  \end{minipage}
  \begin{center}
    \fbox{\parbox{0.75\linewidth}{\textbf{Inverse Properties for $a^x$ and $\log_a x$}
      \begin{center}
        \begin{tabular}{l@{\quad}*{2}{R@{\ =\ }L@{\qquad}}L}
          Base $a$:& a^{\log_a x}&x,& \log_a a^x&x,& a>0,~ a\neq 1,~ x>0\\
          Base $e$:& e^{\ln x}&x,   & \ln e^x   &x,& x>0
        \end{tabular}
      \end{center}
    }}
  \end{center}
  
  \begin{ex*}
    Simplify
  \end{ex*}
  \begin{extasks}[after-item-skip=\stretch{1}](2)
    \task $e\inv[\ln 0.3]$
    \task $e^{\ln \pi x-\ln 2}$
    \task $\ln\parens{\frac{1}{e}}$
    \task $e^{4\ln x}$
  \end{extasks}
  \vspace*{\stretch{1}}

  \pagebreak
  \begin{ex*}
    Write each expression in terms of one logarithm:
    
    \noindent
    \begin{minipage}[t]{0.55\linewidth}~
      \begin{extasks}[after-item-skip=2.25in](1)
        \task $\ln(a+b)+\ln(a-b)-2\ln c$
        \task $\frac{1}{3}\ln(x+2)^3 +\frac{1}{2}\sbrkt{\ln x-\ln(x^2+3x+2)^2}$
      \end{extasks}
    \end{minipage}%
    \begin{minipage}[t]{0.45\linewidth}~
      \begin{flushright}
        \fbox{\parbox{0.9\linewidth}{\textbf{Laws of the Natural Logarithm}
      
          For $x,y>0$:
            \begin{enumerate}
              \item $\ln(xy)=\ln(x)+\ln(y)$
              \item $\ln\parens{\frac{x}{y}}=\ln(x)-\ln(x)$
              \item $\ln(x^r)=r\ln(x)$
              \item $\ln(1)=0$
              \item $\log_a(x)=\dfrac{\ln(x)}{\ln(a)}$
            \end{enumerate}
          }}
      \end{flushright}
    \end{minipage}
    \vspace*{\stretch{1}}
  \end{ex*}
  \begin{center}
    \fbox{\parbox{0.9875\linewidth}{
      \textit{Note:} Many common mistakes come from using the logarithm rules incorrectly:
      $$\ln A-\ln B\neq \dfrac{\ln A}{\ln B}\qquad \ln(A+B)\neq\ln(A)\ln(B)$$
    }}
  \end{center}
  \pagebreak
  \begin{ex*}
    Solve:
    \begin{extasks}(2)
      \task $2^x=55$
      \task $5^{3x}=29$\\[40pt]
      \task $e^{2x}-5e^x-14=0$
      \task $4e^{2x}-7e^x=15$\\[40pt]
    \end{extasks}
  \end{ex*}
  \begin{ex*}
    Solve for $y$ in terms of $x$:
    \begin{extasks}(2)
      \task $\ln(y-40)=5x$
      \task $\ln(y^2-1)-\ln(y+1)=\ln(\sin x)$
    \end{extasks}
  \end{ex*}
  \vfill
  \begin{ex*}
    Solve:
    \begin{tasks}[after-item-skip=\stretch{1}](2)
      \task[] $\ln(t)+\ln(t^2)=6$
      \task[] $e^{x^2+2x-3}=1$
      \task[] $\ln x=-1$
      \task[] $e\inv[0.3t]=27$
    \end{tasks}
  \end{ex*}
  \vspace*{\stretch{1}}
  \pagebreak
\end{document}
