\documentclass[answers]{exam}
\usepackage{texPreamble}
\usepackage{relsize}
\usepackage{tabularx}
\extraheadheight{0.25in}
\extrafootheight{1.0in}
\extrawidth{1in}
% ----------------------------------------------------------------
\firstpagefootrule
\runningfootrule
\begin{document}
%\relscale{1.4}
\section{4.7: L'H\^ opital's Rule}
\noindent
\fbox{\parbox{0.9875\linewidth}{
  \textbf{Theorem 4.12: L'H\^opital's Rule}
  
  Suppose $f$ and $g$ are differentiable on an open interval $I$ containing $a$ with $g'(x)\neq 0$ on $I$ when $x\neq a$. If $\ds\lim_{x\to a} f(x)=\lim_{x\to a} g(x)=0$, then
    \[\lim_{x\to a}\frac{f(x)}{g(x)}=\lim_{x\to a}\frac{f'(x)}{g'(x)}\]
  provided the limit on the right exists (or is $\pm\infty$). The rule also applies if $x\to a$ is replaced with $x\to\pm\infty$, $x\to a^+$, or $x\to a\inv[]$.
}}

\vspace*{10pt}

\noindent
\fbox{\parbox{0.9875\linewidth}{
  \textbf{Theorem 4.13: L'H\^opital's Rule $(\infty/\infty)$}
  
  Suppose $f$ and $g$ are differentiable on an open interval $I$ containing $a$ with $g'(x)\neq 0$ on $I$ when $x\neq a$. If $\ds\lim_{x\to a} f(x)= \pm\infty$ and $\lim_{x\to a} g(x)=\pm\infty$, then
    \[\lim_{x\to a}\frac{f(x)}{g(x)}=\lim_{x\to a}\frac{f'(x)}{g'(x)}\]
  provided the limit on the right exists (or is $\pm\infty$). The rule also applies for $x\to\pm\infty$, $x\to a^+$, or $x\to a\inv[]$.
}}

\vspace*{5pt}
\hspace*{\stretch{1}}
\textit{Note:} Limits of the form $\dfrac{0}{0}$ and $\dfrac{\infty}{\infty}$ are called \textit{indeterminate forms}.
\hspace*{\stretch{1}}

\noindent\hrulefill

\textbf{Notes on grading:}
\begin{enumerate}
  \item Unless specifically told to use L'H\^opital's Rule, you may use any valid method to evaluate limits.
  \item Remember to 
    \begin{enumerate}
      \item keep your limit notation until the direct substitution step
      \item connect each step with equal signs
      \item notate the equal signs where L'H\^opital is used
    \end{enumerate}
  \item L'H\^opital does NOT replace the quotient rule!
\end{enumerate}

\pagebreak

\begin{ex*}
  Find the following limits with and without L'H\^opital's Rule:
\end{ex*}
\begin{tasks}[after-item-skip=\stretch{1}, label=~](2)
  \task 
    $\ds\lim_{x\to 2} \frac{x^2-4}{x-2}$
  \task 
    $\ds\lim_{x\to \infty} \frac{2x^2+3x}{x^3+x+1}$
\end{tasks}
\vspace*{\stretch{1}}

\hspace*{\stretch{1}}
\fbox{\parbox{0.675\linewidth}{\centering
\textit{Note:} L'H\^opital's Rule only works for indeterminate forms!}}
\hspace*{\stretch{1}}

\begin{ex*}
  Find the following limit with and without L'H\^opital's Rule:
\end{ex*}
\begin{tasks}[after-item-skip=\stretch{1}, label=~](1)
  \task $\ds\lim_{x\to \frac{\pi}{2}} \frac{\sin(x)}{1-\cos(x)}$
\end{tasks}
\vspace*{\stretch{1}}
\pagebreak

\begin{ex*}
  Find the following limits:
\end{ex*}
\begin{tasks}[after-item-skip=\stretch{1}, label=~](2)
  \task 
    $\ds\lim_{t\to 1} \frac{t^3-1}{4t^3-t-3}$
  \task 
    $\ds\lim_{z\to 0} \frac{\tan(4z)}{\tan(7z)}$
\end{tasks}
\vspace*{\stretch{1}}
\begin{ex*}
  Find the following limits. Repeat L'H\^opital's Rule each time you get an indeterminate form:
\end{ex*}
\begin{tasks}[after-item-skip=\stretch{1}, label=~](1)
  \task 
    $\ds\lim_{x\to 0} \frac{\sin(x)-x}{x^3}$
  \task 
    $\ds\lim_{t\to 0} \frac{t\sin(t)}{1-\cos(t)}$
\end{tasks}
\vspace*{\stretch{1}}
\pagebreak

\begin{ex*}
  Evaluate:
\end{ex*}
\begin{tasks}[after-item-skip=\stretch{1}, label=~](2)
  \task $\ds\lim_{x\to 0^+} \frac{\ln(x)}{x}$
  \task $\ds\lim_{x\to 3} \frac{2x^2-5x+1}{x^2+x-6}$
  \task $\ds\lim_{x\to \frac{1}{2}} \frac{6x^2+5x-4}{4x^2+16x-9}$
  \task $\ds\lim_{x\to \infty} \frac{x-8x^2}{12x^2+5x}$
  \task $\ds\lim_{t\to 0} \frac{e^{2t}-1}{\sin(t)}$
  \task $\ds\lim_{t\to 0} \frac{8^t-5^t}{t}$ 
\end{tasks}
\vspace*{\stretch{1}}
\pagebreak
\noindent

\hspace*{\stretch{1}}
\fbox{\parbox{0.675\linewidth}{\centering
\textit{Note:} $0\cdot \infty$ and $\infty-\infty$ are also indeterminate forms.

L'H\^opital's Rule can be used after these functions are converted into rational functions of indeterminate form.
}}
\hspace*{\stretch{1}}

\begin{ex*}
  Find the following limits. Convert into indeterminate form as needed:
\end{ex*}
\begin{tasks}[after-item-skip=\stretch{1}, label=~](1)
  \task $\ds\lim_{x\to 1\inv[]}(1-x)\tan\parens{\frac{\pi x}{2}}$
  \task $\ds\lim_{x\to \infty} x^2\sin\parens{\frac{1}{4x^2}}$
  \task $\ds\lim_{x\to 0^+} \parens{\csc(x)-\cot(x)+\cos(x)}$
\end{tasks}
\vspace*{\stretch{1}}
\pagebreak

\fbox{\parbox{0.9875\linewidth}{
  \textbf{Indeterminate forms $1^\infty$, $0^0$, and $\infty^0$.}
  
  Assume $\ds\lim_{x\to a} f(x)^{g(x)}$ has the indeterminate form $1^\infty$, $0^0$, or $\infty^0$.
  \begin{enumerate}
    \item Analyze $L=\ds\lim_{x\to a} g(x)\ln\parens{f(x)}$. This limit can be put in the form $0/0$ or $\infty/\infty$, both of which are handled by L'H\^opital's Rule.
    \item When $L$ is finite, $\ds\lim_{x\to a} f(x)^{g(x)}= e^L$. If $L=\infty$ or $L=-\infty$, then \newline$\ds\lim_{x\to a} f(x)^{g(x)}=\infty$ or $\ds\lim_{x\to a} f(x)^{g(x)}=0$, respectively.
  \end{enumerate}
  
  \textit{Note:} $0^\infty$ and $\infty^\infty$ are NOT indeterminate forms.
}}

\begin{tasks}[after-item-skip=\stretch{1}, label=~](2)
  \task $\ds\lim_{x\to 0^+} x^{-1/\ln(x)}$
  \task $\ds\lim_{x\to \infty} (1+2x)^{1/(2\ln(x))}$
\end{tasks}
\vspace*{\stretch{1}}
\pagebreak

\noindent
When working with the exponential indeterminate forms, the transformations are typically very similar:
\begin{center}
  \renewcommand{\arraystretch}{1.5}
  {\begin{tabular}{@{}lL|*{3}{R}@{}}
    \multicolumn{2}{@{}c}{}&\multicolumn{3}{c@{}}{Indeterminate form}\\
    \multirow{3}{*}{$\ds\lim_{x\to a}$}& f^g& 1^\infty& 0^0& \infty^0\\\cline{2-5}
    & e^{g\ln(f)}& e^{\infty\cdot 0}& e^{0\cdot\parens{-\infty}}& e^{0\cdot \infty}\\
    & e^{\frac{\ln(f)}{\sfrac{1}{g}}}& e^{\frac{0}{0}}& e^{\frac{-\infty}{\infty}}& e^{\frac{\infty}{\infty}}
  \end{tabular}}
  
  \vspace*{5pt}
  \textit{Note:} L'H\^ opital's rule is performed on the exponent only!
\end{center}
\begin{tasks}[after-item-skip=\stretch{1}, label=~](2)
  \task $\ds\lim_{x\to 0^+} x^{x^2}$
  \task $\ds\lim_{x\to 0^+} x^{\sqrt x}$
\end{tasks}
\vspace*{\stretch{1}}
\pagebreak


\hspace*{\stretch{1}}
\fbox{\parbox{0.675\linewidth}{\centering
  \textit{Note:} L'H\^opital does not always work!
}}
\hspace*{\stretch{1}}

\begin{tasks}[after-item-skip=\stretch{1}, label=~](3)
  \task $\ds\lim_{x\to 0^+} \frac{\sqrt x}{\sqrt{\sin(x)}}$
  \task $\ds\lim_{x\to 0^+} \frac{\cot(x)}{\csc(x)}$
  \task $\ds\lim_{x\to \infty} \frac{e^{3x}-e\inv[3x]}{e^{3x}+e\inv[3x]}$
\end{tasks}
\vspace*{\stretch{1}}

\begin{ex*}
  Find the following limits:
\end{ex*}
\begin{tasks}[after-item-skip=\stretch{1}, label=~](1)
  \task $\ds\lim_{x\to 2\pi} \frac{x\sin(x)+x^2-4\pi^2}{x-2\pi}$
  \task $\ds\lim_{x\to \frac{\pi}{2}} \frac{2\tan(x)}{\sec^2(x)}$
  \task $\ds\lim_{x\to -1} \frac{x^3-x^2-5x-3}{x^4+2x^3-x^2-4x-2}$
\end{tasks}
\vspace*{\stretch{1}}
\pagebreak

\begin{tasks}[after-item-skip=\stretch{1}, label=~](2)
  \task $\ds\lim_{x\to \infty} \frac{27x^2+3x}{3x^2+x+1}$
  \task $\ds\lim_{x\to 0} \frac{x+\sin(x)}{x+\cos(x)}$
  \task $\ds\lim_{t\to 0} \frac{2t}{\tan(t)}$
  \task $\ds\lim_{x\to 0} \frac{\sqrt{1+2x}-\sqrt{1-4x}}{x}$
\end{tasks}
\vspace*{\stretch{1}}
\pagebreak

\begin{tasks}[after-item-skip=\stretch{1}, label=~](2)
  \task $\ds\lim_{x\to 0} \frac{x^2-2x}{x^2-\sin(x)}$
  \task $\ds\lim_{\theta\to \frac{\pi}{2}} \frac{1-\sin(\theta)}{\csc(\theta)}$
  \task $\ds\lim_{x\to 2} \frac{\sqrt[3]{3x+2}-2}{x-2}$
  \task $\ds\lim_{x\to \infty} \frac{100x^3-3}{x^4-2}$
\end{tasks}
\vspace*{\stretch{1}}
\pagebreak

\begin{tasks}[after-item-skip=\stretch{1}, label=~](2)
  \task $\ds\lim_{x\to 0} \frac{\sin^2(3x)}{x^2}$
  \task $\ds\lim_{x\to \infty} x^3\parens{\frac{4}{x}-\sin\parens{\frac{4}{x}}}$
  \task $\ds\lim_{x\to 0} \cot(2x)\sin(6x)$
  \task $\ds\lim_{x\to 0^+} \parens{\cot(x)-\frac{1}{x}}$
\end{tasks}
\vspace*{\stretch{1}}
\pagebreak

\begin{tasks}[after-item-skip=\stretch{1}, label=~](2)
  \task $\ds\lim_{\theta\to 0} \parens{\frac{1}{1-\cos(\theta)}-\frac{2}{\sin^2(\theta)}}$
  \task $\ds\lim_{x\to 0} \parens{1-2x}^\frac{1}{x}$
  \task $\ds\lim_{\theta\to 0^+} \parens{\sin(\theta)}^{\tan(\theta)}$
  \task $\ds\lim_{x\to 0^+} \parens{\tan x}^x$
\end{tasks}
\vspace*{\stretch{1}}
\pagebreak

\begin{tasks}[after-item-skip=\stretch{1}, label=~](1)
  \task $\ds\lim_{x\to\infty} \parens{1+\frac{a}{x}}^x$
  \task $\ds\lim_{x\to 0} \parens{e^{ax}+x}^\frac{1}{x}$
  \task $\ds\lim_{x\to 0^+} \frac{x^x-1}{\ln(x)+x-1}$
\end{tasks}
\vspace*{\stretch{1}}
\pagebreak

\begin{defn*}[Growth Rates of Functions (as $x\to\infty$)]
  Suppose $f$ and $g$ are functions with $\ds\lim_{x\to \infty} f(x)=\lim_{x\to \infty} g(x)=\infty$. Then $f$ \textit{grows faster than} $g$ as $x\to\infty$ if
    \[\lim_{x\to \infty} \frac{g(x)}{f(x)}=0\quad\textnormal{ or, equivalently, if}\quad \lim_{x\to \infty} \frac{f(x)}{g(x)}=\infty.\]
  The functions $f$ and $g$ have \textit{comparable growth rates} if
    \[\lim_{x\to \infty} \frac{f(x)}{g(x)}=M,\]
  where $0<M<\infty$ ($M$ is positive and finite)
\end{defn*}

\vspace*{5pt}

\noindent
\fbox{\parbox{0.9875\linewidth}{
  \textbf{Theorem 4.14: Ranking Growth Rates as $x\to\infty$}
  
  Let $f\ll g$ mean that $g$ grows faster than $f$ as $x\to\infty$. With positive real numbers $p,q,r,$ and $s$ and with $b>1$,
    \[\parens{\ln(x)}^q\ll x^p\ll x^p\parens{\ln(x)}^r\ll x^{p+s}\ll b^x\ll x^x\]
}}
\begin{ex*}
  Rank the functions in order of increasing growth rates as $x\to\infty$:
\end{ex*}
\begin{tasks}[after-item-skip=\stretch{1}, label=~](2)
  \task $x^3,\,\ln(x)\,,x^x,$ and $2^x$
  \task $x^{100},\,\ln\parens{x^{10}},\,x^x,$ and $10^x$
\end{tasks}
\vspace*{\stretch{1}}

\begin{ex*}
  Use limits to compare and rank growth rates of the following functions:
\end{ex*}
\begin{tasks}[after-item-skip=\stretch{1}, label=~](2)
  \task $\ln\parens{x^{20}};\ \ln(x)$
  \task $\ln(x);\ \ln\parens{\ln(x)}$
  \task $100^x;\ x^x$
  \task $e^{x^2};\ x^{x/10}$
\end{tasks}
\vspace*{\stretch{1}}
\pagebreak
\end{document}
