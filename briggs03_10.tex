\documentclass[answers]{exam}
\usepackage{texPreamble}
\usepackage{relsize}
\usepackage{tkz-euclide}
\usetkzobj{all}
\usepackage{tabularx}
\extraheadheight{0.25in}
\extrafootheight{1.0in}
\extrawidth{1in}
% ----------------------------------------------------------------
\firstpagefootrule
\runningfootrule
\begin{document}
\tikzset{custNode/.style={color=black,inner sep=0.5pt,fill=none,opacity=1.0, text opacity=1}}
\tikzset{dLine/.style={mark=none, dashed, opacity=1.0, blue!85, line width=0.5pt}}
%%% Need following lines in preamble
%%%%% \usepackage{tkz-euclide} %%%%%
%%%%% \usetkzobj{all}          %%%%%
\newcommand{\blankTriangle}{
      \coordinate (O) at (1.5,6.75);
      \coordinate (A) at (5.5,6.75);
      \coordinate (B) at (5.5,8.75);
      \draw (O)--(A)--(B)--cycle;
      
      \tkzMarkAngle[fill= ClemsonOrange!90,size=1.25cm, opacity=0.8](A,O,B)
      \tkzLabelAngle[pos = 1.0](A,O,B){$y$}
      }
%\relscale{1.4}
\section{3.10: Derivatives of Inverse Trigonometric Functions}
\begin{ex*}
  Recall that $y=\sin\inv(x) \iff \sin(y)=x$. Use this fact and implicit differentiation to derive the derivative of $\sin\inv(x)$.
  \begin{flushright}
    \begin{tikzpicture}[scale=1.0]
      \blankTriangle
      \begin{axis}[
        axis lines=center,
        axis line style={->},
        xmin=-(pi/2+0.95), xmax=pi/2+0.95,
        ymin=-(pi/2+1.5), ymax=pi/2+1.5,
        xtick={-1.570796327,-1,1,1.570796327},
        ytick={-1.570796327,-1,1,1.570796327},
        xticklabels={$-\frac{\pi}{2}$,-1,1,$\frac{\pi}{2}$},
        yticklabels={$-\frac{\pi}{2}$,-1,1,$\frac{\pi}{2}$},
        ticklabel style={font=\small, inner sep=0.75pt,fill=white},
        xlabel=$x$, xlabel style={at={(ticklabel* cs:0.93)},anchor=north west},
        ylabel=$y$, ylabel style={at={(ticklabel* cs:0.9)},anchor=south west},
        every axis plot/.append style={line width=0.95pt, color=blue, samples=100},
        ]
        \addplot[-] expression[domain=-1:1, ClemsonOrange]{asin(x)*pi/180};
        \addplot[-] expression[domain=-0.99:0.99, ClemsonPurple] {1/sqrt(1-x^2)};
        \addplot[dLine] coordinates {(-1,-5) (-1,5)};
        \addplot[dLine] coordinates {(1,-5) (1,5)};
      \end{axis}
    \end{tikzpicture}
  \end{flushright}
  
  \noindent
  Next, extend this definition for the derivative of $\sin\inv(f(x))$.
\end{ex*}
\vspace*{60pt}

\begin{ex*}
  Find the derivative of the following
\end{ex*}
\begin{tasks}[after-item-skip=\stretch{1}, label=~](2)
  \task $\ds f(x)=\sqrt{1-x^2}\,\arcsin(x)$
  \task $\ds y=\sin\inv\parens{\sqrt2 t}$
\end{tasks}
\vspace*{70pt}

\pagebreak
\begin{ex*}
  Recall that $y=\tan\inv(x) \iff \tan(y)=x$. Use this fact and implicit differentiation to derive the derivative of $\tan\inv(x)$.
  \begin{flushright}
    \begin{tikzpicture}[scale=1.0]
      \blankTriangle
      \begin{axis}[
        axis lines=center,
        axis line style={->},
        xmin=-3.5, xmax=3.5,
        ymin=-(pi/2+0.75), ymax=pi/2+0.75,
        xtick={-1.570796327,-1,1,1.570796327},
        ytick={-1.570796327,-1,1,1.570796327},
        xticklabels={$-\frac{\pi}{2}$,-1,1,$\frac{\pi}{2}$},
        yticklabels={$-\frac{\pi}{2}$,-1,1,$\frac{\pi}{2}$},
        ticklabel style={font=\small, inner sep=0.75pt,fill=white},
        xlabel=$x$, xlabel style={at={(ticklabel* cs:0.93)},anchor=north west},
        ylabel=$y$, ylabel style={at={(ticklabel* cs:0.9)},anchor=south west},
        every axis plot/.append style={line width=0.95pt, color=blue, samples=100},
        ]
        \addplot[-] expression[domain=-5:5, ClemsonOrange]{atan(x)*pi/180};
        \addplot[-] expression[domain=-5:5, ClemsonPurple] {1/(x^2+1)};
        \addplot[dLine] coordinates {(-5,pi/2) (5,pi/2)};
        \addplot[dLine] coordinates {(-5,-pi/2) (5,-pi/2)};
      \end{axis}
    \end{tikzpicture}
  \end{flushright}
  \noindent
  Next, extend this definition for the derivative of $\tan\inv(f(x))$.
\end{ex*}
\vspace*{60pt}

\begin{ex*}
  Find the derivative of the following
\end{ex*}
\begin{tasks}[after-item-skip=\stretch{1}, label=~](2)
  \task $\ds y=\sqrt{\tan\inv(x)}$
  \task $\ds y=\tan\inv\parens{\sqrt x}$
\end{tasks}
\vspace*{70pt}  
\pagebreak

\noindent
\fbox{\parbox{0.9875\linewidth}{
\textbf{Derivatives of Inverse Trigonometric Functions}
\begin{center}
  \begin{tabular}{R@{\ =\ }L@{\hspace*{40pt}}R@{\ =\ }L}
    \ds\ddx\sbrkt{\sin\inv(x)}& \dfrac{1}{\sqrt{1-x^2}}& \ds\ddx\sbrkt{\cos\inv(x)}& -\dfrac{1}{\sqrt{1-x^2}}\\[30pt]
    \ds\ddx\sbrkt{\tan\inv(x)}& \dfrac{1}{1+x^2}& \ds\ddx\sbrkt{\cot\inv(x)}& -\dfrac{1}{1+x^2}\\[30pt]
    \ds\ddx\sbrkt{\sec\inv(x)}& \dfrac{1}{\abs{x}\sqrt{x^2-1}}& \ds\ddx\sbrkt{\csc\inv(x)}&-\dfrac{1}{\abs{x}\sqrt{x^2-1}}
  \end{tabular}
\end{center}}}
\vspace*{\stretch{1}}

\noindent
\textbf{Derivative of $y=\sec\inv(x)$}

\begin{center}
  \begin{minipage}{0.3\linewidth}
    \begin{align*}
      y&=\sec\inv(x)\\[7.5pt]
      \sec(y)&=x\\[7.5pt]
      \sec(y)\tan(y)\dydx&=1\\[7.5pt]
      \dydx&=\frac{1}{\sec(y)\tan(y)}
    \end{align*}
  \end{minipage}%
  \hspace*{0.15\linewidth}
  \begin{minipage}{0.3\linewidth}
    \begin{tikzpicture}
      \blankTriangle
      \draw ($(O)!0.5!(B)$) node[above] {$x$};
      \draw ($(A)!0.5!(B)$) node[right] {$\sqrt{x^2-1}$};
      \draw ($(O)!0.5!(A)$) node[below] {$1$};
    \end{tikzpicture}
    
    \vspace*{-15pt}
    \begin{align*}
      \sin^2(x)+\cos^2(x)=1\\[7.5pt]
      \tan^2(x)+1=\sec^2(x)\\[7.5pt]
      1+\cot^2(x)=\csc^2(x)
    \end{align*}
  \end{minipage}%
\end{center}
\vspace*{\stretch{1}}
%Now we rewrite $\sec(y)\tan(y)$ in terms of $x$. 
Note that the restricted domain of $\sec(y)$ is $[0,\nicefrac{\pi}{2})\cup(\nicefrac{\pi}{2},\pi]$, and the restricted domain of $\tan(y)$ is $\parens{-\frac{\pi}{2},\frac{\pi}{2}}$. If we look at these quadrants of the unit circle, we see that the product $\sec(y)\tan(y)$ is always positive, so the resulting derivative must always be positive:
\vspace*{\stretch{1}}
  \[\ddx\sbrkt{\sec\inv(x)}=\frac{1}{\abs{x}\sqrt{x^2-1}}\]
\pagebreak

\begin{ex*}
  Find the derivatives of the following functions:
\end{ex*}
\begin{tasks}[after-item-skip=\stretch{1}, label=~](2)
  \task $h(t)=e^{\sec\inv(t)}$
  \task $y=\arccos\parens{e^{2x}}$
  \task $y=\sin\inv(2x+1)$
  \task $y=\sec\inv(5r)$
  \task $f(x)=\csc\inv\parens{\tan(e^x)}$
  \task $f(x)=\tan\inv(10x)$
\end{tasks}
\vspace*{\stretch{1}}
\pagebreak

\begin{tasks}[after-item-skip=\stretch{1}, label=~](2)
  \task $y=x\sin\inv(x)+\sqrt{1-x^2}$
  \task $h(t)=\cot\inv(t)+\cot\inv\parens{\frac{1}{t}}$
  \task $f(x)=2x\tan\inv(x)-\ln\parens{1+x^2}$
  \task $f(t)=\ln\parens{\tan\inv(t)}$
\end{tasks}
\vspace*{\stretch{1}}
\begin{ex*}
  Find the equation of the tangent line to $f(x)=\cos\inv(x^2)$ at $\parens{\frac{1}{\sqrt2},\frac{\pi}{3}}$.
\end{ex*}
\vspace*{\stretch{1}}
\begin{ex*}
  Find the equation of the tangent line to $f(x)=\sec\inv(e^x)$ at $\parens{\ln(2),\frac{\pi}{3}}$.
\end{ex*}
\vspace*{\stretch{1}}
\pagebreak

\end{document}
