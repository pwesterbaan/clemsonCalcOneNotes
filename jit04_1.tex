\documentclass[answers]{exam}
\usepackage{texPreamble}
\usepackage{relsize}
\usepackage{tabularx}
\extraheadheight{0.25in}
\extrafootheight{1.0in}
\extrawidth{1in}
% ----------------------------------------------------------------

\begin{document}
\section{JIT 4.1: Functions And Their Graphs}
  \begin{itemize}
    \item 
      A way to relate two quantities to each other.
      %\\[\stretch{1}]
    %\item 
      %\\[\stretch{1}]\pagebreak
  \end{itemize}
  \begin{minipage}{0.5\linewidth}
    \subsection*{Graphically:}
    \begin{center}
      \begin{tikzpicture}[scale=1.0]
        \begin{axis}[
          axis lines=center,
          axis line style={->},
          xmin=-1.75, xmax=5,
          ymin=-0.25,  ymax=1.75,
          xtick={-3,-2,...,4},
          %ytick={-1,-0.5,0.5,1,...,2.5},
          xlabel=$x$, xlabel style={at={(1,0.15)}, anchor=west},
          ylabel=$f(x)$, ylabel style={at={(0.25,1)}, anchor=south},
          ]
        \def\f(#1){0.5*(exp(-cos((#1 r)*pi)-exp(sin((#1 r)))-exp(-1)+1)}
        \addplot[samples=501,blue] {\f(\x)};
        \end{axis}
      \end{tikzpicture}
    \end{center}
  \end{minipage}%
  \begin{minipage}{0.5\linewidth}
    \subsection*{Tabularly:}
    \begin{center}
      \begin{tabular}{@{}ll@{}}
        $x$& $f(x)$\\\midrule
        -1& $\sfrac32$\\
         0& 0\\
         1& $\sfrac12$\\
         2& 1\\
         3& $\sfrac14$
      \end{tabular}
    \end{center}
    \vspace*{65pt}
  \end{minipage}
  \begin{defn*}
    A \textbf{function} $f$ defined from a set $A$ to a set $B$ is a rule that associates with each element of the set $A$ one, and only one, element of set $B$.
  \end{defn*}
  \vspace*{\stretch{1}}
  \begin{defn*}
    The \textbf{domain} of a function is the set of all input values.
  \end{defn*}
  \vspace*{\stretch{1}}
  \begin{defn*}
    The \textbf{range} of a function is the set of all output values.
  \end{defn*}
  \vspace*{\stretch{1}}
  \pagebreak
  \begin{ex*}
    For $f(x)=2x^2$, find 

    \begin{minipage}{0.5\linewidth}
      \begin{enumerate}[label={}]
        \item $f(4)$
        \item $f(-3)$
        \item $f(4+h)$
        \item $f(x+\Delta x)$
        \item $f\parens{\sqrt{\dfrac{x}{2}}}$
      \end{enumerate}
    \end{minipage}%
    \begin{minipage}{0.5\linewidth}
      \begin{enumerate}[label={}]
        \item Domain of $f(x)$:
        \item Range of $f(x)$:
        \item 
        \item 
        \item 
      \end{enumerate}
    \end{minipage}
  \end{ex*}\ 
  \\[\stretch{1}]
  \begin{ex*}
    For $g(x)=\sqrt x+1$, find 
    \begin{enumerate}[label={}]
      \item Domain of $g(x)$:\\[\stretch{1}]
      \item Range of $g(x)$:\\[\stretch{1}]
    \end{enumerate}
  \end{ex*}
  
  \begin{ex*}
    For $h(x)=\sqrt{3-x}-2$, find 
    \begin{enumerate}[label={}]
      \item Domain of $h(x)$:\\[\stretch{1}]
      \item Range of $h(x)$:\\[\stretch{1}]
    \end{enumerate}
  \end{ex*}
  \pagebreak
  \begin{ex*}
    For $j(x)=\sqrt[3]{3-x}-2$, find 
    \begin{enumerate}[label={}]
      \item Domain of $j(x)$:\\[\stretch{1}]
      \item Range of $j(x)$:\\[\stretch{1}]
    \end{enumerate}
  \end{ex*}
  \begin{ex*}
    For $\kappa(\nu)=\dfrac{\nu^2-1}{\nu-1}$, find 
    \begin{enumerate}[label={}]
      \item Domain of $\kappa(\nu)$:\\[\stretch{1}]
      \item Range of $\kappa(\nu)$:\\[\stretch{1}]
    \end{enumerate}
  \end{ex*}
  \begin{ex*}
    For $\ell(t)=40t-5t^2$, find 
    \begin{enumerate}[label={}]
      \item Domain of $\ell(t)$:\\[\stretch{1}]
      \item Range of $\ell(t)$:\\[\stretch{1}]
    \end{enumerate}
  \end{ex*}
  \pagebreak
  \begin{ex*}
    For $m(\omega)=40\omega-5\omega^2$, find 
    \begin{enumerate}[label={}]
      \item Domain of $m(\omega)$:\\[\stretch{0.5}]
      \item Range of $m(\omega)$:\\[\stretch{0.5}]
    \end{enumerate}
    \begin{ex*}
    A cylindrical water tower with a radius of 10m and a height of 50m is filled to a height of $h$. The volume $V$ of water (in cubic meters) is given by the function $g(h)=100\pi h$. Identify the independent and dependent variables of for this function, and then determine an appropriate domain.\\[\stretch{1}]
  \end{ex*}
  \end{ex*}
  \pagebreak
\end{document}