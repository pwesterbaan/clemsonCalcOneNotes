\documentclass[answers]{exam}
\usepackage{texPreamble}
\usepackage{relsize}
\usepackage{tabularx}
\extraheadheight{0.25in}
\extrafootheight{1.0in}
\extrawidth{1in}
% ----------------------------------------------------------------

\begin{document}
    \section{JIT 15.1: Trigonometric Identities}
      \begin{defn*}
        The \textbf{Pythagorean Identity for trigonometric functions} is
          $$\sin^2(\theta)+\cos^2(\theta)=1$$
      \end{defn*}
      \pagebreak
      \begin{defn*}
        The \textbf{Angle Sum Formulas} are
        \begin{align*}
          \sin(A\pm B)&=\sin(A)\cos(B)\pm\cos(A)\sin(B)\\
          \cos(A\pm B)&=\cos(A)\cos(B)\mp\sin(A)\sin(B)
        \end{align*}
        \textbf{Note:} Since $\cos(\theta)$ is even and $\sin(\theta)$ is odd, we can derive the difference formula from the sum formula.
      \end{defn*}
      \pagebreak
      \begin{defn*}
        The \textbf{double-angle formulas} are a special case of the angle-sum formulas:
        
        \begin{minipage}{0.5\linewidth}
          \begin{align*}
            \sin(2\theta)&=\sin(\theta+\theta)\\
              &=\sin(\theta)\cos(\theta)+\cos(\theta)\sin(\theta)\\
              &=\boxed{2\sin(\theta)\cos(\theta)}
          \end{align*}
        \end{minipage}%
        \begin{minipage}{0.5\linewidth}
          \begin{align*}
            \cos(2\theta)&=\cos(\theta+\theta)\\
              &=\cos(\theta)\cos(\theta)-\sin(\theta)\sin(\theta)\\
              &=\boxed{\cos^2(\theta)-\sin^2(\theta)}
          \end{align*}
        \end{minipage}\\[15pt]
        
        \textbf{Note:} Using the Pythagorean Identity, we have 2 additional representations of $\cos(2\theta)$.
      \end{defn*}
      \pagebreak
      \begin{defn*}
        The \textbf{half-angle formulas} are derived from the double angle formula:
        \begin{align*}
          \sin(\theta)&=\pm\sqrt{\frac{1-\cos(2\theta)}{2}}\\
          \cos(\theta)&=\pm\sqrt{\frac{1+\cos(2\theta)}{2}}
        \end{align*}
      \end{defn*}
      \pagebreak
      \begin{ex*}
        Solve all the following on $\sbrkt{0,2\pi}$.
        \begin{tasks}(2)
          \task $2\theta\cos(\theta)+\theta=0$
          \task $\sin(\theta)=\frac{1}{2}$\\[2in]
          \task $4\cos^2(x)-3=0$
          \task $2\sin^2(x)-\sin(x)-1=0$\\[2in]
          \task $\sin(3x)=\dfrac{\sqrt2}{2}$
          \task $\cos(3x)=\sin(3x)$
        \end{tasks}
      \end{ex*}
      \pagebreak
\end{document}