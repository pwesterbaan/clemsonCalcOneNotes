\documentclass[mathNotesPreamble]{subfiles}
\begin{document}
\relscale{1.4} %TODO
\section{14.4: Length of Curves}
  \begin{defn*}[Arc Length for Vector Functions]
    Consider the parameterized curve $\vecr(t)=\bracket{f(t),g(t),h(t)}$, where $f'$, $g'$, and $h'$ are continuous, and the curve is traversed once for $a\leq t\leq b$. The \textbf{arc length} of the curve between $\parens{f(a), g(a), h(a)}$ and $\parens{f(b), g(b), h(b)}$ is
    \[L=\int_a^b \sqrt{f'(t)^2+g'(t)^2+h'(t)^2}\,dt=\int_a^b\abs{\vecr'(t)}\,dt.\]
  \end{defn*}
  
  \noindent
  \fbox{\parbox{0.9875\linewidth}{
    \textbf{Theorem 14.3: Arc Length as a Function of a Parameter}\\
    Let $\vecr(t)$ describe a smooth curve, for $t\geq a$. The arc length is given by 
      \[s(t)=\int_a^t \abs{v(u)}\,du,\]
    where $\abs{\vecv}=\abs{\vecr'}$. Equivalently, $\displaystyle \frac{ds}{dt}=\abs{\vecv(t)}$. If $\abs{\vecv(t)}=1$, for all $t\geq a$, then the parameter $t$ corresponds to arc length.
  }}
  \pagebreak
  
\end{document}