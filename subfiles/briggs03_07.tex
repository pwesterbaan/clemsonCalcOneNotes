\documentclass[../mathNotesPreamble]{subfiles}
\begin{document}
%\relscale{1.4}
\section{3.7: The Chain Rule}

\begin{thmBox*}[Theorem 3.13: The Chain Rule]
  Suppose $y=f(u)$ is differentiable at $u=g(x)$ and $u=g(x)$ is differentiable at $x$. The composite function $y=f(g(x))$ is differentiable at $x$, and its derivative can be expressed in two equivalent ways.
    \begin{align}
      &\dfrac{dy}{dx} = \dfrac{dy}{du}\cdot \dfrac{du}{dx}\\[10pt]
      &\dfrac{d}{dx}\parens{f\parens{g(x)}}=f'\parens{g(x)}\cdot g'(x)
    \end{align}
\end{thmBox*}

\begin{ex*}
  Take the derivatives of the following functions
\end{ex*}
\begin{tasks}[after-item-skip=\stretch{1}](2)
  \task $y=\parens{3x^3+1}^2$
  \task $y=\parens{3x^3+1}^7$
  \task $y=6\cos^2(x)$
  \task $y=\sin\parens{x+\cot(x)}$
\end{tasks}
\vspace*{\stretch{1}}
\pagebreak

\begin{thmBox*}
  To use the chain rule,
    \begin{itemize}
      \item Identify the inner and outer function
      \item Take the derivative of the outside, leaving the original inner function
      \item Multiply by the derivative of the inner function
    \end{itemize}
\end{thmBox*}

\begin{tasks}[resume, after-item-skip=\stretch{1}](2)
  \task $y(x)=e\inv[4x]$
  \task $y(x)=\parens{\dfrac{x-2}{2x+1}}^9$
  \task $y(x)=\sqrt{\sec(x)}$
  \task $y(x)=2\parens{8x-1}^3$
\end{tasks}
\vspace*{\stretch{1}}
\pagebreak

\begin{tasks}[resume, after-item-skip=\stretch{1}](2)
  \task $y(x)=\parens{\frac{x}{2}-1}\inv[10]$
  \task $y(t)=e^{\sin(t)}+\sin(e^t)$
  \task $y(x)=x^2 e^{x^2}$
  \task $\dfrac{f(x)}{g(x)}=f(x)\cdot\sbrkt{g(x)}\inv$
\end{tasks}
\vspace*{\stretch{1}}
\pagebreak

\begin{tasks}[resume, after-item-skip=\stretch{1}](2)
  \task $y(x)=f\parens{g\parens{h(x)}}$
  \task $y(x)=-12e^{3x^7}$
  \task* $y(x)=\dfrac{\cos^2(x)}{e^x\parens{x^2+4}}$
\end{tasks}
\vspace*{\stretch{1}}
\pagebreak

\end{document}
