\documentclass[../mathNotesPreamble]{subfiles}
\begin{document}
%  \relscale{1.4}
  \section{8.9: Improper Integrals}
    \begin{defn*}[Improper Integrals over Infinite Intervals]
      \begin{enumerate}
        \item \label{infInterval_case1} If $f$ is continuous on $[a,\infty)$, then
          \[\int_a^\infty f(x)\,dx=\lim_{b\to \infty} \int_a^b f(x)\,dx.\]
        \item \label{infInterval_case2} If $f$ is continuous on $(-\infty,b]$, then
          \[\int_{-\infty}^b f(x)\,dx=\lim_{a\to -\infty} \int_a^b f(x)\,dx.\]
        \item \label{infInterval_case3} If $f$ is continuous on $(-\infty,\infty)$, then
          \[\int_{-\infty}^\infty f(x)\,dx=\lim_{a\to-\infty}\int_a^c f(x)\,dx+\lim_{b\to\infty}\int_c^b f(x)\,dx.\]
          where $c$ is any real number. It can be shown that the choice of $c$ does not affect the value or convergence of the original integral.
      \end{enumerate}
      If the limits in cases \ref{infInterval_case1}-- \ref{infInterval_case3} exist, then the improper integrals \textbf{converge}; otherwise they \textbf{diverge}.
    \end{defn*}

    \begin{ex*}
      Evaluate $\displaystyle \int_1^\infty \frac{\ln(x)}{x}\,dx$ and determine if the integral converges or diverges.
    \end{ex*}
    \vspace*{\stretch{1}}
    \pagebreak

    \begin{ex*}
      Evaluate $\displaystyle \int_{-\infty}^{\infty} \frac{e^{3x}}{1+e^{6x}}\,dx$.
    \end{ex*}
    \vspace*{\stretch{1}}
    \pagebreak

    \begin{ex*}
      For what values of $p$ does $\displaystyle \int_1^\infty \frac{1}{x^p}\,dx$ converge?
    \end{ex*}
    \vspace*{\stretch{1}}
    \pagebreak

    \begin{defn*}[Improper Integrals with Unbounded Integrand]
      \begin{enumerate}
        \item \label{unboundedIntegrand_case1} Suppose $f$ is continuous on $(a,b]$ with $\displaystyle \lim_{x\to a^+} f(x)=\pm\infty$. Then
          \[\int_a^b f(x)\,dx=\lim_{c\to a^+} \int_c^b f(x)\,dx.\]
        \item \label{unboundedIntegrand_case2} Suppose $f$ is continuous on $[a,b)$ with $\displaystyle \lim_{x\to b^-} f(x)=\pm\infty$. Then
          \[\int_a^b f(x)\,dx=\lim_{c\to b^-} \int_a^c f(x)\,dx.\]
        \item \label{unboundedIntegrand_case3} Suppose $f$ is continuous on $[a,b]$ except at the interior point $p$ where $f$ is unbounded. Then
          \[\int_a^b f(x)\,dx=\lim_{c\to p^-} \int_a^c f(x)\,dx + \lim_{d\to p^+}\int_d^b f(x)\,dx.\]
      \end{enumerate}
      If the limits in cases \ref{unboundedIntegrand_case1}-- \ref{unboundedIntegrand_case3} exist, then the improper integrals \textbf{converge}; otherwise, they \textbf{diverge}.
    \end{defn*}
    \pagebreak

    \begin{ex*}
      Determine which of the following integrals are improper integrals
    \end{ex*}
    \begin{tasks}[after-item-skip=\stretch{1}, label=, item-indent=0pt](2)
      \task $\displaystyle \int_0^1 \sec(x)\,dx$
      \task $\displaystyle \int_{\pi/2}^{3\pi/4} \tan(x)\,dx$
      \task $\displaystyle \int_1^e \ln(x)\,dx$
      \task $\displaystyle \int_0^1 \arctan(x)\,dx$
      \task $\displaystyle \int_0^{0.5} \ln(x)\,dx$
      \task $\displaystyle \int_{-10}^{-1} \frac{1}{x^{1/3}}\,dx$
    \end{tasks}
    \vspace*{\stretch{1}}
    \pagebreak

    \begin{ex*}
      Evaluate $\displaystyle \int_1^9 \frac{dx}{(x-1)^{2/3}}.$ Does this integral converge or diverge?
    \end{ex*}
    \vspace*{\stretch{1}}
    \pagebreak

    \begin{ex*}
      Evaluate $\displaystyle \int_{-1}^{1} \frac{e^{2/x}}{x^2}\,dx$. Does this integral converge or diverge?
    \end{ex*}
    \vspace*{\stretch{1}}
    \pagebreak

    \begin{thmBox*}[Theorem 8.2: Comparison Test for Improper Integrals]
      Suppose the functions $f$ and $g$ are continuous on the interval $[a,\infty)$, with\newline $f(x)\geq g(x)\geq 0$, for $x\geq a$.
      \begin{enumerate}
        \item If $\displaystyle \int_a^\infty f(x)\,dx$ converges, then $\int_a^\infty g(x)\,dx$ converges.
        \item If $\displaystyle \int_a^\infty g(x)\,dx$ diverges, then $\int_a^\infty f(x)\,dx$ diverges.
      \end{enumerate}
    \end{thmBox*}

    \begin{ex*}
      Determine if the integral $\displaystyle \int_2^\infty \frac{x^3}{x^4-x^3-1}\,dx$ converges or diverges.
    \end{ex*}
    \vspace*{\stretch{1}}
    \pagebreak

  \begin{ex*}[Gabriel's Horn]
      Let $R$ be the region bounded by the graph of $y=1/x$ and the $x$-axis for $x\geq 1$. 
    \end{ex*}
    \begin{tasks}[after-item-skip=\stretch{1}, label=, item-indent=0pt](1)
      \task What is the volume of the solid generated when $R$ is revolved around the $x$-axis?
      \task What is the surface area of the solid generated when $R$ is revolved about the $x$-axis?
    \end{tasks}
    \vspace*{\stretch{1}}
    \pagebreak

\end{document}
