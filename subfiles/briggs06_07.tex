\documentclass[../mathNotesPreamble]{subfiles}
\begin{document}
  \relscale{1.4}
  \section{6.7: Physical Applications}

  \begin{defn*}[Mass of a One-Dimensional Object]
    Suppose a thin bar or wire is represented by the interval $a\leq x\leq b$ with a density function $\rho$ (with units of mass per length). The \textbf{mass} of the object is
      \[m=\int_a^b \rho(x)\,dx.\]
  \end{defn*}

  \begin{ex*}
    A thin bar, represented by the interval $0\leq x\leq 4$, has density in units of kg/m given by $\rho(x)=5e^{-2x}$. What is the mass of the bar?
  \end{ex*}
  \vspace*{\stretch{1}}
  \pagebreak

  \begin{defn*}[Work]
    The work done by a variable force $F$ moving an object along a line from $x=a$ to $x=b$ in the direction of the force is
      \[W=\int_a^b F(x)\,dx.\]
  \end{defn*}

  \begin{ex*}
    According to \textbf{Hooke's Law}, the force required to keep a spring in a compressed or stretched position $x$ units from the equilibrium position is $F(x)=kx$, where the positive spring constant $k$ measures the stiffness of the spring.
    \vspace*{\baselineskip}

    Suppose a force of $40 N$ is required to stretch a spring 0.1$m$ from its equilibrium position. Assuming the spring obeys Hooke's Law, how much work is required to stretch the spring 0.4$m$ beyond is equilibrium position? 
  \end{ex*}
  \vspace*{\stretch{1}}
  \pagebreak

  \begin{ex*}
    Imagine a chain of length $L$ meters with constant density $\rho$ kg/m is hanging vertically. Using $g$ to represent the force due to gravity, the work required to lift the chain is
      \[W=\int_0^L \rho g\parens{L-y}\,dy\]
    A $50$ meter long chain hangs vertically from a cylinder attached to a winch. Assume there is no friction in the system and the chain has a density of $3$\nobreakspace kg/m. How much work is required to wind the entire chain onto the cylinder if a $60$-kg load is attached to the end of the chain? Use $g$ for the acceleration due to gravity.
  \end{ex*}
  \vspace*{\stretch{1}}
  \pagebreak

  \begin{ex*}
    A $30$-meter long rope hangs freely from a ledge. The rope has a density of $5$\nobreakspace kg/m.  How much work is done if the top $1/3$ of the rope is pulled up to the ledge? Use $g$ for the acceleration due to gravity.
  \end{ex*}
  \vspace*{\stretch{1}}
  \pagebreak

  \begin{thmBox*}[Procedure: Solving Pumping Problems]
    \begin{enumerate}
      \item 
        Draw a $y$-axis in the vertical direction (parallel to gravity) and choose a convenient origin. Assume the interval $\sbrkt{a,b}$ corresponds to the vertical extent of the fluid.
      \item 
        For $a\leq y\leq b$, find the cross-sectional area $A(y)$ of the horizontal slices and the distance $D(y)$ the slices must be lifted.
      \item 
        The work required to lift the water is
          \[W=\int_a^b \rho gA(y)D(y)\,dy.\]
    \end{enumerate}
  \end{thmBox*}

  \begin{ex*}
    
  \end{ex*}
  \vspace*{\stretch{1}}
  \pagebreak

  \begin{thmBox*}[Procedure: Solving Force-on-Dam Problems]
    \begin{enumerate}
      \item 
        Draw a $y$-axis on the face of the dam in the vertical direction and choose a convenient origin (often taken to be the base of the dam).
      \item 
        Find the width function $w(y)$ for each value of $y$ on the face of the dam.
      \item 
        If the base of the dam is at $y=0$ and the top of the dam is at $y=a$, then the total force on the dam is
          \[F=\int_a^b \rho g\underbrace{(a-y)}_{\textnormal{depth}}\underbrace{w(y)}_{\textnormal{width}}\,dy.\]
    \end{enumerate}
  \end{thmBox*}

\end{document}
