\documentclass[../mathNotesPreamble]{subfiles}
\begin{document}
%  \relscale{1.4}
  \section{8.3: Trigonometric Integrals}
  \textbf{Important trigonometric identities}
  \vspace*{\stretch{1}}
  \begin{center}
    \begin{tabularx}{0.9\linewidth}{@{}
      >{\hsize=0.9\hsize}X
      >{\hsize=1.1\hsize}X
      @{}}\toprule
      Pythagorean Identities&
      \vspace*{0.25\baselineskip}
      $\sin^2(\theta)+\cos^2(\theta)=1$\\[4\baselineskip]\midrule
      %
      Angle sum formulas& 
      \vspace*{0.25\baselineskip}
      $\begin{aligned}
        \sin(\alpha\pm\beta)&=\sin(\alpha)\cos(\beta)\pm\cos(\alpha)\sin(\alpha)\\[0.25\baselineskip]
        \cos(\alpha\pm\beta)&=\cos(\alpha)\cos(\beta)\mp\sin(\alpha)\sin(\beta)\\[0.25\baselineskip]
      \end{aligned}$\\\midrule
      %
      Double angle formulas&
      \vspace*{0.25\baselineskip}
      $\begin{aligned}
        \sin(2\theta)&=2\sin(\theta)\cos(\theta)\\[0.25\baselineskip]
        \cos(2\theta)&=\cos^2(\theta)-\sin^2(\theta)\\[0.25\baselineskip]
      \end{aligned}$\\\midrule
      %
      Half angle formulas&
      \vspace*{0.25\baselineskip}
      $\begin{aligned}
        \sin^2(\theta)&=\dfrac{1-\cos(2\theta)}{2}\\[0.25\baselineskip]
        \cos^2(\theta)&=\dfrac{1-\cos(2\theta)}{2}\\[0.25\baselineskip]
      \end{aligned}$\\\bottomrule
    \end{tabularx}
  \end{center}
  \vspace*{\stretch{1}}
  \pagebreak

  \textbf{Derivation of angle sum formulas}

  \noindent
  \begin{minipage}{0.5\linewidth}
    \begin{align*}
      \sin(\alpha)&=\frac{\overline{DE}}{\overline{EF}}=\dfrac{\overline{DE}}{\sin(\beta)}
       &\Rightarrow \overline{DE}=\sin(\alpha)\sin(\beta)\\[\baselineskip]
      %
      \cos(\alpha)&=\frac{\overline{DF}}{\overline{EF}}=\dfrac{\overline{DF}}{\sin(\beta)}
       &\Rightarrow \overline{DF}=\cos(\alpha)\sin(\beta)\\[\baselineskip]
      %
      \sin(\beta)&=\frac{\overline{CE}}{\overline{AE}}=\dfrac{\overline{CE}}{\cos(\beta)}
       &\Rightarrow \overline{CE}=\sin(\alpha)\cos(\beta)\\[\baselineskip]
      %
      \cos(\beta)&=\frac{\overline{AC}}{\overline{AE}}=\dfrac{\overline{AC}}{\cos(\beta)}
       &\Rightarrow \overline{AC}=\cos(\alpha)\cos(\beta)\\[\baselineskip]
    \end{align*}
  \end{minipage}%
  \begin{minipage}{0.45\linewidth}
    \begin{flushright}
      \begin{tikzpicture}[declare function={c=4; s=3; h=sqrt(c^2+s^2); FE=2.25;}]
        \coordinate (A) at (0,0);
        \coordinate (C) at (c,0);
        \coordinate (E) at (c,s);
        \coordinate (F) at ($(E)+(-FE*s/h,FE*c/h)$);
        \coordinate (B) at ($(C)!(F)!(A)$);
        \coordinate (D) at ($(F)!(E)!(B)$);

        \draw[fill=ClemsonOrange!50, opacity=0.5] (A) -- (C) -- (E) -- cycle;
        \draw[fill=ClemsonPurple!50, opacity=0.5, text opacity=1.0] (A) -- node[pos=0.5, above left] {$1$} (F) -- (E);
        \draw (F) -- (B) (D) -- (E);

        \node[below left, inner sep=1pt] at (A) {$A$};
        \node[below, inner sep=2pt] at (B) {$B$};
        \node[below right, inner sep=1pt] at (C) {$C$};
        \node[left, inner sep=1.5pt] at (D) {$D$};
        \node[right, inner sep=2pt] at (E) {$E$};
        \node[above, inner sep=2pt] at (F) {$F$};

        \tkzLabelAngle[pos = 1.05](C,A,E){$\alpha$}
        \tkzLabelAngle[pos = 1.05](D,F,E){$\alpha$}
        \tkzLabelAngle[pos = 1.05](E,A,F){$\beta$}

        \draw[decorate, decoration={brace, amplitude=5pt}] ($(B)-(0,20pt)$)--($(A)-(0,20pt)$) node[pos=0.5, below, inner sep=7.5pt] {$\cos(\alpha+\beta)$};
        \draw[decorate, decoration={brace, amplitude=5pt}] ($(A)-(10pt,0)$)--($(A|-F)-(10pt,0)$) node[pos=0.5, above, inner sep=7.5pt, sloped] {$\sin(\alpha+\beta)$};
      \end{tikzpicture}
    \end{flushright}
  \end{minipage}%

  \textbf{Derivation of the double angle formulas}

  \begin{align*}
    \sin(2\theta)&=\sin(\theta+\theta)
      =\sin(\theta)\cos(\theta)+\cos(\theta)\sin(\theta)
      =2\sin(\theta)\cos(\theta)\\[0.5\baselineskip]
    \cos(2\theta)&=\cos(\theta+\theta)
      =\cos(\theta)\cos(\theta)-\sin(\theta)\sin(\theta)
      =\cos^2(\theta)-\sin^2(\theta)
  \end{align*}
  \vspace*{\stretch{1}}

  \textbf{Derivation of the half angle formulas}

  \noindent
  Start with the cosine double angle formula:
  \begin{align*}
    \cos(2\theta)&=\cos^2(\theta)-\sin^2(\theta)
      =\boxed{2\cos^2(\theta)-1}
      =\boxed{1-2\sin^2(\theta)}
  \end{align*}
  Solve for either $\sin^2(\theta)$ or $\cos^2(\theta)$:
  \begin{align*}
    \sin^2(\theta)&=\frac{1-\cos(2\theta)}{2}&
    \cos^2(\theta)&=\frac{1+\cos(2\theta)}{2}
  \end{align*}
  \vspace*{\stretch{1}}
  \pagebreak

  \begin{ex*}
    Evaluate the integral $\displaystyle \int \cos^5(x)\,dx$.
  \end{ex*}
  \vspace*{\stretch{1}}
  \pagebreak

  \begin{ex*}
    Evaluate the integral $\displaystyle \int\sin^3(x)\cos^{3/2}(x)\,dx$.
  \end{ex*}
  \vspace*{\stretch{1}}
  \pagebreak

  \begin{ex*}
    Evaluate the integral $\displaystyle \int 20\sin^2(x)\cos^2(x)\,dx$
  \end{ex*}
  \vspace*{\stretch{1}}
  \pagebreak

  \begin{ex*}
    Evaluate the integral $\displaystyle \int\sec^6(x)\tan^4(x)\,dx$.
  \end{ex*}
  \vspace*{\stretch{1}}
  \pagebreak

  \begin{ex*}
    Evaluate the integral $\displaystyle \int 35\tan^5(x)\sec^4(x)\,dx$.
  \end{ex*}
  \vspace*{\stretch{1}}
  \pagebreak

  \begin{ex*}
    Consider the region bounded by $y=\sec(x)$ and $y=\cos(x)$ for $0\leq x\leq \nicefrac{\pi}{3}$. Find the volume of the solid generated when rotating this region about the line $y=-1$.
  \end{ex*}
  \begin{flushright}
    \begin{tikzpicture}[declare function={
      PI=3.141592653589793;}]
      \begin{axis}[
        axis lines=center,
        axis line style={black,->},
        xmin=-0.95*PI, xmax=0.95*PI,
        vasymptote=PI/3,  
        ymin=-1.75, ymax=2.5,
        xtick={-3.141592653589793,
               -2.0943951023931953,
               -1.0471975511965976,
               0.0,
               1.0471975511965976,
               2.0943951023931953,
               3.141592653589793},
        xticklabels={$\displaystyle -\pi$,
                     $\displaystyle \nicefrac{-2\pi}{3}$,
                     $\displaystyle \nicefrac{\pi}{3}$,
                     $\displaystyle 0$,
                     $\displaystyle \nicefrac{\pi}{3}$,
                     $\displaystyle \nicefrac{2\pi}{3}$,
                     $\displaystyle \pi$},
        height=1.75in, width=0.9*0.5\linewidth,
        ticklabel style={font=\footnotesize,inner sep=0.5pt,fill=white,opacity=1.0, text opacity=1},
        xlabel=$x$, xlabel style={at={(ticklabel* cs:1)},anchor=north west},
        ylabel=$y$, ylabel style={at={(ticklabel* cs:1)},anchor=south west},
        every axis plot/.append style={line width=0.95pt, color=blue, samples=100}
        ]
        \addplot[-, red, name path =A] expression[domain=-PI:PI]{cos(deg(x))} node[black, above left, pos=0.35, font=\normalsize, inner sep=1pt] {$y=\cos(x)$};
        \foreach \n in {-1,0,1}{
          \addplot[-, name path = B] expression[domain=\n*PI-0.9*PI/2:\n*PI+0.9*PI/2, unbounded coords=jump]{sec(deg(x))};
          \addplot[fill=ClemsonPurple!55] fill between[of=A and B, soft clip={domain=0:PI/3}];
        }
      \end{axis}
    \end{tikzpicture}
  \end{flushright}
  \vspace*{\stretch{1}}
  \pagebreak

  \begin{ex*}
    Find the length of the curve $y=\ln\parens{2\sec(x)}$ on the interval $\sbrkt{0,\nicefrac{\pi}{6}}$.
  \end{ex*}
  \vspace*{\stretch{1}}
  \pagebreak

  \vspace*{\stretch{1}}
  \begin{center}
    \renewcommand{\arraystretch}{1.65}
    \relscale{0.9}
    \begin{tabularx}{\linewidth}{@{}
      >{\hsize=0.65\hsize}X
      >{\hsize=1.35\hsize}X
      @{}}\toprule
      $\displaystyle \int \sin^m(x)\cos^n(x)\,dx$& \textbf{Strategy}\\
      $m$ odd and positive, $n$ real& 
      Split off $\sin(x)$, rewrite the resulting even power of $\sin(x)$ in terms of $\cos(x)$, and then use $u=\cos(x)$.\\
      %
      $n$ odd and positive, $m$ real& 
      Split off $\cos(x)$, rewrite the resulting even power of $\cos(x)$ in terms of $\sin(x)$, and then use $u=\sin(x)$.\\
      %
      $m$ and $n$ both even, nonnegative integers&
      Use half-angle formulas to transform the integrand into a polynomial in $\cos(2x)$, and apply the preceding strategies once again to powers of $\cos(2x)$ greater than $1$.\\
      \midrule
      $\displaystyle \int \tan^m(x)\sec^n(x)\,dx$\\
      $n$ even and positive, $m$ real& 
      Split off $\sec^2(x)$, rewrite the remaining even power of $\sec(x)$ in terms of $\tan(x)$, and use $u=\tan(x)$.\\
      %
      $m$ odd and positive, $n$ real&
      Split off $\sec(x)\tan(x)$, rewrite the remaining even power of $\tan(x)$ in terms of $\sec(x)$, and use $u=\sec(x)$.\\
      %
      $n$ even and positive, $n$ odd and positive&
      Rewrite $\tan^m(x)$ in terms of $\sec(x)$\\\midrule
      %
      $\displaystyle \int \sec^n(x)\,dx$\\
      $n$ odd&
      Use integration by parts with $u=\sec^{n-2}(x)$ and \newline$dv=\sec^2(x)\,dx$\\
      %
      $m$ even&
      Split off $\sec^2(x)$, rewrite the remaining powers of $\sec(x)$ in terms of $\tan(x)$, and use $u=\tan(x)$.\\\midrule
      %
      $\displaystyle \int \tan^m(x)\,dx$&
      Split off $\tan^2(x)$ and rewrite in terms of $\sec(x)$. Expand into difference of integrals substituting $u=\tan(x)$. Repeat the process as needed for remaining powers of $\tan(x)$.\\
      \bottomrule
    \end{tabularx}
  \end{center}
  \vspace*{\stretch{1}}
  \pagebreak

\end{document}
