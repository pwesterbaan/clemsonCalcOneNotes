\documentclass[../mathNotesPreamble]{subfiles}
\begin{document}
%  \relscale{1.4}
  \section{8.5: Partial Fractions}
  \begin{ex*}
    Simplify $\displaystyle f(x)=\frac{1}{x-2}+\frac{2}{x+4}$ by finding a common denominator.
  \end{ex*}
  \vspace*{\stretch{0.5}}

  \begin{thmBox*}[Procedure: Partial Fractions with Simple Linear Factors]
    Suppose $f(x)=p(x)/q(x)$, where $p$ and $q$ are polynomials with no common factors and with the degree of $P$ less than the degree of $q$. Assume $q$ is the product of simple linear factors. The partial fraction decomposition is obtained as follows.
    \begin{enumerate}[label=\textbf{Step \arabic*:}, itemindent=1.5\labelwidth]
      \item \textbf{Factor the denominator $q$} in the form $(x-r_1)(x-r_2)\dots(x-r_n)$
      \item \textbf{Partial fraction decomposition}
        \[\frac{p(x)}{q(x)}=\frac{A_1}{(x-r_1)}+\frac{A_2}{(x-r_2)}+\dots+\frac{A_n}{(x-r_n)}.\]
      \item \textbf{Clear denominators} Multiply both sides of the equation in Step 2 by $q(x)=(x-r_1)(x-r_2)\dots(x-r_n)$
      \item \textbf{Solve for coefficients} Equate like powers of $x$ in Step 3 to solve for the undetermined coefficients $A_1,\dots,A_n$.
    \end{enumerate}
  \end{thmBox*}

  \begin{ex*}
    Perform partial fraction decomposition on $\displaystyle f(x)=\frac{3x}{x^2+2x-8}$.
  \end{ex*}
  \vspace*{\stretch{1}}
  \pagebreak

  \begin{ex*}
    $\displaystyle \int \frac{28x^3-56x^2+9}{x^2-2x}$
  \end{ex*}
  \vspace*{\stretch{1}}
  \pagebreak

  \begin{thmBox*}[Procedure: Partial Fractions for Repeated Linear Factors]
    Suppose the repeated linear factor $(x-r)^m$ appears in the denominator of a proper rational function in reduced form. The partial fraction decomposition has a partial fraction for each power of $(x-r)$ up to and including the $m$th power; that is, the partial fraction decomposition contains the sum  
    \[\frac{A_1}{(x-r)}+\frac{A_3}{(x-r)^2}+\frac{A_3}{(x-r)^3}+\dots+\frac{A_m}{(x-r)^m}\]
    where $A_1,\dots,A_m$ are constants to be determined.
  \end{thmBox*}
  \begin{ex*}
    Setup the partial fraction decomposition for $\displaystyle f(x)=\frac{x^3-8x+19}{x^4+3x^3}$.
  \end{ex*}
  \vspace*{\stretch{1}}
  \begin{ex*}
    Setup the partial fraction decomposition for $\displaystyle g(x)=\frac{2}{x^5-6x^4+9x^3}$.
  \end{ex*}
  \vspace*{\stretch{1}}
  \pagebreak

  \begin{ex*}
    Evaluate $\displaystyle \int \frac{x^2+1}{(2x-3)(x-2)^2}\,dx$.
  \end{ex*}
  \vspace*{\stretch{1}}
  \pagebreak

  \begin{ex*}
    Evaluate $\displaystyle \int \frac{8}{3x^3+7x^2+4x}\,dx$.
  \end{ex*}
  \vspace*{\stretch{1}}
  \pagebreak


  \begin{thmBox*}[Procedure: Partial Fractions with Simple Irreducible Quadratic Factors]
    Suppose a simple irreducible factor $ax^2+bx+c$ appears in the denominator of a proper rational function in reduced form. The partial fraction decomposition contains a term of the form
      \[\frac{Ax+B}{ax^2+bx+c},\]
    where $A$ and $B$ are unknown coefficients to be determined.
  \end{thmBox*}

  \begin{ex*}
    Perform partial fraction decomposition on the following fractions or identify them as irreducible.
  \end{ex*}
  \begin{tasks}[after-item-skip=\stretch{1}, label=, item-indent=0mm](1)
    \task $\displaystyle \frac{1}{x^2-13x+43}$
    \task $\displaystyle \frac{x^2}{(x-4)(x+5)}$
  \end{tasks}
  \vspace*{\stretch{1}}
  \pagebreak

  \begin{ex*}
    Perform partial fraction decomposition on the following fractions or identify them as irreducible.
  \end{ex*}
  \begin{tasks}[after-item-skip=\stretch{1}, label=, item-indent=0mm](1)
    \task $\displaystyle \frac{7}{(x^2+1)^2}$
    \task $\displaystyle \frac{1}{x^2+11x+28}$
  \end{tasks}
  \vspace*{\stretch{1}}
  \pagebreak

  \begin{ex*}
    Evaluate $\displaystyle \int \frac{4x}{(x+1)(x^2+1)}\,dx$
  \end{ex*}
  \vspace*{\stretch{1}}
  \pagebreak

  \begin{ex*}
    Evaluate $\displaystyle \int \frac{3x^2+2x+12}{(x^2+4)^2}\,dx$
  \end{ex*}
  \vspace*{\stretch{1}}
  \pagebreak

  \begin{ex*}
    Evaluate $\displaystyle \int \frac{1}{x\sqrt{1+2x}}\,dx$ using the substitution $u=\sqrt{1+2x}$.
  \end{ex*}
  \vspace*{\stretch{1}}
  \pagebreak

  \begin{thmBox*}[Summary: Partial Fraction Decomposition]
    Let $f(x)=p(x)/q(x)$ be a proper rational function in reduced form. Assume the denominator $q$ has been factored completely over the real numbers and $m$ is a positive integer.
    \begin{enumerate}
      \item \textbf{Simple linear factor:} A factor $x-r$ in the denominator requires the partial fraction $\dfrac{A}{x-r}$.
      \item \textbf{Repeated linear factor: } A factor $(x-r)^m$ with $m>1$ in the denominator requires the partial fractions
        \[\frac{A_1}{(x-r)}+\frac{A_2}{(x-r)^2}+\frac{A_3}{(x-r)^3}+\dots+\frac{A_m}{(x-r)^m}.\]
      \item \textbf{Simple irreducible quadratic factor: } An irreducible factor $ax^2+bx+c$ in the denominator requires the partial fraction 
        \[\frac{Ax+B}{ax^2+bx+c}.\]
      \item \textbf{Repeated irreducible quadratic factor:} An irreducible factor $(ax^2+bx+c)^m$ with $m>1$ in the denominator requires the partial fractions
        \[\frac{A_1x+B_1}{ax^2+bx+c}+\frac{A_2x+B_2}{(ax^2+bx+c)^2}+\dots+\frac{A_mx+B_m}{(ax^2+bx+c)^m}.\]
    \end{enumerate}
  \end{thmBox*}
  \vspace*{\stretch{1}}
  \pagebreak

\end{document}
