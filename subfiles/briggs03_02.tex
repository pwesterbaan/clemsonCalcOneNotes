\documentclass[../mathNotesPreamble]{subfiles}
\begin{document}
%\relscale{1.4}
\section{3.2: The Derivative as a Function}
\begin{defn*}[The Derivative Function]\ 

  The \textbf{derivative} of $f$ is the function
    $$f'(x)=\lim_{h \to 0} \frac{f(x+h)-f(x)}{h}$$
  provided the limit exists and $x$ is in the domain of $f$. If $f'(x)$ exists, we say that $f$ is \textbf{differentiable} at $x$. If $f$ is differentiable at every point on an open interval $I$, we say that $f$ is differentiable on $I$.
\end{defn*}
\vspace*{10pt}

\textit{Note:} The derivative of $f$ has several notations:
\begin{quote}
  \begin{tasks}(4)
    \task[] $f'(x)$
    \task[] $\dfrac{d}{dx}\parens{f(x)}$
    \task[] $D_x\parens{f(x)}$
    \task[] $y'(x)$
  \end{tasks}
\end{quote}

\textit{Note:} The derivative of $f$ evaluated at $a$ has several notations:
\begin{quote}
  \begin{tasks}(4)
    \task[] $f'(a)$
    \task[] $\left.\dfrac{d}{dx} f(x)\right|_{x=a}$
    \task[] $\left.\dfrac{df}{dx}\right|_{x=a}$
    \task[] $y'(a)$
  \end{tasks}
\end{quote}
\vspace*{2.5pt}
\begin{ex*}
  Use the limit definition of a derivative to find the derivative function $f'(x)$ for the function $f(x)=5x^2-6x+1$.
\end{ex*}

\begin{flushright}
\vspace*{\stretch{1}}
  \begin{tikzpicture}
    \begin{axis}[
      axis lines=center,
      axis line style={->},
      xmin=-1, xmax=2,
      ymin=-6, ymax=6,
      xtick={-4,-3,...,4},
      ytick={-7,-5,...,7},
      minor x tick num=4,
      minor y tick num=1,
      enlargelimits={abs=0.5},
      ticklabel style={font=\normalsize, inner sep=0.75pt,fill=white},
      every axis plot/.append style={line width=0.95pt}
      ]
      \addplot[<->] expression[domain=-0.6:1.8, blue, samples=50] {5*x^2-6*x+1};
    \end{axis}
  \end{tikzpicture}
\end{flushright}
\vspace*{-35pt}
\pagebreak

\begin{ex*}
  Find the derivative of the following functions. If a point is specified, find the tangent line at that point.\\
  
  $f(w)=\sqrt{4w-3},\ w=3$
  \vspace*{\stretch{1}}
  
  $g(v)=\dfrac{v}{v+2}, v=0$
  \vspace*{\stretch{1}}
\end{ex*}
\pagebreak
  
  $h(m)=1+\sqrt m,\ m=\sfrac{1}{4},\ m=1$
  \vspace*{\stretch{1}}
  
  $\dfrac{d}{dx}\parens{\sqrt{ax+b}}$. \hfill Then find $\dfrac{d}{dx}\parens{f(x)}$ where $f(x)=\sqrt{5x+9}$ and find $f'(-1)$.
  \vspace*{\stretch{1}}
  \pagebreak
  
  $\dfrac{d}{dx}\parens{ax^2+bx+c}$
  \vspace*{\stretch{1}}

\noindent
\begin{minipage}{0.5\linewidth}
  \begin{tikzpicture}
    \begin{axis}[
      grid=both,
      grid style={line width=0.35pt, draw=gray!75},
      axis lines=center,
      axis equal,
      axis line style={->},
      xmin=-5.75, xmax=5.5,
      ymin=-2.25, ymax=6.25,
      xtick={-6,-4,...,8},
      ytick={-4,-1,2,5,8},
      minor x tick num=1,
      minor y tick num=2,
      ticklabel style={font=\normalsize, inner sep=0.75pt,fill=white,opacity=0.65, text opacity=1},
      xlabel=$x$, xlabel style={at={(ticklabel* cs:1)},anchor=north west},
      ylabel=$y$, ylabel style={at={(ticklabel* cs:1)},anchor=south west},
      every axis plot/.append style={line width=0.95pt}
      ]
      \addplot[-] expression[domain=-5:-2,blue] {-x};
      \addplot[-] expression[domain=-2:0,blue] {x+4};
      \addplot[-] expression[domain=0:5,blue] {-0.5*x+4};
      \addplot[-] expression[domain=-5:-2,red!80] {-1};
      \addplot[-] expression[domain=-2:0,red!80] {1};
      \addplot[-] expression[domain=0:5,red!80] {-0.5} node[below, black, pos=0.6, fill=white, yshift=-5pt, opacity=0.65, text opacity=1] {$y=f'(x)$};
      \addplot[holdot, draw=red!80] coordinates{(-2,-1)(-2,1)(0,1)(0,-0.5)};
    \end{axis}
  \end{tikzpicture}
\end{minipage}%
\begin{minipage}{0.5\linewidth}
  \renewcommand{\arraystretch}{1.25}
  \begin{flushright}
    \begin{tabular}{@{}l@{\hspace*{0.35in}}l@{}}\toprule
      Function& Derivative\\\midrule
      Increasing\\
      Decreasing\\
      Smooth Min/Max\\
      Constant\\
      Linear\\
      Quadratic\\\bottomrule
    \end{tabular}
  \end{flushright}
\end{minipage}
\pagebreak

\begin{ex*}
  Graph the slope graph of the following function
  \vspace*{\stretch{0.15}}
  
\begin{tikzpicture} [scale=1.25]
  \begin{groupplot}[
    group style={group size=1 by 2}, 
    axis lines=center,
    axis equal,
    axis line style={->},
    xmin=-4.5, xmax=2.5,
    ymin=-3, ymax=4,
    xtick={-6,-4,...,8},
    minor x tick num=1,
    ymajorticks=false,
    ticklabel style={font=\normalsize, inner sep=0.75pt,fill=white,opacity=0.65, text opacity=1},
    every axis plot/.append style={line width=0.95pt}
    ] 
    \nextgroupplot 
      \addplot[<->] expression[domain=-4:2, blue, samples=100] {-0.10*(x+4)*(x+2)*(x)*(x-2)+2};
    \nextgroupplot 
  \end{groupplot} 
\end{tikzpicture}
\vspace*{\stretch{1}}
\end{ex*}
\pagebreak

\begin{ex*}
  Graph the slope graph of the following functions
  \vspace*{\stretch{1}}
  
\noindent
\begin{minipage}{0.5\linewidth}
  \begin{center}
    \begin{tikzpicture}[scale=1.2]
      \begin{groupplot}[
        group style={group size=1 by 2}, 
        axis lines=center,
        axis equal,
        axis line style={->},
        xmin=-5.5, xmax=4.5,
        ymin=-3, ymax=3.25,
        xtick={-6,-4,...,8},
        minor x tick num=1,
        ymajorticks=false,
        ticklabel style={font=\normalsize, inner sep=0.75pt,fill=white,opacity=0.65, text opacity=1},
        every axis plot/.append style={line width=0.95pt}
        ] 
        \nextgroupplot 
          \addplot[<-] expression[domain=-5.5:-3, blue, samples=100] {1.5*x+7.5};
          \addplot[-] expression[domain=-3:-1, blue, samples=100] {3-sqrt(4-(x+1)^2};
          \addplot[-] expression[domain=-1:1,blue] {-x};
          \addplot[-] expression[domain=1:3,blue] {-1};
          \addplot[->] expression[domain=3:4.5,blue] {3*ln(x-2)-1};
        \nextgroupplot 
      \end{groupplot} 
    \end{tikzpicture}
  \end{center}
\end{minipage}%
\begin{minipage}{0.5\linewidth}
  \begin{center}
    \begin{tikzpicture}[scale=1.2]
      \begin{groupplot}[
        group style={group size=1 by 2}, 
        axis lines=center,
        axis line style={->},
        xmin=-2.25*pi, xmax=2.25*pi,
        ymin=-1.25, ymax=1.25,
        xtick={-6,-4,...,8},
        %minor x tick num=1,
        xmajorticks=false,
        ymajorticks=false,
        ticklabel style={font=\normalsize, inner sep=0.75pt,fill=white,opacity=0.65, text opacity=1},
        every axis plot/.append style={line width=0.95pt}
        ] 
        \nextgroupplot 
          \addplot[-] expression[domain=-2*pi:2*pi, blue, samples=1000] {sin(\x r)};
        \nextgroupplot 
      \end{groupplot} 
    \end{tikzpicture}
  \end{center}
\end{minipage}%

\vspace*{\stretch{1}}
\end{ex*}
\pagebreak

\begin{thmBox*}[When is a Function Not Differentiable at a Point?]
  A function $f$ is \textit{not} differentiable at $a$ if at least one of the following conditions holds:
    \begin{enumerate}
      \item $f$ is not continuous at $a$
      \item $f$ has a corner at $a$
      \item $f$ has a vertical tangent at $a$
    \end{enumerate}
\end{thmBox*}

\begin{center}
  \begin{tikzpicture}[scale=0.825]
    \begin{groupplot}[
      group style={group size=3 by 1}, 
      axis lines=center,
      axis equal,
      axis line style={->},
      xmin=-0.5, xmax=4.5,
      ymin=-4, ymax=4,
      xmajorticks=false,
      ymajorticks=false,
      ticklabel style={font=\normalsize, inner sep=0.75pt,fill=white,opacity=0.65, text opacity=1},
      every axis plot/.append style={line width=0.95pt, blue}
      ]
      \nextgroupplot
        \addplot[<-] expression[domain=-10:2,samples=100]{(x+4)^0.5-1};
        \addplot[->] expression[domain=2:6.5,samples=100]{(x+4)^0.5};
        \addplot[holdot] coordinates{(2,1.4495)(2,2.4495)};
      \nextgroupplot
        \addplot[<->] expression[domain=-2.75:6.5,samples=200]{1.75*abs(x-3)^(2/3)-2};
      \nextgroupplot
        \addplot[<->] expression[domain=-2.75:6.5,samples=200]{1.5*(x-3)/abs(x-3)*abs(x-3)^(1/3)};
    \end{groupplot}
  \end{tikzpicture}
\end{center}

\begin{center}

  \fbox{\parbox{0.75\linewidth}{
    \textbf{Theorem: Differentiable Implies Continuous}
  
    If $f$ is differentiable at $a$, then $f$ is continuous at $a$.
  }}\\[15pt]
  \fbox{\parbox{0.75\linewidth}{
    \textbf{Theorem: Not Continuous Implies Not Differentiable}
    
    If $f$ is not continuous at $a$, then $f$ is not differentiable at $a$.
    }}
\end{center}
\vspace*{\stretch{1}}

\begin{center}
  \begin{tikzpicture}[scale=1]
    \begin{groupplot}[
      group style={group size=2 by 1}, 
      axis lines=center,
      axis line style={->},
      xmin=-4.5, xmax=4.5,
      ymin=-4, ymax=4,
      xmajorticks=false,
      ymajorticks=false,
      ticklabel style={font=\normalsize, inner sep=0.75pt,fill=white,opacity=0.65, text opacity=1},
      every axis plot/.append style={line width=0.95pt, blue}
      ]
      \nextgroupplot
        \addplot[->] expression[domain=0:4.5, samples=200] {sqrt(x)};
      \nextgroupplot
        \addplot[<->] expression[domain=-4.5:-1.125, samples=200] {-1/((x+1)*(x-1))};
        \addplot[<->] expression[domain=-0.8762:0.8835, samples=200] {-x/((x+1)*(x-1))};
        \addplot[<->] expression[domain=1.125:4.5, samples=200] {-1/((x+1)*(x-1))};
    \end{groupplot}
  \end{tikzpicture}
\end{center}
\pagebreak
\begin{defn*}
  The \textbf{normal} line at $(a,f(a))$ is the line perpendicular to the tangent line that crosses the point $(a,f(a))$.
\end{defn*}

\begin{ex*}
  Find the derivative of $g(x)=\sqrt{x-2}$. Use your result to find the tangent line and the normal line at $x=11$.
\end{ex*}
\pagebreak
\begin{ex*}
  Find the tangent line and normal line of $h(x)=\dfrac{2}{\sqrt{x^2+x-2}}$ at $x=4$.
\end{ex*}
\pagebreak
\end{document}
