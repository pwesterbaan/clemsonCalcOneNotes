\documentclass[../mathNotesPreamble]{subfiles}
\begin{document}
\tikzset{custNode/.style={color=black,inner sep=0.5pt,fill=none,opacity=1.0, text opacity=1}}
\tikzset{dLine/.style={mark=none, dashed, opacity=1.0, blue!85, line width=0.5pt}}
%\relscale{1.4}
\section{JIT 9.1: Definition of $\arcsin x$, the Inverse Sine Function}
  Recall that a function has an inverse if it is 1-to-1 (e.g. it passes the horizontal line test).
  \begin{center}
    \begin{tikzpicture}
      \begin{groupplot}[
        group style={group size=3 by 1, horizontal sep=1cm},
        width=0.36\linewidth,
        axis lines=center,
        axis line style={->},
        ticklabel style={font=\footnotesize,inner sep=0.5pt,fill=white,opacity=1.0, text opacity=1},
        xlabel=$x$, xlabel style={at={(ticklabel* cs:1)},anchor=north west},
        ylabel=$y$, ylabel style={at={(ticklabel* cs:1)},anchor=south west},
        every axis plot/.append style={line width=0.95pt, color=blue}
        ]
        \nextgroupplot[
          xmin=-4.25, xmax=6.25,
          ymin=-4.25, ymax=6.25,
          xmajorticks=false,
          ymajorticks=false,
          every axis plot/.append style={samples=100},
          ]
          \addplot[dashed] expression[domain=-4.25:6.25, black]{x};
          \addplot[->] expression[domain=-4.25:ln(6.25), ClemsonPurple] {e^x} node[custNode, below right, pos=1] {$a^x$};
          \addplot[->] expression[domain=exp(-4.25):6.25, ClemsonOrange] {ln(x)} node[custNode, above left, pos=1] {$\log_a(x)$};
        \nextgroupplot[
          xmin=-0.5, xmax=3.25,
          ymin=-0.5, ymax=3.25,
          xmajorticks=false,
          ymajorticks=false,
          every axis plot/.append style={samples=500},
          ]
          \addplot[dashed] expression[domain=-0.5:3.25, black]{x};
          \addplot[dashed] expression[domain=-0.5:0, ClemsonPurple]{x^2};
          \addplot[->] expression[domain=0:sqrt(3.25), ClemsonPurple]{x^2} node[custNode, below right, pos=1] {$x^2$};
          \addplot[->] expression[domain=0:3.25, ClemsonOrange]{sqrt(x)} node[custNode, above left, pos=1] {$\sqrt x$};
        \nextgroupplot[
          xmin=-(pi/2+0.5), xmax=(pi/2+0.775),
          ymin=-(pi/2+0.5), ymax=(pi/2+0.65),
          xmajorticks=false,
          ymajorticks=false,
          every axis plot/.append style={samples=100},
          ]
          \addplot[dashed] expression[domain=-(pi/2+0.5):(pi/2+0.775), black]{x};
          \addplot[dashed] expression[domain=-pi/2-0.5:-pi/2, ClemsonPurple]{sin(x*180/pi)};
          \addplot[dashed] expression[domain=pi/2:pi/2+0.775, ClemsonPurple]{sin(x*180/pi)};
          \addplot[-] expression[domain=-pi/2:pi/2, ClemsonPurple]{sin(x*180/pi)} node[custNode, below right, pos=0.85] {$\sin(x)$};
          \addplot[-] expression[domain=-1:1, ClemsonOrange]{asin(x)*pi/180} node[custNode, above, pos=1, xshift=-5pt] {$\arcsin(x)$};
      \end{groupplot}
    \end{tikzpicture}
    
    Notice that $x^2$ and $\sin(x)$ are on restricted domains.
  \end{center}

Without restriction on its domain, $\sin(x)$ is NOT 1-to-1:
\begin{center}
  \begin{tikzpicture}
    \begin{axis}[
      axis lines=center,
      axis line style={->},
      xmin=-7, xmax=7,
      ymin=-1.25, ymax=1.25,
      xtick={-6.28318, -4.7123889, ..., 6.28318},
      xticklabels={$-2\pi$, $-\frac{3\pi}{2}$,$-\pi$, $-\frac{\pi}{2}$, ,
         $\frac{\pi}{2}$,$\pi$, $\frac{3\pi}{2}$, $2\pi$},
      height=1.75in, width=0.9\linewidth,
      ticklabel style={font=\small, inner sep=0.75pt,fill=white},
      xlabel=$x$, xlabel style={at={(ticklabel* cs:1)},anchor=north west},
      ylabel=$y$, ylabel style={at={(ticklabel* cs:1)},anchor=south west},
      every axis plot/.append style={line width=0.95pt, color=blue}
      ]
      \addplot[domain=-6.28:6.28, ClemsonPurple, samples=101] {sin(deg(x))} node[custNode, pos=0.7, above right] {$\sin(x)$};
    \end{axis}
  \end{tikzpicture}
\end{center}

The range of $\sin(x)$ is $[-1,1]$ and all of these values are attained on a restricted domain of $\sbrkt{\nicefrac{-\pi}{2},\nicefrac{\pi}{2}}$:

\begin{center}
  \begin{tikzpicture}
    \begin{groupplot}[
      group style={group size=2 by 1, horizontal sep=4cm},
      axis lines=center,
      axis line style={->},
      xmin=-(pi/2+0.95), xmax=pi/2+0.95,
      ymin=-(pi/2+0.95), ymax=pi/2+0.95,
      width=0.4\linewidth,
      ticklabel style={font=\small, inner sep=0.75pt,fill=white},
      xlabel=$x$, xlabel style={at={(ticklabel* cs:1)},anchor=north west},
      ylabel=$y$, ylabel style={at={(ticklabel* cs:1)},anchor=south west},
      every axis plot/.append style={line width=0.95pt, color=blue, samples=100}      
      ]
      \nextgroupplot[
        xtick={-1.570796327,1.570796327},
        ytick={-1,1},
        xticklabels={$-\frac{\pi}{2}$,$\frac{\pi}{2}$},
        ]
        \addplot[-] expression[domain=-pi/2:pi/2, ClemsonPurple]{sin(x*180/pi)};
        \node[custNode] at (-pi/2,1) {$\sin(x)$};
        \addplot[soldot, black] coordinates{(-pi/2,-1)} node[custNode, below, xshift=5pt, yshift=-4pt] {$\parens{\frac{-\pi}{2},-1}$};
        \addplot[soldot, black] coordinates{(pi/2,1)} node[custNode, above, xshift=5pt, yshift=4pt] {$\parens{\frac{\pi}{2},1}$};
      \nextgroupplot[
        xtick={-1,1},
        ytick={-1.570796327,1.570796327},
        yticklabels={$-\frac{\pi}{2}$,$\frac{\pi}{2}$},
        ]
        \addplot[-] expression[domain=-1:1, ClemsonOrange]{asin(x)*pi/180};
        \node[custNode] at (-pi/2,1) {$\arcsin(x)$};
        \addplot[soldot, black] coordinates{(-1,-pi/2)} node[custNode, below left] {$\parens{-1,\frac{-\pi}{2}}$};
        \addplot[soldot, black] coordinates{(1,pi/2)} node[custNode, above right] {$\parens{1,\frac{\pi}{2}}$};
    \end{groupplot}
  \end{tikzpicture}
\end{center}
  \pagebreak
  
  \begin{defn*}[Inverse Sine and Cosine]
    \def\tmp{5pt}
    $y=\sin\inv(x)$ is the value of $y$ such that $x=\sin(y)$, where $\nicefrac{-\pi}{2}\leq y\leq \nicefrac{\pi}{2}$.\\[\tmp]
    $y=\cos\inv(x)$ is the value of $y$ such that $x=\cos(y)$, where $0\leq y\leq \pi$.\\[\tmp]
    The domain of both $\sin\inv(x)$ and $\cos\inv(x)$ is $\set{x\mid -1\leq x\leq 1}$.
  \end{defn*}

  \textit{Note}: The inverse sine function can be denoted as $\arcsin(x)$ or $\sin\inv(x)$.
  
  This means that $\sin\inv(x)\neq \dfrac{1}{\sin(x)}$. 
  
  Similarly, $\arccos(x)$ and $\cos\inv(x)$ denote the inverse cosine functions.
  \vfill
\begin{center}
  \begin{tikzpicture}
    \begin{groupplot}[
      group style={group size=2 by 1, horizontal sep=2cm},
      axis lines=center,
      axis line style={->},
      width=0.45\linewidth,
      ticklabel style={font=\small, inner sep=0.75pt,fill=white},
      xlabel=$x$, xlabel style={at={(ticklabel* cs:1)},anchor=north west},
      ylabel=$y$, ylabel style={at={(ticklabel* cs:1)},anchor=south west},
      every axis plot/.append style={line width=0.95pt, color=blue, samples=100}      
      ]
      \nextgroupplot[
        xmin=-(pi/2+0.95), xmax=pi/2+0.95,
        ymin=-(pi/2+0.95), ymax=pi/2+0.95,
        xtick={-1.570796327,-1,1,1.570796327},
        ytick={-1.570796327,-1,1,1.570796327},
        xticklabels={$-\frac{\pi}{2}$,-1,1,$\frac{\pi}{2}$},
        yticklabels={$-\frac{\pi}{2}$,-1,1,$\frac{\pi}{2}$},
        ]
        \addplot[-] expression[domain=-pi/2:pi/2, ClemsonPurple]{sin(x*180/pi)} node[custNode, below right, pos=0.95] {$\sin(x)$};
        \addplot[-] expression[domain=-1:1, ClemsonOrange]{asin(x)*pi/180} node[custNode, above, pos=1] {$\sin\inv(x)$};
      \nextgroupplot[
        xmin=-1.5, xmax=pi+0.95,
        ymin=-1.5, ymax=pi+0.95,
        xtick={-1,1,3.141592654},
        ytick={-1,1,3.141592654},
        xticklabels={-1,1,$\pi$},
        yticklabels={-1,1,$\pi$},
        ]
        \addplot[-] expression[domain=0:pi, ClemsonPurple]{cos(x*180/pi)} node[custNode, above right, pos=0.95] {$\cos(x)$};
        \addplot[-] expression[domain=-1:1, ClemsonOrange]{acos(x)*pi/180} node[custNode, above, pos=0.0, xshift=7.5pt, yshift=2.5pt] {$\cos\inv(x)$};
    \end{groupplot}
  \end{tikzpicture}
\end{center}
  \vfill
  \begin{ex*}
    Solve the following:
  \end{ex*}
  \begin{tasks}[after-item-skip=\stretch{1}, label=~](3)
    \task $\sin\inv(0)$
    \task $\arcsin(1)$
    \task $\sin\inv\parens{\dfrac{\sqrt 3}{2}}$
    \task $\cos\inv\parens{-1}$
    \task $\cos\inv\parens{-\dfrac{1}{2}}$
    \task $\arccos\parens{-\dfrac{\sqrt3}{2}}$
  \end{tasks}
  \vfill 
  \pagebreak
\section{JIT 9.2: The Functions $\arctan x$ and $\arcsec x$}
  Similar to $\sin\inv(x)$, we also have inverse functions for the restricted $\tan(x)$ and $\sec(x)$ functions.
  \begin{defn*}[Inverse Tangent and Secant]
    \def\tmp{5pt}
    $y=\tan\inv(x)$ is the value of $y$ such that $x=\tan(y)$, where $\nicefrac{-\pi}{2}< y< \nicefrac{\pi}{2}$.\\[\tmp]
    The domain of $\tan\inv(x)$ is $\set{x\mid -\infty< x< \infty}$.\\[\tmp]
    $y=\sec\inv(x)$ is the value of $y$ such that $x=\sec(y)$, where $0\leq y\leq \pi,\  y\neq\nicefrac{\pi}{2}$.\\[\tmp]
    The domain of $\sec\inv(x)$ is $(-\infty,-1]\cup[1,\infty)$
  \end{defn*}

\begin{center}
  \begin{tikzpicture}[declare function={
      limOne=4; limTwo=4;}]
    \begin{groupplot}[
      group style={group size=2 by 1, horizontal sep=2cm},
      axis lines=center,
      axis line style={->},
      width=0.425\linewidth,
      ticklabel style={font=\small, inner sep=0.75pt,fill=white},
      xlabel=$x$, xlabel style={at={(ticklabel* cs:1)},anchor=north west},
      ylabel=$y$, ylabel style={at={(ticklabel* cs:1)},anchor=south west},
      every axis plot/.append style={line width=0.95pt, color=blue, samples=100}      
      ]
      \nextgroupplot[
        xmin=-limOne, xmax=limOne,
        ymin=-limOne, ymax=limOne,
        xtick={-pi/2,-1,1,pi/2},
        ytick={-pi/2,-1,1,pi/2},
        xticklabels={$-\frac{\pi}{2}$,-1,1,$\frac{\pi}{2}$},
        yticklabels={$-\frac{\pi}{2}$,-1,1,$\frac{\pi}{2}$},
        ]
        \addplot[dLine] coordinates {(-pi/2,-limOne) (-pi/2,limOne)};
        \addplot[dLine] coordinates {(pi/2,-limOne) (pi/2,limOne)};
        \addplot[dLine] coordinates {(-limOne,pi/2) (limOne,pi/2)};
        \addplot[dLine] coordinates {(-limOne,-pi/2) (limOne,-pi/2)};
        \addplot[-] expression[domain=-1.32:1.32,
          ClemsonPurple]{tan((x) r)} node[custNode, right, pos=0.95,
          xshift=7pt] {$\tan(x)$};
        \addplot[-] expression[domain=-limOne:limOne,
          ClemsonOrange]{atan(x)*pi/180} node[custNode, below right,
          pos=0.74] {$\tan\inv(x)$};
      \nextgroupplot[
        xmin=-limTwo, xmax=limTwo,
        ymin=-limTwo, ymax=limTwo,
        xtick={-1.570796327,-1,1,1.570796327,3.14159265},
        ytick={-1.570796327,-1,1,1.570796327,3.14159265},
        xticklabels={$-\frac{\pi}{2}$,-1,1,$\frac{\pi}{2}$,$\pi$},
        yticklabels={$-\frac{\pi}{2}$,-1,1,$\frac{\pi}{2}$,$\pi$},
        ]
        \addplot[dLine] coordinates {(pi/2,-limTwo) (pi/2,limTwo)};
        \addplot[dLine] coordinates {(-limTwo,pi/2) (limTwo,pi/2)};
        \addplot[-] expression[domain=0:1.5,
          ClemsonPurple]{1/cos((x) r)} node[custNode, right,
          pos=0.225, xshift=7.5pt] {$\sec(x)$};
        \addplot[-] expression[domain=1.7:pi,
          ClemsonPurple]{1/cos((x) r)};
        \addplot[-] expression[domain=-limTwo:-1,
          ClemsonOrange]{acos(1/x)*pi/180};
        \addplot[-] expression[domain=1:limTwo,
          ClemsonOrange]{acos(1/x)*pi/180} node[custNode, below,
          pos=0.7] {$\sec\inv(x)$};
    \end{groupplot}
  \end{tikzpicture}
\end{center}
  
  The \textit{Just In Time} book does not include information on $\cos\inv(x), \cot\inv(x)$ and $\csc\inv(x)$. Section 1.4 of \textit{Briggs} provides a very nice reference:
  \vfill 
  \begin{defn*}[Other Inverse Trigonometric Functions]
    \def\tmp{5pt}
    $y=\cot\inv(x)$ is the value of $y$ such that $x=\cot(y)$, where $0< y<\pi$.\\[\tmp]
    The domain of $\cot\inv(x)$ is $\set{x\mid -\infty< x< \infty}$.\\[\tmp]
    $y=\csc\inv(x)$ is the value of $y$ such that $x=\csc(y)$, where $\nicefrac{-\pi}{2}\leq y\leq \nicefrac{\pi}{2},\ y\neq 0$.\\[\tmp]
    The domain of $\csc\inv(x)$ is $(-\infty,-1]\cup[1,\infty)$
  \end{defn*}
  \vfill 
  \pagebreak
  
\vfill
  
\begin{center}
  \def\arraystretch{1.5}\tabcolsep=15pt
  \begin{tabular}{@{}LCC@{}}\toprule
    \textnormal{Function}& \lnret[c]{Restricted\\[-10pt]Domain}& \textnormal{Range}\\\midrule
    \sin(x)& [\nicefrac{-\pi}{2},\nicefrac{\pi}{2}]& [-1,1]\\
    \cos(x)& [0,\pi]& [-1,1]\\
    \tan(x)& \parens{\nicefrac{-\pi}{2},\nicefrac{\pi}{2}}& (-\infty,\infty)\\
    \cot(x)& (0,\pi)& (-\infty,\infty)\\
    \sec(x)& [0,\nicefrac{\pi}{2})\cup (\nicefrac{\pi}{2},\pi]& (-\infty,-1]\cup[1,\infty)\\
    \csc(x)& [\nicefrac{-\pi}{2},0)\cup(0,\nicefrac{\pi}{2}]& (-\infty,-1]\cup[1,\infty)\\\bottomrule
  \end{tabular}
\end{center}  
\vfill

\begin{center}
  \def\arraystretch{1.5}\tabcolsep=15pt
  \begin{tabular}{@{}LCC@{}}\toprule
   \textnormal{Function}& \textnormal{Domain}& \textnormal{Range}\\\midrule
    \sin\inv(x) &  [-1,1] &  [\nicefrac{-\pi}{2},\nicefrac{\pi}{2}] \\
    \cos\inv(x) &  [-1,1] &  [0,\pi] \\
    \tan\inv(x) &  (-\infty,\infty) &  \parens{\nicefrac{-\pi}{2},\nicefrac{\pi}{2}} \\
    \cot\inv(x) &  (-\infty,\infty) &  (0,\pi) \\
    \sec\inv(x) &  (-\infty,-1]\cup[1,\infty) &  [0,\nicefrac{\pi}{2})\cup (\nicefrac{\pi}{2},\pi]\\
    \csc\inv(x) &  (-\infty,-1]\cup[1,\infty) &  [\nicefrac{-\pi}{2},0)\cup(0,\nicefrac{\pi}{2}] \\
\bottomrule
  \end{tabular}
\end{center}
\vfill
\pagebreak
\vfill

\begin{center}
  \begin{tikzpicture}[declare function={
    limOne=4; limTwo=4; limThree=4; limFour=4;}]
    \begin{groupplot}[
      group style={group size=2 by 3, horizontal sep=2cm},
      width=0.44\linewidth,
      axis lines=center,
      axis line style={->},
      ticklabel style={font=\footnotesize,inner sep=0.5pt,fill=white,opacity=1.0, text opacity=1},
      xlabel=$x$, xlabel style={at={(ticklabel* cs:1)},anchor=north west},
      ylabel=$y$, ylabel style={at={(ticklabel* cs:1)},anchor=south west},
      every axis plot/.append style={line width=0.95pt, color=blue, samples=100}
      ]
%%%%%%%%%%%%%%%%%%%%%%%%%%%%%%%%%%%%%%%%%%%%%%%%%%%%%%%%%%%%%%%%%%%%%%%%%%%%%%%
      \nextgroupplot[
        xmin=-(pi/2+0.95), xmax=pi/2+0.95,
        ymin=-(pi/2+0.95), ymax=pi/2+0.95,
        xtick={-pi/2,-1,1,pi/2},
        ytick={-pi/2,-1,1,pi/2},
        xticklabels={$-\frac{\pi}{2}$,-1,1,$\frac{\pi}{2}$},
        yticklabels={$-\frac{\pi}{2}$,-1,1,$\frac{\pi}{2}$},
        ]
        \addplot[dLine] coordinates {(-1,-limOne) (-1,limOne)};
        \addplot[dLine] coordinates {(1,-limOne) (1,limOne)};
        \addplot[dLine] coordinates {(-limOne,1) (limOne,1)};
        \addplot[dLine] coordinates {(-limOne,-1) (limOne,-1)};
        \addplot[-] expression[domain=-pi/2:pi/2,
          ClemsonPurple]{sin((x) r)} node[custNode, below right,
          pos=0.95] {$\sin(x)$};
        \addplot[-] expression[domain=-1:1,
          ClemsonOrange]{asin(x)*pi/180} node[custNode, above, pos=1]
          {$\sin\inv(x)$}; 
%%%%%%%%%%%%%%%%%%%%%%%%%%%%%%%%%%%%%%%%%%%%%%%%%%%%%%%%%%%%%%%%%%%%%%%%%%%%%%%
      \nextgroupplot[
        xmin=-1.5, xmax=pi+0.95,
        ymin=-1.5, ymax=pi+0.95,
        xtick={-1,1,pi},
        ytick={-1,1,pi},
        xticklabels={-1,1,$\pi$},
        yticklabels={-1,1,$\pi$},
        ]
        \addplot[dLine] coordinates {(-1,-limTwo) (-1,limTwo)};
        \addplot[dLine] coordinates {(1,-limTwo) (1,limTwo)};
        \addplot[dLine] coordinates {(-limTwo,1) (limTwo,1)};
        \addplot[dLine] coordinates {(-limTwo,-1) (limTwo,-1)};
        \addplot[-] expression[domain=0:pi,
          ClemsonPurple] {cos((x) r)} node[custNode, above right,
          pos=0.95] {$\cos(x)$};
        \addplot[-] expression[domain=-1:1,
          ClemsonOrange]{acos(x)*pi/180} node[custNode, above,
          pos=0.0, xshift=7.65pt, yshift=2.5pt] {$\cos\inv(x)$}; 
%%%%%%%%%%%%%%%%%%%%%%%%%%%%%%%%%%%%%%%%%%%%%%%%%%%%%%%%%%%%%%%%%%%%%%%%%%%%%%%
      \nextgroupplot[
        xmin=-limOne, xmax=limOne,
        ymin=-limOne, ymax=limOne,
        xtick={-pi/2,-1,1,pi/2},
        ytick={-pi/2,-1,1,pi/2},
        xticklabels={$-\frac{\pi}{2}$,-1,1,$\frac{\pi}{2}$},
        yticklabels={$-\frac{\pi}{2}$,-1,1,$\frac{\pi}{2}$},
        ]
        \addplot[dLine] coordinates {(-pi/2,-limTwo) (-pi/2,limTwo)};
        \addplot[dLine] coordinates {(pi/2,-limTwo) (pi/2,limTwo)};
        \addplot[dLine] coordinates {(-limTwo,pi/2) (limTwo,pi/2)};
        \addplot[dLine] coordinates {(-limTwo,-pi/2) (limTwo,-pi/2)};
        \addplot[-] expression[domain=-1.32:1.32,
          ClemsonPurple]{tan((x) r)} node[custNode, right, pos=0.95,
          xshift=7pt] {$\tan(x)$};
        \addplot[-] expression[domain=-limOne:limOne,
          ClemsonOrange]{atan(x)*pi/180} node[custNode, below right,
          pos=0.75] {$\tan\inv(x)$};
%%%%%%%%%%%%%%%%%%%%%%%%%%%%%%%%%%%%%%%%%%%%%%%%%%%%%%%%%%%%%%%%%%%%%%%%%%%%%%%
      \nextgroupplot[
        xmin=-limThree, xmax=limThree,
        ymin=-limThree, ymax=limThree,
        xtick={-1,1,pi},
        ytick={-1,1,pi},
        xticklabels={-1,1,$\pi$},
        yticklabels={-1,1,$\pi$},
        ]
        \addplot[dLine] coordinates {(0,-limTwo) (0,limTwo)};
        \addplot[dLine] coordinates {(pi,-limTwo) (pi,limTwo)};
        \addplot[dLine] coordinates {(-limTwo,0) (limTwo,0)};
        \addplot[dLine] coordinates {(-limTwo,pi) (limTwo,pi)};
        \addplot[-] expression[domain=0.25:3,
          ClemsonPurple]{cot((x) r)} node[custNode, right, pos=0.1,
          xshift=5pt] {$\cot(x)$};
        \addplot[-] expression[domain=-limThree:-0.0001, ClemsonOrange]{atan(1/x)*pi/180+pi};
        \addplot[-] expression[domain=0.0001:limThree,
          ClemsonOrange]{atan(1/x)*pi/180} node[custNode, above,
          pos=0.73] {$\cot\inv(x)$};
%%%%%%%%%%%%%%%%%%%%%%%%%%%%%%%%%%%%%%%%%%%%%%%%%%%%%%%%%%%%%%%%%%%%%%%%%%%%%%%        
      \nextgroupplot[
        xmin=-limTwo, xmax=limTwo,
        ymin=-limTwo, ymax=limTwo,
        xtick={-pi/2,-1,1,pi/2,pi},
        ytick={-pi/2,-1,1,pi/2,pi},
        xticklabels={$-\frac{\pi}{2}$,-1,1,$\frac{\pi}{2}$,$\pi$},
        yticklabels={$-\frac{\pi}{2}$,-1,1,$\frac{\pi}{2}$,$\pi$},
        ]
        \addplot[dLine] coordinates {(pi/2,-limTwo) (pi/2,limTwo)};
        \addplot[dLine] coordinates {(-limTwo,pi/2) (limTwo,pi/2)};
        \addplot[-] expression[domain=0:1.5,
          ClemsonPurple]{1/cos((x) r)} node[custNode, right,
          pos=0.225, xshift=7.5pt] {$\sec(x)$};
        \addplot[-] expression[domain=1.7:pi, ClemsonPurple]{1/cos((x) r)};
        \addplot[-] expression[domain=-limTwo:-1, ClemsonOrange]{acos(1/x)*pi/180};
        \addplot[-] expression[domain=1:limTwo,
          ClemsonOrange]{acos(1/x)*pi/180} node[custNode, below,
          pos=0.725] {$\sec\inv(x)$};
%%%%%%%%%%%%%%%%%%%%%%%%%%%%%%%%%%%%%%%%%%%%%%%%%%%%%%%%%%%%%%%%%%%%%%%%%%%%%%%
      \nextgroupplot[ 
        xmin=-limFour, xmax=limFour,
        ymin=-limFour, ymax=limFour,
        xtick={-pi/2,-1,1,pi/2,pi},
        ytick={-pi/2,-1,1,pi/2,pi},
        xticklabels={$-\frac{\pi}{2}$,-1,1,$\frac{\pi}{2}$,$\pi$},
        yticklabels={$-\frac{\pi}{2}$,-1,1,$\frac{\pi}{2}$,$\pi$},
        ]
        \addplot[dLine] coordinates {(0,-limTwo) (0,limTwo)};
        \addplot[dLine] coordinates {(-limTwo,0) (limTwo,0)};
        \addplot[-] expression[domain=-1.5:-0.25, ClemsonPurple]{1/sin((x) r)};
        \addplot[-] expression[domain=0.25:1.5,
          ClemsonPurple]{1/sin((x) r)} node[custNode, right,
          pos=0.225, xshift=7.5pt] {$\csc(x)$};
        \addplot[-] expression[domain=-limFour:-1, ClemsonOrange]{asin(1/x)*pi/180};
        \addplot[-] expression[domain=1:limFour,
          ClemsonOrange]{asin(1/x)*pi/180} node[custNode, above,
          pos=0.725] {$\csc\inv(x)$};
%%%%%%%%%%%%%%%%%%%%%%%%%%%%%%%%%%%%%%%%%%%%%%%%%%%%%%%%%%%%%%%%%%%%%%%%%%%%%%%
    \end{groupplot}
  \end{tikzpicture}
\end{center}

\vfill
  \pagebreak
  \begin{ex*}
    Solve the following:
  \end{ex*}
  \begin{tasks}[label=~](3)
    \task $\sin\inv\parens{-\frac{1}{\sqrt{2}}}$
    \task $\sec\inv(2)$
    \task $\cot\inv\parens{-\sqrt{3}}$
  \end{tasks}
  \vspace*{15pt}
  
\section{JIT 9.3: Inverse Trigonometric Identities}
  While $\sin(x)$ and $\sin\inv(x)$ are inverse functions, the inverse relationship only holds when working in the correct domains:
  \begin{tasks}[label=~](2)
    \task $\sin\inv\parens{\sin(\pi)}=\sin\inv(0)=0\neq \pi$
    \task $\sin\parens{\sin\inv(-1)}=\sin\parens{\nicefrac{-\pi}{2}}=-1$
  \end{tasks}
  \vspace*{15pt}
  \begin{ex*}
    Solve the following:
  \end{ex*}
    \begin{tasks}[after-item-skip=\stretch{1}, label=~](2)
      \task $\tan(\tan\inv(5))$
      \task $\tan\inv\parens{\tan\dfrac{3\pi}{4}}$
      \task $\cos\parens{\arcsin\dfrac{1}{2}}$
      \task $\cos\inv\parens{\cos(5\pi)}$
    \end{tasks}
  \vfill
  \pagebreak
  
  \begin{tasks}[after-item-skip=\stretch{1}, label=~](3)
    \task $\sin\inv\parens{\sin\parens{\dfrac{7\pi}{3}}}$
    \task $\tan\parens{\sec\inv(10)}$
    \task $\sin\parens{2\sin\inv\parens{\dfrac{3}{5}}}$
  \end{tasks}
    \vfill
  \begin{ex*}
    Simplify the following using triangles.
  \end{ex*}
  \begin{tasks}[after-item-skip=\stretch{1}, label=~](1)
    \task $\cos\parens{\tan\inv(x)}$
    \task $\sec\parens{\sin\inv(x)}$
    \task $\cos\parens{2\sin\inv(x)}$
    \task $\sin\parens{2\tan\inv(x)}$
  \end{tasks}
  \vfill
  \pagebreak
  
\end{document}
