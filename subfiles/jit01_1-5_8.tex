\documentclass[../mathNotesPreamble]{subfiles}
\begin{document}
  \section{JIT 1.1: Multiplying and Dividing Fractions}
  \begin{itemize}
    \item 
      When we multiply fractions, multiply the numerators together and multiply the denominators together:
      
      $$\dfrac{a}{b}\cdot\dfrac{c}{d}=\dfrac{a\cdot c}{b\cdot d}$$\\[\stretch{1}]
      
    \item 
      We can use cancellation so that simplifying the final answer is much simpler:
      
      $$\text{e.g. }\ \dfrac{30}{84}=\dfrac{2\cdot3\cdot5}{2\cdot2\cdot3\cdot7}=\dfrac{5}{2\cdot7}=\dfrac{5}{14}$$\\[\stretch{1}]
    \pagebreak
    \item 
      To divide by a fraction, invert and multiply (think of subtracting the negative):
      
      $$\dfrac{\dfrac{a}{b}}{\dfrac{c}{d}}=\dfrac{a}{b}\cdot\dfrac{d}{c}$$\\[\stretch{1}]
    \item 
      Types of numbers:
      \begin{itemize}
        \item 
          Real numbers: $\bbr$
          \begin{itemize}
            \item 
              Infinite number of digits
            \item 
              This is essentially all numbers except complex numbers.
          \end{itemize}
        \item
          Rational numbers: $\bbq$
          \begin{itemize}
            \item 
              Any number that can be written as a fraction.
          \end{itemize}
        \item
          Integers: $\bbz$
          \begin{itemize}
            \item
              $\set{\dots,-3,-2,-1,0,1,2,3,\dots}$
          \end{itemize}
      \end{itemize}\ 
      \\[\stretch{0.25}]\pagebreak
  \end{itemize}
  \section{JIT 1.2: Adding and Subtracting Fractions}
  \begin{itemize}
    \item 
      To add fractions together, the fractions must have common denominators, then we add the numerators.
      $$\dfrac{a}{b}+\dfrac{c}{b}=\dfrac{a+c}{b}\hspace*{100pt} \dfrac{a}{b}\!\parens{\dfrac{d}{d}}+\dfrac{c}{d}\!\parens{\dfrac{b}{b}}=\dfrac{ad+bc}{bd}$$\\[\stretch{1}]\pagebreak
  \end{itemize}
  \section{JIT 1.3: Parenthesis}
  \begin{itemize}
    \item 
      Parenthesis are computed first when following order of operations:       
      PEMDAS or BODMAS \ \;(GEMDAS)\\[\stretch{1}]
    \item 
      Distribution is multiplying all the terms on the inside of the parentheses
      \\[\stretch{1}]\pagebreak
  \end{itemize}
  \section{JIT 1.4: Exponents}
  \begin{enumerate}[label=\alph*)\ ]
    \item 
      $a^m\cdot a^n=a^{m+n}$
    \item 
      $\dfrac{a^m}{a^n}=a^{m-n}$
    \item 
      $\parens{a^m}^n=a^{m\cdot n}$
    \item 
      $\parens{a\cdot b}^n=a^n\cdot b^n$
    \item 
      $\parens{\dfrac{a}{b}}^n=\dfrac{a^n}{b^n}$
    \item 
      $\parens{\dfrac{a}{b}}\inv[n]=\parens{\dfrac{b}{a}}^n$
  \end{enumerate}
  \begin{ex*}\ 
  
    \begin{tasks}[after-item-skip=\stretch{1}](2)
      \task[] $\dfrac{1}{a^3}-\parens{\dfrac{1}{a^5}-\dfrac{1}{a^2}}$
      \task[] $\parens{\dfrac{x\inv[2]}{x^8}}\inv[2]$
      \task[] $\dfrac{\parens{x^3y\inv[2]}^6}{\parens{y\inv[5]x\inv[2]}\inv[3]}$
      \task[] $\parens{x\inv+y\inv}\inv$
    \end{tasks}
    \vspace*{\stretch{1}}
  \end{ex*}
  \pagebreak
  \section{JIT 1.5: Roots}
  \begin{enumerate}[label=\alph*)\ ]
    \item 
      $\sqrt[n]{a}=a^{\sfrac{1}{n}}$
    \item 
      $\sqrt[n]{a^m}=a^{\sfrac{m}{n}}$
  \end{enumerate}
  \begin{ex*}\ 
  
    \begin{tasks}[after-item-skip=\stretch{1}](2)
      \task[] $8^{\sfrac{2}{3}}$
      \task[] $\parens{\dfrac{-1}{27}}^\frac{4}{3}$
      \task[] $\parens{-32}^\frac{4}{5}$
      \task[] $\parens{0.008}^\frac{4}{3}$
      \task[] $\parens{81x^2-4y^2}\inv[\sfrac12]$
      \task[] $\dfrac{2^{\sfrac47}}{2^{\sfrac32}}$
    \end{tasks}
    \vspace*{\stretch{1}}
  \end{ex*}
  \pagebreak
  
  \section{JIT 1.8: Intervals}
    \begin{itemize}
      \item An \textit{open interval}, denoted $\parens{a,b}$, is a set containing all numbers between $a$ and $b$ and excludes the endpoints $a$ and $b$. This can be represented mathematically as $a<x<b$, or graphically using a number line:
        \begin{center}
          \begin{tikzpicture}[scale=1.0]
              \begin{axis}[
                axis line style={<->},
                xticklabel style = {yshift=-2pt},
                axis y line=none,
                axis x line*=center,
                ymin=0, ymax=1,
                xmin=-2, xmax=6,
                width=2.75in, height=0.75in,
                xtick={0,4},
                xticklabels={$a$, $b$},
                ticklabel style={font=\normalsize,inner sep=0.5pt,fill=white,opacity=1.0, text opacity=1, yshift=-2.5pt},
                every axis plot/.append style={blue,line width=1.5pt}
                ]
                \addplot[-] expression[domain=0:4]{0};
                \addplot[holdot] coordinates{(0,0)(4,0)};
              \end{axis}
            \end{tikzpicture}
        \end{center}
      \item A \textit{closed interval}, denoted $\sbrkt{a,b}$, is a set containing all numbers between $a$ and $b$ and includes the endpoints $a$ and $b$. This can be represented mathematically as $a\leq x\leq b$, or graphically using a number line:
        \begin{center}
          \begin{tikzpicture}[scale=1.0]
              \begin{axis}[
                axis line style={<->},
                xticklabel style = {yshift=-2pt},
                axis y line=none,
                axis x line*=center,
                ymin=0, ymax=1,
                xmin=-2, xmax=6,
                width=2.75in, height=0.75in,
                xtick={0,4},
                xticklabels={$a$, $b$},
                ticklabel style={font=\normalsize,inner sep=0.5pt,fill=white,opacity=1.0, text opacity=1, yshift=-2.5pt},
                every axis plot/.append style={blue,line width=1.5pt}
                ]
                \addplot[-] expression[domain=0:4]{0};
                \addplot[soldot] coordinates{(0,0)(4,0)};
              \end{axis}
            \end{tikzpicture}
          \end{center}
    \end{itemize}
    \begin{defn*}
      Let $A$ and $B$ be two sets of objects of any sort.
      \begin{enumerate}
        \item The set of all objects that are in \textit{both} $A$ and $B$ is called $A$ \textit{intersection} $B$ and is denoted $A\cap B$.
        \item The set of all objects that are in \textit{either} $A$ or $B$ or both is called $A$ \textit{union} $B$ and is denoted $A\cup B$.
        \item The set that contains no elements is called the \textit{empty set} and is written $\emptyset$.
      \end{enumerate}
    \end{defn*}
  \pagebreak
\end{document}
% ----------------------------------------------------------------