\documentclass[answers]{exam}
\usepackage{texPreamble}
\usepackage{relsize}
\usepackage{tabularx}
\extraheadheight{0.25in}
\extrafootheight{1.0in}
\extrawidth{1in}
% ----------------------------------------------------------------

\begin{document}
  \section{2.3: Techniques for Computing Limits}
    \begin{ex*}\ 
    
      \begin{tasks}(2)
        \task $\ds\lim_{x\to 3} \frac{1}{2}x-7$
        \task $\ds\lim_{x\to 2} 6$
      \end{tasks}
      \vspace*{\stretch{1}}
    \end{ex*}
    \begin{defn*}[Briggs]
      \textbf{Limit Laws:} Assume $\ds\lim_{x\to a} f(x)$ and $\ds\lim_{x\to a} g(x)$ exist. The following properties hold, where $c$ is a real number, and $n>0$ is an integer.
      
    \begin{center}
      \def\arraystretch{2.5}
      \begin{tabular}{llL}
        1.&\textbf{Sum:}
          &\ds\lim_{x\to a}\parens{f(x)+g(x)}=\lim_{x\to a}f(x)+\lim_{x\to a}g(x)\\
        2.&\textbf{Difference:}
          &\ds\lim_{x\to a}\parens{f(x)-g(x)}=\lim_{x\to a}f(x)-\lim_{x\to a}g(x)\\
        3.&\textbf{Constant multiple:}
          &\ds\lim_{x\to a}{cf(x)}=c\lim_{x\to a}f(x)\\
        4.&\textbf{Product:}
          &\ds\lim_{x\to a}{f(x)g(x)}=\parens{\lim_{x\to a}f(x)}\parens{\lim_{x\to a}g(x)}\\
        5.&\textbf{Quotient:}
          &\ds\lim_{x\to a}{\frac{f(x)}{g(x)}}=\frac{\lim_{x\to a}f(x)}{\lim_{x\to a}g(x)}, \text{ provided} \lim_{x\to a} g(x)\neq 0\\
        6.&\textbf{Power:}
          &\lim_{x\to a}\parens{f(x)}^n=\parens{\lim_{x\to a}f(x)}^n\\
        7.&\textbf{Root:}
          & \lim_{x\to a}\parens{f(x)}^{\sfrac{1}{n}}=\parens{\lim_{x\to a} f(x)}^{\sfrac{1}{n}}
       \end{tabular}
     \end{center}
    \end{defn*}
    \pagebreak
    \begin{ex*}
      Suppose $\ds \lim_{x\to 2} f(x)=4, \lim_{x\to 2} g(x)=5$ and $\ds\lim_{x\to 2}h(x)=8$. 
      \begin{tasks}(1)
        \task $\ds\lim_{x\to 2} \frac{f(x)-g(x)}{h(x)}$\\[15pt]
        \task $\ds\lim_{x\to 2} \parens{6f(x)g(x)+h(x)}$\\[15pt]
        \task $\ds\lim_{x\to 2} \parens{g(x)}^3$
      \end{tasks}
      \vspace*{\stretch{1}}
    \end{ex*}
    \begin{ex*}
      For $g(x)=\dfrac{x+6}{x^2-36}$, find
      \begin{enumerate}
        \item $\ds\lim_{x\to 0} g(x)$\\[25pt]
        \item $\ds\lim_{x\to -6} g(x)$
      \end{enumerate}
      \vspace*{\stretch{1}}
    \end{ex*}
    \pagebreak
    \begin{ex*}
      $\ds\lim_{x\to 2} \frac{\sqrt{2x^3+9}+3x-1}{4x+1}$
      \vspace*{\stretch{1}}
    \end{ex*}
    \begin{ex*}
      $\ds\lim_{x\to 1} f(x)$ where $f(x)=\begin{cases}
        -2x+4& \text{if }x\leq 1\\
        \sqrt{x-1}& \text{if } x>1
      \end{cases}$
      \vspace*{\stretch{1}}
    \end{ex*}
    \pagebreak
    \begin{ex*}
      $\ds\lim_{x\to 2} \frac{x^2-6x+8}{x^2-4}$
      \vspace*{\stretch{1}}
    \end{ex*}
    \begin{ex*}
      $\ds\lim_{x\to 1} \frac{\sqrt x-1}{x-1}$
      \vspace*{\stretch{1}}
    \end{ex*}
    \begin{ex*}
      $\ds\lim_{x\to -4} \sqrt{16-x^2}$
      \vspace*{\stretch{1}}
    \end{ex*}
    \pagebreak
    \begin{ex*}
      $\ds\lim_{x\to 2}\frac{x^3-6x^2+8x}{\sqrt{x-2}}$
      \vspace*{\stretch{1}}
    \end{ex*}
    \begin{ex*}
      $\ds\lim_{y\to a} \frac{\parens{y-a}^{12}+6y-6a}{y-a}$
      \vspace*{\stretch{1}}
    \end{ex*}
    \pagebreak
    
    \noindent
    \textbf{The Squeeze Theorem:} Assume the functions $f,g$ and $h$ satisfy $f(x)\leq g(x)\leq h(x)$ for all values of $x$ near $a$, except possibly at $a$. If $\lim_{x\to a}f(x)= \lim_{x\to a}h(x)=L$, then $\lim_{x\to a} g(x)=L$.
    \begin{ex*}
      Consider the function $f(x)=x^2 \sin\parens{\sfrac{1}{x}}$. What is $\ds\lim_{x\to 0}f(x)$?
      \vspace*{\stretch{1}}
    \end{ex*}
    \begin{ex*}
      Use the squeeze theorem on $-\abs x\leq x\sin \frac{1}{x}\leq \abs x$.
      \vspace*{\stretch{1}}
    \end{ex*}
    \pagebreak
    \begin{ex*}
      $\ds\lim_{x\to 0} \frac{\sin^2 x}{1-\cos x}$
      \vspace*{\stretch{1}}
    \end{ex*}
    \begin{ex*}
      $\ds\lim_{x\to 0} \frac{1-\cos 2x}{\sin x}$
      \vspace*{\stretch{1}}
    \end{ex*}
    \pagebreak
\end{document}
