\documentclass[mathNotesPreamble]{subfiles}
\begin{document}
%\relscale{1.4} %TODO
\section{14.2: Calculus of Vector-Valued Functions}
  \begin{defn*}[Derivative and Tangent Vector]
    Let $\vecr(t)=f(t)\bfi+g(t)\bfj+h(t)\bfk$, where $f$,$g$, and $h$ are differentiable functions on $(a,b)$. Then $\vecr$ has a \textbf{derivative} (or is \textbf{differentiable}) on $(a,b)$ and
      \[\vecr'(t)=f'(t)\bfi+g'(t)\bfj+h'(t)\bfk.\]
    Provided $\vecr'(t)\neq \bfO$, $\vecr'(t)$ is a \textbf{tangent vector} at the point corresponding to $\vecr(t)$.
  \end{defn*}
  \begin{ex*}
    For the following functions below, find $\vecr'(t)$
  \end{ex*}
  \begin{tasks}[after-item-skip=\stretch{1}](1)
    \task $\vecr(t)=\bracket{e^{-t^2},\,\log_2(t-4),\,\sin(t)}$
    \task $\vecr(t)=3\bfi-2\tan(t)\bfj+e^t\bfk$
  \end{tasks}
  \vspace*{\stretch{1}}
  \begin{ex*}
    For $\vecr(t)=\bracket{3t,\,\sec(2t),\,\cos(t)}$ compute $\vecr'(t)$ at $t=\frac{\pi}{4}$.
  \end{ex*}
  \vspace*{\stretch{1}}
  \pagebreak
  
  \begin{defn*}[Unit Tangent Vector]
    Let $\vecr(t)=f(t)\bfi+g(t)\bfj+h(t)\bfk$ be a smooth parameterized curve, for $a\leq t\leq b$. The \textbf{unit tangent vector} for a particular value of $t$ is
      \[\mathbf T(t)=\frac{\vecr'(t)}{\abs{\vecr'(t)}}.\]
  \end{defn*}
  \begin{ex*}
    For $\vecr(t)=\bracket{3\sin(t),\,-2\cos(2t),\,3\cos(t)}$, find the unit tangent vector.
  \end{ex*}
  \vspace*{\stretch{1}}
  \begin{ex*}
    For $\vecr(t)=\bracket{\sin(6t),\,3t,\,\cos(3t)}$, compute $\mathbf T\parens{\frac{\pi}{3}}$.
  \end{ex*}
  \vspace*{\stretch{1}}
  \pagebreak

  \noindent
  \fbox{\parbox{0.9875\linewidth}{
    \textbf{Derivative Rules}\\
    Let $\vecu$ and $\vecv$ be differentiable vector-valued functions, and let $f$ be a differentiable scalar-valued function, all at a point $t$. Let $\mathbf c$ be a constant vector. The following rules apply.
    \begin{enumerate}
      \TabPositions{0.6\linewidth}
      \item $\displaystyle \ddt(\mathbf c)=\bfO$ \tab \textcolor{blue}{Constant Rule}
      \item $\displaystyle \ddt\parens{\vecu(t)+\vecv(t)}=\vecu'(t)+\vecv'(t)$ \tab \textcolor{blue}{Sum Rule}
      \item $\displaystyle \ddt\parens{f(t)\vecu(t)}=f'(t)\vecu(t)+f(t)\vecu'(t)$ \tab \textcolor{blue}{Product Rule}
      \item $\displaystyle \ddt\parens{\vecu(f(t))}=\vecu'\parens{f(t)}f'(t)$ \tab \textcolor{blue}{Chain Rule}
      \item $\displaystyle \ddt\parens{\vecu(t)\cdot\vecv(t)}=\vecu'(t)\cdot\vecv(t)+\vecu(t)\cdot\vecv'(t)$ \tab \textcolor{blue}{Dot Product Rule}
      \item $\displaystyle \ddt\parens{\vecu(t)\times\vecv(t)}=\vecu'(t)\times\vecv(t)+\vecu(t)\times\vecv'(t)$ \tab \textcolor{blue}{Cross Product Rule}
    \end{enumerate}
  }}
  \begin{ex*}
    Given $\vecu(t)=\bracket{4t^2,\,1,\,3t}$ and $\vecv(t)=\bracket{e^{-2t},\,-2e^t,\,e^t}$, find $\displaystyle \ddt\sbrkt{\vecu(t)\cdot\vecv(t)}$.
  \end{ex*}
  \pagebreak
  
  \begin{defn*}[Indefinite Integral of a Vector-Valued Function]
    Let $\vecr(t)=f(t) \bfi+g(t) \bfj+h(t) \bfk$ be a vector function, and let \newline $\mathbf R(t)=F(t) \bfi+G(t) \bfj+H(t) \bfk$, where $F$, $G$, and $H$ are antiderivatives of $f$, $g$, and $h$, respectively. The \textbf{indefinate integral} of $\vecr$ is
      \[\int\vecr(t)\,dt=\mathbf R(t)+\mathbf C,\]
    where $\mathbf C$ is an arbitrary constant vector. Alternatively, in component form,
      \[\int\bracket{f(t),\,g(t),\,h(t)}\,dt=\bracket{F(t),\,G(t),\,H(t)}+\bracket{C_1,\,C_2,\,C_3}.\]
  \end{defn*}
  \begin{ex*}
    Find $\vecr(t)$ such that $\vecr'(t)=\bracket{\frac{t}{t^2+1},\,t^2e^{-t^3},\,\frac{-2t}{\sqrt{t^2+16}}}$ and $\vecr(0)=\bracket{3,\,\frac{5}{3},\,-5}$.
  \end{ex*}
  \vspace*{\stretch{1}}
  \pagebreak

  \begin{defn*}[Definite Integral of a Vector-Valued Function]
    Let $\vecr(t)=f(t)\bfi+g(t)\bfj+h(t)\bfk$, where $f$, $g$, and $h$ are integrable on the interval $\sbrkt{a,b}$. The \textbf{definite integral} of $\vecr$ on $\sbrkt{a,b}$ is
      \[\int_a^b \vecr(t)\,dt=\parens{\int_a^b f(t)\,dt}\bfi+\parens{\int_a^b g(t)\,dt}\bfj+\parens{\int_a^b h(t)\,dt}\bfk\]
  \end{defn*}
  \begin{ex*}
    $\displaystyle \int_{-\pi}^{\pi} \bracket{\sin(t),\,\cos(t),\, 8t}\,dt$
  \end{ex*}
  \vspace*{\stretch{1}}
  \pagebreak

  
\end{document}